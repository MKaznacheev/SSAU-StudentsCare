\subsection{6}

Длина тонкого однородного стержня равна $l=0{,}5\;\text{м}$, а его масса --- $m=0{,}4\;\text{кг}$. Он вращается около оси, проходящей через его середину перпендикулярно ему, с ускорением $\varepsilon=3\;\text{рад/с}^2$. Поскольку в данном случае момент инерции остаётся постоянным и равным
\[
I=\frac{ml^2}{12},
\]
то справедливо уравнение
\[
M=I\varepsilon=\frac{ml^2}{12}\varepsilon=0{,}025\;\text{Н}\cdot\text{м}.
\]