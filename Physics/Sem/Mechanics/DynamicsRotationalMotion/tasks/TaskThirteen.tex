\subsection{18}

Радиус шара равен $r=0{,}2\;\text{м}$, а сферы --- $R=0{,}5\;\text{м}$. По закону сохранения энергии
\[
mgh=\frac{mv^2}{2}+\frac{I\omega^2}{2}=\frac{\left(mr^2+I\right)v^2}{2r^2}\LR v=r\sqrt{\frac{2mgh}{mr^2+I}}.
\]
При этом $I=\dfrac{2mr^2}{5}$, а значит,
\[
v=\sqrt{\frac{10gh}{7}}.
\]
При этом
\[
\frac{mv^2}{R+r}=mg\cos\alpha\LR\cos\alpha=\frac{10}{7(R+r)}h,
\]
где $\alpha$ --- угол между линией соединяющей центры шара и сферы и вектором силы тяжести. При этом
\[
h=(R+r)-(R+r)\cos\alpha=(R+r)-\frac{10}{7}h\LR h=\frac{7}{17}(R+r).
\]
Тогда
\[
v=\sqrt{\frac{10g(R+r)}{17}}
\]
и
\[
\omega=\frac{v}{r}=\frac{1}{r}\sqrt{\frac{10g(R+r)}{17}}\approx10{,}04\;\text{рад/с}.
\]