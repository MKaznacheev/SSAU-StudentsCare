\subsection{13}

Момент инерции диска относительно оси вращения равен $\dfrac{MR^2}{2}$, а валика --- $\dfrac{mr^2}{2}$. Момент инерции системы равен сумме моментов диска и валика:
\[
I=\frac{MR^2}{2}+\frac{mr^2}{2}=\frac{1}{2}\left(MR^2+mr^2\right).
\]
При этом,
\[
M=2Tr
\]
и
\[
M=I\varepsilon.
\]
Поскольку
\[
\varepsilon=\frac{a}{r},
\]
то
\[
I\frac{a}{r}=2Tr\LR T=\frac{a}{4r^2}\left(MR^2+mr^2\right).
\]
По второму закону Ньютона в проекции на вертикальную ось, центр масс движется в соответствии с уравнением
\[
(m+M)g-2T=(m+M)a\LR a=\frac{2(m+M)gr^2}{M\left(R^2+2r^2\right)+3mr^2}.
\]