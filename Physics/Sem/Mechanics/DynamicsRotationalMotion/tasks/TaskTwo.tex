\subsection{7}

Для диска площади $S=\pi R^2$ истинно соотношение
\[
M=I\varepsilon,
\]
поскольку $I=\text{const}$. При этом на некоторую элемент диска массы $dm$ и радиуса $r$ действует сила трения
\[
dF=\mu g\,dm.
\]
В силу однородности диска,
\[
\frac{dm}{ds}=\frac{m}{S}\LR dm=\frac{m}{S}\,ds=\frac{m}{\pi R^2}\,d(\pi r^2)=\frac{2mr}{R^2}\,dr.
\]
Тогда
\[
dF=\frac{2\mu mgr}{R^2}\,dr.
\]
Значит,
\[
dM=rdF=\frac{2\mu mgr^2}{R^2}\,dr
\]
и
\[
M=\int\limits_0^R\frac{2\mu mgr^2}{R^2}\,dr=\frac{2\mu mgR}{3}.
\]
При этом
\[
I=\frac{mR^2}{2},\quad\varepsilon=\frac{\omega_0}{\tau},
\]
где $\tau$ --- время до остановки диска. Наконец,
\[
\frac{2\mu mgR}{3}=\frac{mR^2}{2}\cdot\frac{\omega_0}{\tau}\LR\tau=\frac{3\omega_0R}{4\mu g}.
\]