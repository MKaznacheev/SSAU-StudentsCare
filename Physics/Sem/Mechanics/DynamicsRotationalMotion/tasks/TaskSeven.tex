\subsection{12}

Масса тонкого стержня длины $l=1\;\text{м}$ равна $m=1\;\text{кг}$. Ось вращения проходит через нижнюю точку стержня перпендикулярно плоскости, в которой происходит его движение. В некоторый момент времени стержень составляет с вертикалью угол $\beta=60^\circ$. Момент инерции некоторого элемента стержня равен
\[
dI=r^2\,dm.
\]
Ввиду однородности стержня,
\[
\frac{dm}{dr}=\frac{m}{l}\LR dm=\frac{m}{l}\,dr
\]
и
\[
dI=\frac{mr^2}{l}\,dr\LR I=\frac{m}{l}\int\limits_0^l r^2\,dr=\frac{ml^2}{3}
\]
Тогда момент импульса всего стержня равен
\[
L=\frac{ml^2}{3}\omega.
\]

При этом
\[
M=\dot L=I\varepsilon
\]
с одной стороны и
\[
M=\bigl|[\vec R_C, m\vec g]\bigr|=\frac{mgl}{2}\sin\varphi
\]
--- с другой, где $\varphi$ --- угол между стержнем и вертикалью. Тогда
\[
\frac{mgl}{2}\sin\varphi=I\varepsilon\LR\varepsilon=\frac{mgl}{2I}\sin\varphi=\frac{3g}{2l}\sin\varphi.
\]
При этом,
\[
\varepsilon=\frac{d\omega}{dt}=\frac{\omega\,d\omega}{d\varphi}.
\]
Тогда
\[
\omega\,d\omega=\frac{3g}{2l}\sin\varphi\,d\varphi\LR\omega^2=C-\frac{3g}{l}\cos\varphi.
\]
Поскольку $\omega^2(0)=0$, то
\[
C=\frac{3g}{l}
\]
и
\[
\omega=\sqrt{\frac{3g}{l}(1-\cos\varphi)}.
\]

Наконец,
\[
L=\frac{ml^2}{3}\sqrt{\frac{3g}{l}(1-\cos\varphi)}=m\sqrt{\frac{gl^3}{3}(1-\cos\varphi)}=1{,}28\;\text{кг$\cdot$м$^2$/с}.
\]