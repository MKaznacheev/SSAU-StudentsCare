\subsection{15}

На блок снизу и тело массой $m_1$ сверху действуют силы натяжения нити равные $T_1$. При этом на тело массой $m_1$ снизу и тело массой $m_2$ сверху действуют силы натяжения нити равные $T_2$. Запишем второй закон Ньютона для тела массой $m_2$ в проекции на вертикальную ось:
\[
m_2g-T_2=m_2a\LR T_2=m_2(g-a).
\]
Аналогично для тела массой $m_1$:
\[
m_1g+T_2-T_1=m_1a\LR T_1=(m_2+m_1)(g-a).
\]
Для блока имеем
\[
M=I\varepsilon=T_1R.
\]
При этом
\[
I=\frac{MR^2}{2}
\]
и
\[
\varepsilon=\frac{a}{R}.
\]
Тогда
\[
\frac{MR^2}{2}\cdot\frac{a}{R}=(m_2+m_1)(g-a)R\LR a=\frac{2(m_2+m_1)}{M+2m_2+2m_1}g.
\]
Значит,
\[
T_1=\frac{M(m_2+m_1)}{M+2m_2+2m_1}g
\]
и
\[
T_2=\frac{Mm_2}{M+2m_2+2m_1}g.
\]