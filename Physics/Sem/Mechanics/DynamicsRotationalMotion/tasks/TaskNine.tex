\subsection{14}

Радиус маховика, раскрученного до скорости
\[
\omega_0=400\;\text{об/мин}\approx41{,}9\;\text{рад/с},
\]
составляет $r=0{,}4\;\text{м}$. Его масса равна $m=100\;\text{кг}$. Остановка произошла через $\tau=80\;\text{с}$. Угловое ускорение в таком случае составляет
\[
\varepsilon=\frac{\omega_0}{\tau}.
\]
Момент инерции равен
\[
I=\frac{mr^2}{2}.
\]
Тогда момент сил трения равен
\[
M=I\varepsilon=\frac{mr^2\omega_0}{2\tau}=4{,}19\;\text{Н$\cdot$м}.
\]