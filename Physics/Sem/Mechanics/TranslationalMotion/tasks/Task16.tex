\subsection{16}

Движение материальной точки в плоскости $XY$ описывается законом $x=At$, $y=At(1+Bt)$, где $A$ и $B$ --- положительные постоянные. Определим уравнение траектории, а так же радиус-вектор, скорость и ускорение точки в зависимости от времени.

Выразим время через соответствующее ему значение абсциссы:
\[
t=\frac{x}{A}.
\]
Исходя из этого выражения, найдём уравнение траектории:
\[
y(x)=A\cdot\frac{x}{A}\cdot\left(1+B\cdot\frac{x}{A}\right)=x+\frac{Bx^2}{A}.
\]

Выражение радиус-вектор точки через единичные базисные векторы плоскости имеет вид $\vec r=x\vec i+y\vec j$. Скорость равна
\[
\vec v=\frac{d\vec r}{dt}=A\vec i+A(1+2Bt)\vec j,
\]
а ускорение ---
\[
\vec a=\frac{d\vec v}{dt}=2AB\vec j.
\]
Модули этих величин вычисляются по формулам
\begin{gather*}
r=\sqrt{x^2+y^2}=\sqrt{A^2t^2+A^2t^2(1+Bt)^2}=At\sqrt{1+(1+Bt)^2}, \\
v=\sqrt{v_x^2+v_y^2}=\sqrt{A^2+A^2(1+2Bt)^2}=A\sqrt{1+(1+2Bt)^2}
\end{gather*}
и
\[
a=|2AB|=2AB.
\]