\subsection{16}

Материальная точка движется в плоскости по закону
\[
x=At,\quad y=At(1+Bt),
\]
где $A=\text{const}>0$ и $B=\text{const}>0$. Исходя из выражения
\[
t=\frac{x}{A}
\]
найдём уравнение траектории:
\[
y(x)=A\cdot\frac{x}{A}\cdot\left(1+B\cdot\frac{x}{A}\right)=x+\frac{Bx^2}{A}.
\]

Радиус-вектор точки примет вид
\[
\vec r=x\vec i+y\vec j=At\vec i+At(1+Bt)\vec j,
\]
а его модуль ---
\[
r=\sqrt{x^2+y^2}=\sqrt{A^2t^2+A^2t^2(1+Bt)^2}=At\sqrt{1+(1+Bt)^2}.
\]

Вектор скорости точки будет равняться
\[
\vec V=\frac{d\vec r}{dt}=A\vec i+A(1+2Bt)\vec j,
\]
а её модуль ---
\[
V=\sqrt{V_x^2+V_y^2}=\sqrt{A^2+A^2(1+2Bt)^2}=A\sqrt{1+(1+2Bt)^2}.
\]

Теперь найдём вектор ускорения:
\[
\vec a=\frac{d\vec V}{dt}=2AB\vec j.
\]
Очевидно при этом, что
\[
a=2AB.
\]