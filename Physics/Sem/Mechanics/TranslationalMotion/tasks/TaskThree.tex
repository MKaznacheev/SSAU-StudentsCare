\subsection{11}

Автомобили движутся по прямой в одну сторону, начав движение одновременно из одного и того же пункта. Зависимости путей от времени для них выглядят следующим образом:
\[
s_1=Ft+Bt^2,\quad s_2=Ct+Dt^2+At^3.
\]
Тогда выражения их скоростей принимают вид
\[
v_1=\frac{ds_1}{dt}=F+2Bt
\]
и
\[
v_2=\frac{ds_2}{dt}=C+2Dt+3At^2.
\]
Обозначим скорость второго автомобиля в системе координат, связанной с первым, как
\[
\vec u=\vec v_2-\vec v_1.
\]
Поскольку векторы $\vec v_1$ и $\vec v_2$ коллинеарны, то
\[
u=\left|\vec v_2-\vec v_1\right|=v_2-v_1=C-F+2(D-B)t+3At^2.
\]