\subsection{20}

Положение точки на прямой в зависимости от времени дается уравнением $x=-9t+6t^2$, $[x]=\text{м}$. Найдём средние скорости перемещения $\langle v_1\rangle$ на интервале от $t_1=0\;\text{с}$ до $t_2=1\;\text{с}$ и $\langle v_2\rangle$ на интервале $t_2=1\;\text{с}$ до $t_3=2\;\text{с}$.

За временной интервал $t_1\div t_2$ точка совершила перемещение $\vec r(t_2)-\vec r(t_1)$. В терминах известных нам проекций, запишем
\[
r_{12}=x(t_2)-x(t_1)=x(t_2)=-9t_2+6t_2^2=-3\;\text{м},
\]
поскольку $x(t_1)=0\;\text{м}$. Поскольку $t_2$ --- время движения точки, то проекция средней скорости за это время равна
\[
\langle v_1\rangle=\frac{r_{12}}{t_2}=-3\;\text{м/с}.
\]

За временной интервал $t_2\div t_3$ точка совершила перемещение $\vec r(t_3)-\vec r(t_2)$. В терминах известных нам проекций, запишем
\[
r_{23}=x(t_3)-x(t_2)=-9(t_3-t_2)+6(t_3^2-t_2^2)=9\;\text{м}.
\]
Поскольку $t_3-t_2$ --- время движения точки, то проекция средней скорости за это время равна
\[
\langle v_2\rangle=\frac{r_{23}}{t_3-t_2}=9\;\text{м/с}.
\]