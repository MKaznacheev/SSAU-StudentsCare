\subsection{15}

Частица движется в плоскости $XY$ из точки с координатами $x_0=y_0=0$ со скоростью $\vec v=a\vec i+bt\vec j$, где $a$ и $b$ --- некоторые постоянные. Найдём уравнение $y(x)$ её траектории.

Получим зависимость радиус-вектора от времени:
\[
\vec r=\int\vec v\,dt=a\vec i\int dt+b\vec j\int t\,dt=at\vec i+\frac{bt^2}{2}\vec j+\vec r_0
\]
(можно убедится, что константа интегрирования равна $\vec r_0$, то есть радиус-вектору в начальный момент времени, подставив в выражение $\vec r$ значение $t=0\;\text{с}$). Конечно, $\vec r=x\vec i+y\vec j$, а значит $\vec r_0=x_0\vec i+y_0\vec j=\vec0$. Тогда
\[
\vec r=at\vec i+\frac{bt^2}{2}\vec j.
\]
Иначе говоря,
\[
x(t)=at,\quad y(t)=\frac{bt^2}{2}.
\]
Выразив $t$ через $x$, подставим полученное в $y(t)$:
\[
y(x)=\cfrac{b\left(\cfrac{x}{a}\right)^2}{2}=\frac{bx^2}{2a^2}.
\]