\subsection{5}

Материальная точка движется в плоскости согласно уравнениям $x(t)=5+3t+4t^2$ и $y(t)=2-3t+6t^2$, $[x]=[y]=\text{м}$. Найдём модули скорости и ускорения точки в момент времени $t_0=5\;\text{с}$.

Радиус вектор $\vec r=\vec r(t)$ точки может быть полностью описан двумя координатами $x=x(t)$ и $y=y(t)$:
\[
\vec r=x\vec i+y\vec j,
\]
где $\vec i$ и $\vec j$ --- единичные базисные векторы прямоугольной декартовой системы координат. В нашем случае, формула для нахождения скорости принимает вид
\[
\vec v=\frac{d\vec r}{dt}=\dot x\vec i+\dot y\vec j=(8t+3)\vec i+(12t-3)\vec j.
\]
По формуле
\[
\vec a=\frac{d\vec v}{dt}=\ddot x\vec i+\ddot y\vec j=8\vec i+12\vec j
\]
можно найти ускорение. Как мы видим, ускорение постоянно и в любой момент времени принимает значение
\[
a=\sqrt{8^2+12^2}=4\sqrt{13}\approx14{,}4\;\text{м/с}^2.
\]
Скорость в момент времени $t_0$ равна
\[
\vec v_0=(8t_0+3)\vec i+(12t_0-3)\vec j=43\vec i+57\vec j.
\]
Тогда по модулю
\[
v_0=\sqrt{43^2+57^2}=\sqrt{5098}\approx71{,}4\;\text{м/с}.
\]