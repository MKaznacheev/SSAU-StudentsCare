\subsection{12}

Известно, что первую половину времени движения тело двигалось со скоростью $v_1=40\;\text{км/ч}$, а вторую --- со скоростью $v_2=60\;\text{км/ч}$. Полное время движения обозначим через $t$. Тогда путь, пройденный за первую половину времени составляет
\[
s_1=v_1\cdot\frac{t}{2},
\]
а за вторую ---
\[
s_2=v_2\cdot\frac{t}{2}.
\]
Средняя скорость тела в таком случае равна
\[
v_\text{ср}=\frac{s_1+s_2}{t}=\cfrac{v_1\cdot\cfrac{t}{2}+v_2\cdot\cfrac{t}{2}}{t}=\frac{v_1+v_2}{2}=\frac{40+60}{2}=50\;\text{км/ч}.
\]