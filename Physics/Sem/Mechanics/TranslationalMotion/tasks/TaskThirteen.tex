\subsection{21}

Движение материальных точек происходит в соответсвии с уравнениями
\[
x_1=20+4t-4t^2,\quad x_2=2+t+0{,}5t^2.
\]
Очевидно, что
\[
s_1=|x_1-20|=\left|4t-4t^2\right|\Rightarrow
\left[
\begin{NiceArray}{ll}
	s_1=4t^2-4t, & \text{при $t\Geq1$} \\
	s_1=4t-4t^2, & \text{при $0\Leq t<1$}
\end{NiceArray}
\right.
\]
и
\[
s_2=|x_2-2|=\left|t+0{,}5t^2\right|=t+0{,}5t^2.
\]
Тогда
\[
v_1=\left|\frac{ds_1}{dt}\right|=
\left[
\begin{NiceArray}{ll}
	8t-4, & \text{при $t\Geq1$} \\
	|4-8t|, & \text{при $0\Leq t<1$}
\end{NiceArray}
\right.
,\quad a_1=\left|\frac{dv_1}{dt}\right|=8\;\text{м/с}^2
\]
и
\[
v_2=\frac{ds_1}{dt}=1+t,\quad a_2=\frac{dv_1}{dt}=1\;\text{м/с}^2.
\]
Вычислим момент времени, когда скорости точек совпадут:
\[
v_1=v_2\LR8t_0-4=1+t_0\LR t_0=\frac{5}{7}\approx0{,}71\;\text{с}.
\]
При этом
\[
v_1(t_0)=8\cdot0{,}71-4=1{,}68\;\text{м/с},\quad v_2(t_0)=1+0{,}71=1{,}71\;\text{м/с}
\]
и
\[
a_1(t_0)=8\;\text{м/с}^2,\quad a_2(t_0)=1\;\text{м/с}^2.
\]