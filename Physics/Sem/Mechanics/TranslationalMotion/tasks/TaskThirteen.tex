\subsection{21}

Движение материальных точек происходит в соответсвии с уравнениями
\[
x_1=20+4t-4t^2,\quad x_2=2+t+0{,}5t^2.
\]
Очевидно, что
\[
\vec r_1=x_1\vec i
\]
и
\[
\vec r_2=x_2\vec i,
\]
где $i=1$. Тогда
\[
\vec v_1=\frac{d\vec r_1}{dt}=(4-8t)\vec i,\quad\vec a_1=\frac{d\vec v_1}{dt}=-8\vec i
\]
и
\[
\vec v_2=\frac{d\vec r_2}{dt}=(1+t)\vec i,\quad \vec a_2=\frac{d\vec v_2}{dt}=\vec i.
\]
Вычислим момент времени, когда скорости точек совпадут:
\[
\vec v_1=\vec v_2\LR(4-8t)\vec i=(1+t)\vec i\LR4-8t=1+t\LR t=\frac{1}{3}\approx0{,}33\;\text{с}.
\]
При этом
\[
v_1(t_0)=v_2(t_0)=1+\frac{1}{3}=\frac{4}{3}\approx1{,}33\;\text{м/с}.
\]
Ускорения постоянны в любой момент времени, поэтому
\[
a_1(t_0)=8\;\text{м/с}^2,\quad a_2(t_0)=1\;\text{м/с}^2.
\]