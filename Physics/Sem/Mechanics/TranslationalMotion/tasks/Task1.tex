\subsection{1}

Автомобиль половину пути движется со скоростью $v_1=72\;\text{км/ч}$, а вторую половину пути --- со скоростью $v_2=40\;\text{км/ч}$. Найдём среднюю скорость автомобиля $v_\text{ср}$.

Очевидно, что
\[
s_1=s_2=\frac s 2,
\]
где $s$ --- полный путь, который проехал автомобиль, $s_1$ и $s_2$ --- первая и вторая половины пути соответственно. Тогда время прохождения первой и второй половины пути соответсвенно равны
\[
t_1=\frac{s_1}{v_1}=\frac{s}{2v_1}
\]
и
\[
t_2=\frac{s_2}{v_2}=\frac{s}{2v_2}.
\]
Средняя скорость вычисляется по формуле
\[
v_\text{ср}=\frac{s}{t},
\]
где $t=t_1+t_2$ --- полное время в пути. Преобразуем последнюю формулу так, чтобы средняя скорость была выражена через известные нам величины:
\[
v_\text{ср}=\frac{s}{t_1+t_2}=\cfrac{s}{\cfrac{s}{2v_1}+\cfrac{s}{2v_2}}=\frac{2v_1v_2}{v_1+v_2}.
\]
Подставляя значения скоростей, получим
\[
v_\text{ср}=51\frac{3}{7}\approx51{,}4\;\text{км/ч}.
\]