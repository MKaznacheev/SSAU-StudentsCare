\subsection{18}

Кинематические уравнения движения двух материальных точек имеют вид $x_1=-2t^2$ и $x_2=2t+t^3$. Определим момент времени, для которого ускорения этих точек будут равны.

Поскольку точки движутся по прямой, их радиус-векторы представляются толь одной координатой $x_i$, $i\in{\,1,2\,}$. Тогда выражения проекций скоростей на направление движения имеют вид
\[
v_1=\frac{dx_1}{dt}=-4t,\quad v_2=\frac{dx_2}{dt}=2+3t^2,
\]
а проекций ускорений ---
\[
a_1=\frac{dv_1}{dt}=-4,\quad a_2=\frac{dv_2}{dt}=6t.
\]
Проекции ускорений двух точек в любой момент времени направлены в противоположные стороны, а потому имеет смысл говорить лишь о равенстве ускорений по модулю:
\[
|a_1|=|a_2|.
\]
Так, найдём соответствующий этому момент времени
\[
|-4|=6t\LR t=0{,}(6)\approx0{,}67\;\text{с}.
\]