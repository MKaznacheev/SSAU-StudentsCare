\subsection{18}

Две материальные точки движутся в соответсвии с уравнениями
\[
x_1=-2t^2,\quad x_2=2t+t^3.
\]
Поскольку $x_0=0\;\text{м}$ и функция $x_i$ монотонно возрастает при $t\Geq0$ в обоих случаях, то пути находятся по формулам
\[
s_1=|x_1-x_0|=|x_1|=2t^2
\]
и
\[
s_2=|x_2-x_0|=|x_2|=2t+t^3
\]
Тогда выражения скоростей имеют вид
\[
v_1=\frac{ds_1}{dt}=4t,\quad v_2=\frac{ds_2}{dt}=2+3t^2,
\]
а ускорений ---
\[
a_1=\frac{dv_1}{dt}=4,\quad a_2=\frac{dv_2}{dt}=6t.
\]
Найдём момент времени, для которого ускорения точек совпадают:
\[
a_1=a_2\LR4=6t\LR t=\frac{2}{3}\approx0{,}67\;\text{с}
\]