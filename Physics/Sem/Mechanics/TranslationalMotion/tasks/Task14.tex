\subsection{14}

Движение точки по прямой задано уравнением $x=2t-0{,}5t^2$, $[x]=\text{м}$. Определим среднюю скорость перемещения точки в интервале времени от $t_1=1\;\text{с}$ до $t_2=3\;\text{с}$.

При $t_1=1\;\text{с}$, как и при $t_2=3\;\text{с}$, координата точки равна $x_1=x_2=1{,}5\;\text{м}$. Ввиду прямолинейности движения, проекция перемещения $\Delta \vec r=\vec r(t_2)-\vec r(t_1)$ на ось, вдоль которой происходит движение, равна
\[
\Delta r=x_2-x_1=0\;\text{м}.
\]
Тогда $\Delta\vec r=\vec 0$ и средняя скорость равна
\[
v_\text{ср}=\frac{\Delta\vec r}{\Delta t}=\vec0.
\]