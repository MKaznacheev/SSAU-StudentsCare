\subsection{22}

Зависимость пройденного телом пути от времени дается уравнением
\[
s=2t+8t^2+16t^3,\quad[s]=\text{м}.
\]
Найдём расстояние, пройденное телом, его скорость и ускорение через $\tau=2\;\text{с}$ после начала движения.

Скорость и ускорение тела находятся по формулам
\[
v=\frac{ds}{dt}=2+16t+48t^2
\]
и
\[
a=\frac{dv}{dt}=16+96t.
\]
Спустя время $\tau$ с момента начала движения, указанные выше величины будут равны следующим значениям:
\begin{gather*}
s(\tau)=2\tau+8\tau^2+16\tau^3=164\;\text{м}, \\
v(\tau)=2+16\tau+48\tau^2=226\;\text{м/с}, \\
a(\tau)=16+96\tau=208\;\text{м/с}^2.
\end{gather*}