\subsection{4}

Кинематическое уравнение движения материальной точки по прямой имеет вид $x(t)=5+4t-t^2$, $[x]=\text{м}$. Найдём максимальное значение координаты; время, когда точка возвращается в то же место, где она была в начальный момент времени; путь, пройденный точкой с момента $t_1=1\;\text{с}$ до $t_2=6\;\text{с}$.

Максимальное значение координаты $x_m=x(t_m)$ (абсолютный максимум функции координаты от времени) найдём, исходя из следующих соображений. График функции $x(t)$ представляет из себя параболу, ветви которой направлены вниз. Значит, изначально значение координаты возрастает (предполагается, что точка максимума лежит правее нуля), а затем, после момента $t_m$, убывает. Знак скорости в таком случае меняется на противоположный, а её значение обязательно проходит через $0$. Очевидно, что последнее происходит в момент времени $t_m$:
\[
\dot x(t_m)=0.
\]
Решим последнее уравнение:
\[
4-2t_m=0\Leftrightarrow t_m=2\;\text{с}.
\]
В таком случае, $x_m=9\;\text{м}$.

В начальный момент времени точка находилась в положении, которое было задано координатой $x_0=x(0)=5\;\text{с}$. Время $\tau$, когда точка вернётся в то же положение, найдём, решая уравнение
\[
x(\tau)=x_0.
\]
Так,
\[
5+4\tau-\tau^2=5\Leftrightarrow \tau=4\;\text{с}.
\]

Как мы уже поняли, в интервале $(t_1,t_m)$ тело двигалось в одну сторону, а в интервале $(t_m, t_2)$ --- в другую. При этом путь, пройденный телом за каждый из этих промежутков времени, совпадает с модулем соответствующего перемещения. Значит, путь, пройденный телом, в интервале времени $(t_1, t_2)$ можно найти, как
\[
s=|x(t_m)-x(t_1)|+|x(t_2)-x(t_m)|.
\]
С учётом данных, полученных нами в ходе анализа функции координаты от времени, раскроем модули:
\[
s=2x(t_m)-x(t_1)-x(t_2)=4(2t_m-t_1-t_2)+2t_m^2+t_1^2+t_2^2.
\]
Подставив соответствующие значения, придём к значению $s=33\;\text{м}$.