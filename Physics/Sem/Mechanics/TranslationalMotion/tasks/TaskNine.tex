\subsection{17}

Путь, пройденный телом за время $t$, вычисляется по формуле
\[
S=2t-3t^2+4t^3,\quad[S]=\text{м}.
\]
Тогда скорость и ускорение задаются формулами
\[
v=\frac{dS}{dt}=2-6t+12t^2
\]
и
\[
a=\frac{dv}{dt}=-6+24t
\]
соответсвенно. Пусть от начала движения прошло время $t=2\;\text{с}$. Значит,
\begin{gather*}
S(2)=2\cdot2-3\cdot2^2+4\cdot2^3=24\;\text{м}, \\
v(2)=2-6\cdot2+12\cdot2^2=38\;\text{м/с}, \\
a(2)=-6+24\cdot2=42\;\text{м/с}^2.
\end{gather*}