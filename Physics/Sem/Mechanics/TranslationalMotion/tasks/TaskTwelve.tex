\subsection{20}

Положение точки полностью задаётся уравнением
\[
x=-8t+6t^2-t=-9t+6t^2,\quad[x]=\text{м}.
\]
Тогда проекция её перемещения будет равна
\[
r_x=x.
\]
Рассмотрим интервал $t_1\div t_2$, где $t_1=0\;\text{с}$, $t_2=1\;\text{с}$:
\[
r_{12}=|r_x(t_2)-r_x(t_1)|=|r_x(t_2)|=\left|-9\cdot1+6\cdot1^2\right|=3\;\text{м},
\]
поскольку $r_x(t_1)=0$. Средняя скорость на этом интервале равна
\[
\langle v_1\rangle=\frac{r_{12}}{t_2-t_1}=\frac{3}{1-0}=3\;\text{м/с}.
\]

Рассмотрим интервал $t_2\div t_3$, где $t_3=2\;\text{с}$:
\[
r_{23}=|r_x(t_3)-r_x(t_2)|=\left|-9\cdot2+6\cdot2^2+3\right|=9\;\text{м}.
\]
Средняя скорость на этом интервале равна
\[
\langle v_2\rangle=\frac{r_{23}}{t_3-t_2}=\frac{9}{2-1}=9\;\text{м/с}.
\]