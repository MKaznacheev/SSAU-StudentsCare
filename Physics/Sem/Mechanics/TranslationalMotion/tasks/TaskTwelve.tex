\subsection{20}

Положение точки полностью задаётся уравнением
\[
x=-8t+6t^2-t=-9t+6t^2,\quad[x]=\text{м}.
\]
Очевидно, $x_0=0\;\text{м}$, а значит
\[
s=|x-x_0|=|x|=\left|6t^2-9t\right|.
\]
Рассмотрим интервал $t_1\div t_2$, где $t_1=0\;\text{с}$, $t_2=1\;\text{с}$:
\[
s_{12}=s(t_2)=\left|6\cdot1^2-9\cdot1\right|=3\;\text{м}.
\]
Средняя скорость на этом интервале равна
\[
\langle v_1\rangle=\frac{s_{12}}{t_2-t_1}=\frac{3}{1-0}=3\;\text{м/с}.
\]

Рассмотрим интервал $t_2\div t_3$, где $t_3=2\;\text{с}$:
\[
s_{23}=s(t_3)-s(t_2)=\left|6\cdot2^2-9\cdot2\right|-3=3\;\text{м}.
\]
Средняя скорость на этом интервале равна
\[
\langle v_2\rangle=\frac{s_{23}}{t_3-t_2}=\frac{3}{2-1}=3\;\text{м/с}.
\]