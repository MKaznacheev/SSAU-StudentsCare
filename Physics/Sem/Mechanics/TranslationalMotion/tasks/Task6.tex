\subsection{6}

Точка движется вдоль оси $Ox$ согласно графику, изображенному на рисунке~\ref{pic:6.1}. Построим графики изменения ускорения и скорости движения. Определим начальную и средние (в интервале от $t_0=0\;\text{с}$ до $t_1=8\;\text{с}$) скорости движения.

\begin{figure}[h!]
	\begin{center}
		\begin{tikzpicture}[yscale=0.2, xscale=0.6, domain=0:8]
			\draw[very thin, color=gray] (0,0) grid (12,20);
			\draw [arrows = {-Straight Barb[scale=1]}] (0,0) -- (12,0) node [sloped, below] {$t$, с};
			\draw [arrows = {-Straight Barb[scale=1]}] (0,0) -- (0,20) node [sloped, left] {$x$, м};
			\draw node [below left] at (0,0) {$0$};
			\draw node [below] at (4,0) {$4$};
			\draw node [below] at (8,0) {$8$};
			\draw node [left] at (0,16) {$16$};
			\draw[color=red, smooth, thick] plot (\x,{8*\x-\x*\x});
		\end{tikzpicture}
		\caption{. Функция $x(t)$}\label{pic:6.1}
	\end{center}
\end{figure}

Обратим внимание на рисунок~\ref{pic:6.1}. На нём изображена парабола
\[
x(t)=a+bt+ct^2,
\]
где $a$, $b$, $c$ --- некоторые постоянные. Она пересекает ось абсцисс в точках $(0,0)$ и $(8,0)$, а её вершина лежит в точке $(4,16)$, значит
\[
\begin{cases}
x(0)=0 \\
x(8)=0 \\
x(4)=16
\end{cases}
\Leftrightarrow\quad
\begin{cases}
a=0 \\
a+8b+8^2c=0 \\
a+4b+4^2c=16.
\end{cases}
\]
Решая систему, найдём, что $a=0$, $b=8$ и $c=-1$, то есть $x(t)=8t-t^2$. Тогда для скорости имеем
\[
v(t)=\frac{dx}{dt}=8-2t,
\]
а для ускорения ---
\[
a(t)=\frac{dv}{dt}=-2\;\text{м/с}^2.
\]
При этом, конечно, имеются ввиду проекции величин на ось, вдоль которой происходит движение. Построим соответствующие графики (рисунок~\ref{pic:6.2}).

\begin{figure}[h!]
	\begin{center}
		\subfloat{
		\begin{tikzpicture}[scale=0.4, domain=0:8]
			\draw[very thin, color=gray] (0,-10) grid (8,10);
			\draw [arrows = {-Straight Barb[scale=1]}] (0,0) -- (8,0) node [sloped, below] {$t$, с};
			\draw [arrows = {-Straight Barb[scale=1]}] (0,-10) -- (0,10) node [sloped, left] {$v$, м/с};
			\draw node [left] at (0,0) {$0$};
			\draw node [below] at (4,0) {$4$};
			\draw node [left] at (0,8) {$8$};
			\draw[color=red, smooth, thick] plot (\x,{8-2*\x});
		\end{tikzpicture}
		}
		%\hfill
		\subfloat{
		\begin{tikzpicture}[scale=0.6, domain=0:8]
			\draw[very thin, color=gray] (0,-4) grid (8,4);
			\draw [arrows = {-Straight Barb[scale=1]}] (0,0) -- (8,0) node [sloped, below] {$t$, с};
			\draw [arrows = {-Straight Barb[scale=1]}] (0,-4) -- (0,4) node [sloped, left] {$a$, м/с$^2$};
			\draw node [left] at (0,0) {$0$};
			\draw node [left] at (0,-2) {$-2$};
			\draw[color=red, smooth, thick] plot (\x,-2);
		\end{tikzpicture}
		}
		\caption{. Функции $v(t)$ (слева) и $a(t)$ (справа)}\label{pic:6.2}
	\end{center}
\end{figure}

Как видно из графика функции $v(t)$, начальная скорость (скорость в момент времени $t_0=0\;\text{с}$) равняется $v_0=8\;\text{м/с}$. Определить среднюю скорость перемещения точки (точнее, её проекцию) можно по формуле
\[
\vec v_\text{ср}=\frac{\Delta\vec r}{\Delta t}.
\]
Однако, в интересующем нас промежутке времени, $\Delta r=\vec r(t_1)-\vec r(t_0)=\vec 0$, поскольку $x(t_0)=x(t_1)=0\;\text{м}$, а значит $\vec v_\text{ср}=\vec 0$.

Глядя на график функции $x(t)$, нетрудно догадаться, что путь, пройденный точкой в интервале времени $(0,8)$, можно вычислить, как
\[
s=2x(4)=32\;\text{м},
\]
поскольку точка проходит одинаковое расстояние, за промежутки $(0,4)$ и $(4,8)$. Тогда средняя путевая скорость вычисляется по формуле
\[
v_\text{ср}=\frac{s}{t},
\]
где $t=t_1-t_0=t_1$:
\[
v_\text{ср}=\frac{s}{t_1}=4\;\text{м/с}.
\]