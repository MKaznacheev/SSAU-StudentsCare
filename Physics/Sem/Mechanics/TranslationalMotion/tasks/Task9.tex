\subsection{9}

Скорость реки составляет $V=3\;\text{км/ч}$, а скорость движения лодки относительно воды --- $u=6\;\text{км/ч}$. Определим, под каким углом относительно берега должна двигаться лодка, чтобы проплыть поперёк реки.

Чтобы лодка плыла поперёк реки, её скорость~$\vec U$ относительно земли должна быть направлена перпендикулярно течению. Тогда, для компенсанции влияния последнего, скорость лодки $\vec u$ относительно воды должна быть направлена так, как показано на рисунке~\ref{pic:9.1}. То есть она складывается из векторов $\vec U$ и $-\vec V$.
\begin{figure}[h!]
	\begin{center}
		\begin{tikzpicture}
			\draw [rounded corners=6] (0,0.4) -- (0.4,0) foreach \x in {0.8,1.6,...,8} {-- (\x,0.4) -- (\x+0.4,0)};
			\draw [rounded corners=6] (0,3.4) -- (0.4,3) foreach \x in {0.8,1.6,...,8} {-- (\x,3.4) -- (\x+0.4,3)};
			\draw [arrows = {-Stealth[scale=1]}, thick] (4,2.5) to node [sloped, above right] {$\vec V$} (6,2.5);
			\draw [arrows = {-Stealth[scale=1]}, thick] (4,2.5) to node [right] {$\vec U$}  (4,1);
			\draw [arrows = {-Stealth[scale=1]}, thick] (4,2.5) to node [above left] {$\vec u$}  (2,1);
			\draw [densely dashed] (2,1) -- (4,1) -- (6,2.5);
			\draw [densely dashed] (4,2.5) -- (2,2.5) -- (2,1);
			\draw (2,2.2) -- (2.3,2.2) -- (2.3,2.5);
			\draw (3.3,2.5) arc [start angle=180, end angle = 218, radius=0.7];
			\draw node at (3.5,2.35) {$\alpha$};
		\end{tikzpicture}
		\caption{}\label{pic:9.1}
	\end{center}
\end{figure}

Через $\alpha$ обозначим угол между вектором $\vec u$ и линией берега. Из рисунка видно, что
\[
\cos\alpha=\frac{V}{u},
\]
а значит,
\[
\alpha=\arccos\frac{V}{u}=60^\circ.
\]