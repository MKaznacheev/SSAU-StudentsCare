\subsection{19}

Уравнение
\[
s=0{,}14t^2+0{,}01t^3,\quad[s]=\text{м}
\]
задаёт зависимость пройденного точкой пути от времени. Значит, скорость и ускорение точки задаются уравнениями
\[
v=\frac{ds}{dt}=0{,}28t+0{,}03t^2
\]
и
\[
a=\frac{dv}{dt}=0{,}28+0{,}06t
\]
соответсвенно. Найдём время, через которое ускорение точки составит $a_1=1\;\text{м/с}^2$:
\[
a_1=0{,}28+0{,}06t_1\LR t_1=\frac{a_1-0{,}28}{0{,}06}=\frac{1-0{,}28}{0{,}06}=12\;\text{с}.
\]
В этот момент времени скорость будет равна
\[
v_1=0{,}28\cdot12+0{,}03\cdot12^2=7{,}68\;\text{м/с}.
\]

Пусть $t_a=0\;\text{с}$ и $t_b=2\;\text{с}$. Тогда путь, пройденный телом во временном промежутке $t_a\div t_b$ равен
\[
s_{ab}=s(t_b)=0{,}14\cdot2^2+0{,}01\cdot2^3=0{,}64\;\text{м}.
\]
Средняя скорость за промежуток составит
\[
v_\text{ср}=\frac{s_{ab}}{t_b-t_a}=\frac{0{,}64}{2-0}=0{,}32\;\text{м/с}.
\]