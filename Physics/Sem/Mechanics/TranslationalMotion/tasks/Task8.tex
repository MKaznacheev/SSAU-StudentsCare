\subsection{8}

Автомобиль едет по прямой из пункта $A$ в пункт $B$, преодолевая это расстояние за $T=1\;\text{ч}$. Известно, что скорость автомобиля меняется по закону
\[
v(t)=v_0\sin\left(\frac{\pi}{T}t\right),
\]
где время $t$ отсчитывается с момента выезда из пункта $A$, а максимальная скорость автомобиля $v_0=80\;\text{км/ч}$. Определим среднюю путевую скорость автомобиля.

Для начала найдём координату материальной точки в зависимости от времени:
\[
x(t)=\int v(t)\,dt=v_0\int\sin\left(\frac{\pi}{T}t\right)\,dt=-\frac{Tv_0}{\pi}\cos\left(\frac{\pi}{T}t\right).
\]
Отсюда найдём координаты точек $A$ и $B$:
\[
A=x(0)=-\frac{Tv_0}{\pi},\quad B=x(T)=\frac{Tv_0}{\pi}.
\]
Поскольку скорость не меняла направление, то путь равен
\[
s=AB=B-A=\frac{2Tv_0}{\pi}.
\]
Тогда средняя путевая скорость принимает значение
\[
v_\text{ср}=\frac{s}{T}=\frac{2v_0}{\pi}=\frac{160}{\pi}\approx50{,}9\;\text{км/ч}.
\]