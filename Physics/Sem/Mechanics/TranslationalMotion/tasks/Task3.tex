\subsection{3}

Положение материальной точки на оси $Ox$ в зависимости от времени задано уравнением $x=x(t)=4t+8t^2-2t^3$, $[x]=\text{м}$. Найдём среднюю скорость перемещения точки на временном интервале от $t_1=2\;\text{с}$ до $t_2=4\;\text{с}$. Сравним полученное значение с мгновенными скоростями $v_1$ и $v_2$ в моменты времени $t_1$ и $t_2$ соответственно.

Пусть $\vec r=\vec r(t)$ --- радиус вектор материальной точки. Поскольку точка движется по прямой вдоль оси $Ox$, то её положение, а точнее её радиус вектор, полностью определятся координатой $x$: $\vec r=x\vec i$, где $i$ --- единичный базисный вектор на оси абсцисс. Чтобы найти среднюю скорость перемещения точки, воспользуемся формулой
\[
\vec v_\text{ср}=\frac{\Delta\vec r}{\Delta t},
\]
где
\[
\Delta\vec r=\vec r(t_2)-\vec r(t_1)=\bigl(x(t_2)-x(t_1)\bigr)\vec i=\bigl(4(t_2-t_1)+8(t_2^2-t_1^2)-2(t_2^3-t_1^3)\bigr)\vec i
\]
и $\Delta t=t_2-t_1$. То есть 
\begin{multline*}
\vec v_\text{ср}=\frac{4(t_2-t_1)+8(t_2^2-t_1^2)-2(t_2^3-t_1^3)}{t_2-t_1}\vec i= \\
=\bigl(4+8(t_1+t_2)-2(t_1^2+t_1t_2+t_2^2)\bigr)\vec i=-4\vec i.
\end{multline*}
Мгновенная же скорость вычисляется по формуле
\[
\vec v(t)=\frac{d\vec r}{dt}.
\]
С учётом условия задачи,
\[
\vec v(t)=\dot x\vec i=(4+16t-6t^2)\vec i.
\]
В точках $t_1$ и $t_2$ она принимает значения $\vec v_1=12\vec i$ и $\vec v_2=-28\vec i$ соответственно.

Сравнивая полученные результаты, заметим, что скорость точки в разные моменты времени направлена в разные стороны. При этом на исследуемом временном интервале скорость меняет своё направления с положительного на отрицательное.