\subsection{17}

Зависимость пройденного телом пути от времени $t$ дается уравнением $s=2t-3t^2+4t^3$, $[s]=\text{м}$. Найдём зависимость скорости и ускорения от времени, а так же расстояние, пройденное телом, скорость и ускорение через $\tau=2\;\text{с}$ после начала
движения

Скорость и ускорение тела задаются формулами
\[
v=\frac{ds}{dt}=2-6t+12t^2,\quad a=\frac{dv}{dt}=-6+24t
\]
соответсвенно. Значит,
\begin{gather*}
s(\tau)=2\tau-3\tau^2+4\tau^3=24\;\text{м}, \\
v(\tau)=2-6\tau+12\tau^2=38\;\text{м/с}, \\
a(\tau)=-6+24\tau=42\;\text{м/с}^2.
\end{gather*}