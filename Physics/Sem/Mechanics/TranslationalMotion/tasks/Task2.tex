\subsection{2}

Автомобиль половину времени движется со скоростью $v_1=72\;\text{км/ч}$, а вторую половину времени --- со скоростью $v_2=40\;\text{км/ч}$. Найдём среднюю скорость автомобиля.

По условию,
\[
t_1=t_2=\frac{t}{2},
\]
где $t_1$ и $t_2$ --- первая и вторая половина времени соответсвенно, а $t$ --- полное время. Путь, пройденный за время $t_1$, найдём, как
\[
s_1=v_1t_1=\frac{v_1t}{2},
\]
а за время $t_2$, как
\[
s_2=v_2t_2=\frac{v_2t}{2}.
\]
Подставим эти выражения в формулу
\[
v_\text{ср}=\frac{s}{t},
\]
где $s=s_1+s_2$ --- полный путь, пройденный автомобилем:
\[
v_\text{ср}=\frac{s_1+s_2}{t}=\cfrac{\cfrac{v_1t}{2}+\cfrac{v_2t}{2}}{t}=\frac{v_1+v_2}{2}.
\]
В силу известных нам значений, $v_\text{ср}=56\;\text{км/ч}$.