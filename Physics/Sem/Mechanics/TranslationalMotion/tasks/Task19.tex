\subsection{19}

Зависимость пройденного точкой пути от времени задана уравнением
$s=0{,}14t^2+0{,}01t^3$, $[s]=\text{м}$. Определим, через какое время ускорение точки будет равно $a_1=1\;\text{м/с}^2$, мгновенную скорость в этот момент времени и среднюю скорость перемещения за промежуток времени от $t_a=0\;\text{с}$ до $t_b=2\;\text{с}$.

Скорость и ускорение точки вычисляются по формулам
\[
v=\frac{ds}{dt}=0{,}28t+0{,}03t^2
\]
и
\[
a=\frac{dv}{dt}=0{,}28+0{,}06t
\]
соответсвенно. Решим уравнение $a_1=a(t_1)$, для нахождения момента времени $t_1$, которому соответствует ускорение $a_1$:
\[
a_1=0{,}28+0{,}06t_1\LR t_1=\frac{a_1-0{,}28}{0{,}06}=12\;\text{с}.
\]
В то же время скорость будет равна
\[
v_1=v(t_1)=0{,}28t_1+0{,}03t_1^2=7{,}68\;\text{м/с}.
\]

Путь, пройденный телом во временном промежутке $t_a\div t_b$, равен
\[
s_{ab}=s(t_b)=0{,}14t_b^2+0{,}01t_b^3=0{,}64\;\text{м}.
\]
Полное время движения равно $t_b$, а значит, средняя скорость за этот промежуток составит
\[
v_\text{ср}=\frac{s_{ab}}{t_b}=0{,}32\;\text{м/с}.
\]