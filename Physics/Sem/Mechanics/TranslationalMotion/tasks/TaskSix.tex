\subsection{14}

Точка движется по прямой в соответствии с уравнением $x=2t-0{,}5t^2$, $[x]=\text{м}$. Тогда при $t_1=1\;\text{с}$ координата точки равна
\[
x_1=2\cdot1-0{,}5\cdot1^2=1{,}5\;\text{м},
\]
как и при $t_2=3\;\text{с}$:
\[
x_2=2\cdot3-0{,}5\cdot3^2=1{,}5\;\text{м}.
\]
Значит, в промежутке времени $t_1\div t_2$ тело дошло до точки $O$, в которой его скорость сменила направление на противоположное. Найдём время достижения точки $O$, как точку экстремума:
\[
\frac{dx}{dt}=2-t\Rightarrow 2-t_O=0\LR t_O=2\;\text{с}.
\]
Координата точки в это время равнялась
\[
x_O=2\cdot2-0{,}5\cdot2^2=2\;\text{м}.
\]
За промежуток времени $t_1\div t_O$ точка прошла путь
\[
s_1=|x_O-x_1|=|2-1{,}5|=0{,}5\;\text{м},
\]
так же, как и за промежуток$t_O\div t_2$:
\[
s_1=|x_2-x_O|=|1{,}5-2|=0{,}5\;\text{м}.
\]
То есть полный путь составил
\[
s=s_1+s_2=0{,}5+0{,}5=1\;\text{м},
\]
а время в пути ---
\[
t=t_2-t_1=3-1=2\;\text{с}.
\]
Тогда средняя скорость равна
\[
v_\text{ср}=\frac{s}{t}=\frac{1}{2}=0{,}5\;\text{м/с}.
\]