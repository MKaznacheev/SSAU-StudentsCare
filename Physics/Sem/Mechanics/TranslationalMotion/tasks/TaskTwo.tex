\subsection{10}

Если скорость ветра составляет $v=11\;\text{м/с}$, то капля дождя падает под углом $\alpha=30^\circ$ к вертикали. Обозначим её скорость через $\vec u$ (рисунок~\ref{PicTwo}), а вертикальную составляющую скорости --- через $\vec \nu$.

\begin{figure}[h!]
	\begin{center}
		\begin{tikzpicture}
			\draw (0,0) -- (8,0);
			\foreach \x in {1,...,40} \draw (\x/5,0) -- (\x/5-0.2,-0.2);
			\draw [arrows = {-Stealth[scale=1]}, thick] (4,2.5) to node [sloped, above right] {$\vec v$} (5,2.5);
			\draw [arrows = {-Stealth[scale=1]}, thick] (4,2.5) to node [sloped, above] {$\vec u$}  (5,1);
			\draw [arrows = {-Stealth[scale=1]}, thick] (4,2.5) to node [below left] {$\vec \nu$}  (4,1);
			\draw [densely dashed] (4,3) -- (4,0.3);
			\draw [densely dashed] (4,1) -- (5,1) -- (5,2.5);
			\draw (4,1.8) arc [start angle=270, end angle = 303, radius=0.7];
			\draw node at (4.135,2) {$\alpha$};
			\draw (4,1.3) -- (4.3,1.3) -- (4.3,1);
		\end{tikzpicture}
		\caption{}\label{PicTwo}
	\end{center}
\end{figure}

Из рисунка видно, что
\[
\tg\alpha=\frac{v}{\nu}\Rightarrow\nu=\frac{v}{\tg\alpha}.
\]

В том случае, если скорость капли составляет с вертикалью угол $\beta=45^\circ$, скорость ветра должна равняться
\[
v_1=\nu\tg\beta=v\cdot\frac{\tg\beta}{\tg\alpha}=11\cdot\frac{\tg45^\circ}{\tg30^\circ}=11\sqrt3\approx19,1\;\text{м/с}.
\]