\subsection{21}

Движение двух материальных точек заданы уравнениями $x_1=20+4t-4t^2$ и $x_2=2+t+0{,}5t^2$. Определим, в какой момент времени скорости этих точек будут одинаковы, а так же скорости и ускорения точек в этот момент.

Радиус-векторы точек задаются уравнениями $\vec r_1=x_1\vec i$ и $\vec r_2=x_2\vec i$, где $\vec i$ --- единичный базисный вектор прямой, вдоль которой происходит движение. Тогда скорость и ускорение первой точки равны
\[
\vec v_1=\frac{d\vec r_1}{dt}=(4-8t)\vec i,\quad\vec a_1=\frac{d\vec v_1}{dt}=-8\vec i,
\]
а второй ---
\[
\vec v_2=\frac{d\vec r_2}{dt}=(1+t)\vec i,\quad \vec a_2=\frac{d\vec v_2}{dt}=\vec i.
\]
Вычислим момент времени, когда скорости точек совпадут, из уравнения $\vec v_1(t_0)=\vec v_2(t_0)$:
\[
(4-8t_0)\vec i=(1+t_0)\vec i\LR4-8t_0=1+t_0\LR t_0=0{,}(3)\approx0{,}3\;\text{с}.
\]
При этом
\[
\vec v_1(t_0)=(4-8t_0)\vec i=\frac{4}{3}\vec i\approx1{,}3\vec i,\quad \vec v_2(t_0)=(1+t_0)\vec i=\frac{4}{3}\vec i\approx1{,}33\vec i.
\]
Ускорения точек постоянны в любой момент времени.