\subsection{15}

Тело начинает движение из точки с координатами
\[
(x_0,y_0)=(0,0)
\]
со скоростью
\[
\vec V=a\vec i+b\vec jt,
\]
где $a=\text{const}$ и $b=\text{const}$. Получим зависимость радиус-вектора от времени:
\[
\vec r=\int\vec V\,dt=a\vec i\int dt+b\vec j\int t\,dt=a\vec it+\frac{b\vec jt^2}{2}+\vec r_0.
\]
Конечно,
\[
r_0=\sqrt{x_0^2+y_0^2}=\sqrt{0^2+0^2}=0\Rightarrow\vec r_0=\vec0,
\]
а значит,
\[
\vec r=a\vec it+\frac{b\vec jt^2}{2}.
\]
Эта запись означает следующее:
\[
\vec r=\left(at,\frac{bt^2}{2}\right).
\]
Иначе говоря,
\[
x(t)=at,\quad y(t)=\frac{bt^2}{2},
\]
откуда
\[
t=\frac{x}{a}
\]
и
\[
y(x)=\cfrac{b\left(\cfrac{x}{a}\right)^2}{2}=\frac{bx^2}{2a^2}.
\]