\subsection{7}

Материальная точка движется прямолинейно с начальной скоростью $v_0=10\;\text{м/c}$ и с постоянным ускорением $a=-5\;\text{м/с}^2$. Определим, чему равен путь, пройденный точкой, и модуль ее перемещения спустя $t_1=4\;\text{с}$ после начала движения.

В терминах проекций на ось, вдоль которой осуществляется движение, скорость определяется выражением
\[
v(t)=\int a\,dt=at+v_0
\]
(убедится в том, что константой интегрирования является начальная скорость, можно, подставив в уравнение $v=at+\text{const}$ значение $t=0\;\text{с}$). Единственная координата радиус-вектора выражается аналогично:
\[
x(t)=\int v\,dt=\int(at+v_0)\,dt=\frac{at^2}{2}+v_0t+x_0.
\]

Спустя время $t_1$ после начала движения, точка окажется в положении, задаваемом координатой $x(t_1)$. При этом её перемещение окажется равным
\[
\left|\Delta\vec r\,\right|=\left|\vec r(t_1)-\vec r_0\right|=|x(t_1)-x_0|=\left|\frac{at_1^2}{2}+v_0t_1\right|=0\;\text{с}.
\]

Найдём путь, пройденный за указанное время, чисто математически. Функция $x(t)$ немонотонна, а значит может иметь локальные экстремумы в точках интервала $(0, t_1)$.

Пусть $\tau$ --- точка экстремума (локального или абсолютного) функции $x(t)$, если таковая существует. С учётом дифференцируемости функции на всей области определения $\mathbb R$, воспользуемся леммой Ферма: $\dot x(\tau)=0$. Однако $\dot x(\tau)$ есть ни что иное, как $v(\tau)$, а значит $\tau$ найдём из уравнения $v(\tau)=0$. Так,
\[
\tau=-\frac{v_0}{a}=2\;\text{с}.
\]
Чтобы проверить, точно ли точка $\tau$ является точкой экстремума, следует найти вторую производную функции $x(t)$ в этой точке. Как нам известно, $\ddot x(\tau)=a<0$. Последнее говорит о том, что, точка $\tau$ является точкой локального максимума.

Исходя из написанного выше, можно сделать вывод, что в интервале $(0,\tau)$ тело двигалось в одну сторону, а в интервале $(\tau, t_1)$ --- в другую. При этом путь, пройденный телом за каждый из этих промежутков времени, совпадает с модулем соответствующего перемещения. Значит, путь, пройденный телом, в интервале времени $(0, t_1)$ можно найти, как
\[
s=|x(\tau)-x_0|+|x(t_2)-x(\tau)|.
\]
С учётом данных, полученных нами в ходе анализа функции координаты от времени, раскроем модули:
\[
s=2x(\tau)-x(t_1)-x_0=\frac{a(2\tau^2-t_1^2)}{2}+v_0(2\tau-t_1)=20\;\text{м}.
\]