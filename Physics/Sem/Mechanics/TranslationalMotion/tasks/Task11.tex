\subsection{11}

Два автомобиля, выехав одновременно из одного пункта, движутся прямолинейно в одном направлении. Зависимость пройденного пути дается уравнениями $s_1=Ft+Bt^2$ и $ s_2=Ct+Dt^2+At^3$ . Определим скорость второго автомобиля в системе координат, связанной с первым.

Скорости автомобилей вычисляются, как
\[
v_1=\frac{ds_1}{dt}=F+2Bt,\quad v_2=\frac{ds_2}{dt}=C+2Dt+3At^2.
\]
Обозначим скорость второго автомобиля относительного первого через $\vec u=\vec v_2-\vec v_1$. Поскольку автомобили движутся вдоль одной прямой, то векторы $\vec v_1$ и $\vec v_2$ коллинеарны, а значит,
\[
u=v_2-v_1=C-F+2(D-B)t+3At^2.
\]
(в проекции на направление движения).