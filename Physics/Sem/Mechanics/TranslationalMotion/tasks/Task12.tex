\subsection{12}

Автомобиль проехал первую половину времени своего движения со скоростью $v_1=40\;\text{км/ч}$, а вторую половину времени своего движения --- со скоростью $v_2=60\;\text{км/ч}$. Определим среднюю скорость движения автомобиля.

Полное время движения обозначим через $t$. Тогда
\[
t_1=t_2=\frac{t}{2},
\]
где $t_1$ и $t_2$ --- первая и вторая половины времени движения. Путь, пройденный за первую половину времени составляет
\[
s_1=v_1t_1=\frac{v_1t}{2},
\]
а за вторую ---
\[
s_2=v_2t_2=\frac{v_2t}{2}.
\]
Средняя скорость тела вычисляется по формуле
\[
v_\text{ср}=\frac{s}{t},
\]
где $s=s_1+s_2$ --- полный путь, проделанный автомобилем. Так,
\[
v_\text{ср}=\frac{s_1+s_2}{t}=\cfrac{\cfrac{v_1t}{2}+\cfrac{v_2t}{2}}{t}=\frac{v_1+v_2}{2}=50\;\text{км/ч}.
\]