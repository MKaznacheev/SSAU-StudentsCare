\subsection{10}

Груз массой $m=20\;\text{кг}$ поднят на высоту $h=15\;\text{м}$. Условимся считать, что на него действуют лишь консервативные силы, и потенциальная энергия в нулевом положении равна нулю. Тогда
\[
dU=-\delta A=-m\vec g\cdot d\vec S=-mg\,dS,
\]
а значит,
\[
U=\int\limits^0_{H} -mg\,dS=\left.-mgS\right\vert^0_{H}=mgH,
\]
где $H$ --- некоторая высота. Так,
\[
U_1=mgh_1=2940\;\text{Дж}
\]

Вычислим работу, которую совершила вертикальная сила $F=400\;\text{Н}$ при подъёме груза:
\[
A_{01}=\int\limits_0^h\vec F\cdot d\vec S=\int\limits_0^h F\cdot dS=Fh-F\cdot0=Fh=6\;\text{кДж}.
\]