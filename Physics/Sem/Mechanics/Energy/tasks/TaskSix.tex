\subsection{13}

Тело массы $m$ движется по закону
\[
v=b\sqrt s,
\]
где $b=\text{const}$, $s$ --- путь. Известно, что
\[
A=K-K_0=\frac{mv^2}{2}=\frac{mb^2s}{2},
\]
поскольку движение происходит без начальной скорости. При этом
\[
ds=v\,dt=b\sqrt s\,dt\LR \frac{1}{2\sqrt s}\,ds=\frac{b}{2}\,dt.
\]
Проинтегрируем полученное:
\[
\int\limits_0^S\frac{1}{2\sqrt s}\,ds=\int\limits_0^\tau\frac{b}{2}\,dt\LR \left.\sqrt s\right\vert_0^S=\left.\frac{bt}{2}\right\vert_0^\tau\LR\sqrt S=\frac{b\tau}{2}\LR S=\frac{b^2\tau^2}{4}.
\]
Так, за время $\tau$ тело совершит работы
\[
A=\frac{mb^4\tau^2}{8}.
\]