\subsection{11}

Масса материальной точки составляет $m=2\;\text{кг}$. Она движется вдоль оси абсцисс по закону
\[
x=5-2t+t^2-0{,}2t^3.
\]
Тогда
\[
v=\dot x=-2+2t-0{,}6t^2,
\]
и
\[
a=\dot v=2-1{,}2t.
\]
Очевидно, что
\[
F=ma=2m-1{,}2mt=4-2{,}4t.
\]
Если работа равна
\[
\delta A=\vec F\cdot d\vec s,
\]
то мощность равна
\[
N=\frac{\delta A}{dt}=\vec F\cdot\vec v=Fv.
\]
В момент времени $\tau=2\;\text{с}$:
\[
\left\{
\begin{NiceArray}{l}
F(\tau)=-0{,}8\;\text{Н} \\
v(\tau)=-0{,}4\;\text{м/с}
\end{NiceArray}
\right.\;
\Rightarrow N(\tau)=0{,}32\;\text{Вт}.
\]