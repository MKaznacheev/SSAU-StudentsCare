\subsection{8}

Частица, движущаяся по окружности радиуса $R$ обладает кинетической энергией $K$. При этом
\[
K=\alpha S^2,
\]
где $\alpha=\text{const}$, $S$ --- путь, пройденный частицей. Заметим, что
\[
K=\frac{mv^2}{2}.
\]
Приравняем последние два уравнения:
\begin{equation}\label{EqOne}
\frac{mv^2}{2}=\alpha S^2\LR mv^2=2\alpha S^2.
\end{equation}
Продифференцируем полученное по времени:
\[
m\frac{d v^2}{dt}=2\alpha\frac{dS^2}{dt}\LR 2mv\frac{dv}{dt}=4\alpha S\frac{dS}{dt}\LR mva_\tau=2\alpha Sv\LR ma_\tau=2\alpha S.
\]
Учтём, что
\[
a_n=\frac{v^2}{R}=\frac{2\alpha S^2}{mR}\LR ma_n=\frac{2\alpha S^2}{R},
\]
ввиду уравнения~\eqref{EqOne}. Наконец,
\[
F=ma=\sqrt{(ma_\tau)^2+(ma_n)^2}=2\alpha S\sqrt{1+\frac{S^2}{R^2}}.
\]