\subsection{15}

Пусть $r$ --- длина радиус-вектора частицы, $a$ и $k$ --- некоторые постоянные. Частица перешла из точки $M_1(3;2;1)$ в точку $M_2(1;2;3)$. Если
\[
\vec r=x\vec i+y\vec j+z\vec k,
\]
то
\[
r=\sqrt{x^2+y^2+z^2}.
\]

Пусть потенциальная энергия частица выражается формулой
\[
U=\frac{a}{r}.
\]
Тогда
\[
F=-\frac{dU}{dr}=-\frac{d}{dr}\frac{a}{r}=\frac{a}{r^2}.
\]
Направление силы совпадает с направлением радиус вектора:
\[
\vec F=\frac{a}{r^2}\cdot\frac{\vec r}{r}.
\]
Работа силы при переходе из точки $M_1$ в $M_2$ вычисляется, как
\[
A=U_1-U_2=\frac{a}{r_1}-\frac{a}{r_2}=0\;\text{Дж},
\]
поскольку
\[
r_1=\sqrt{3^2+2^2+1^2}=\sqrt{1^2+2^2+3^2}=r_2.
\]

Пусть теперь
\[
U=\frac{kr^2}{2}.
\]
Тогда
\[
F=\frac{dU}{dr}=\frac{d}{dr}\frac{kr^2}{2}=kr;\quad \vec F=-k\vec r
\]
(векторы $\vec F$ и $d\vec r$ противонаправленные).
Работа силы при переходе из точки $M_1$ в $M_2$ вычисляется, как
\[
A=U_1-U_2=\frac{kr_1^2}{2}-\frac{kr_2^2}{2}=0\;\text{Дж},
\]
поскольку, аналогично,
\[
r_1=r_2.
\]