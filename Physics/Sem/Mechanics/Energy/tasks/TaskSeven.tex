\subsection{14}

Некоторая частица обладает потенциальной энергией
\[
U=x+2y^2+3z^3
\]
в зависимости от координат. Поскольку
\[
\vec F=-\nabla U=-\left(\frac{\partial U}{\partial x}\vec i+\frac{\partial U}{\partial x}\vec j+\frac{\partial y}{\partial z}\vec k\right)=-\vec i-4y\vec j-9z^2\vec k.
\]

Радиус-вектор частицы меняется следующим образом:
\[
\vec r_1=\vec i+\vec j+\vec k\longrightarrow \vec r_2=2\vec i+2\vec j+2\vec k.
\]
Учтём, что
\[
x_1=y_1=z_1=1,\quad x_2=y_2=z_2=2.
\]
Тогда, ввиду консервативности силы,
\[
A=U_1-U_2=x_1+2y_1^2+3z_1^3-x_2-2y_2^2-3z_2^3=-28\;\text{Дж}.
\]