\subsection{12}

Дорожка имеет форму окружности радиусом $R=4\;\text{м}$. По закону сохранения энергия,
\[
K_1+U_1=K_2+U_2,
\]
где $K_i$, $U_i$ --- кинетическая и потенциальная энергии в начале движения ($i=1$) и в его конце ($i=2$). При этом $K_1=0$, поскольку движение происходит без начальной скорости. Для остальных величин справедливы выражения
\[
U_1=mgh,\quad U_2=2mgR,\quad K_2=\frac{mv^2}{2},
\]
если нулевую точку принять нижнюю точку дорожки. В верхней точке траектории в проекции на направленную вертикально вверх ось уравнение движения принимает вид
\[
-mg=-ma_\text{ц}\LR a_\text{ц}=g.
\]
Тогда
\[
v^2=a_\text{ц}R=gR
\]
и
\[
K_2=\frac{mgR}{2}.
\]
Суммируя сказанное выше, получим,
\[
mgh=2mgR+\frac{mgR}{2}\LR h=\frac{5}{2}R=10\;\text{м}.
\]