\subsection{16}

Потенциальная энергия частицы выражается, как
\[
U=\alpha\cdot\left(\frac{x^2}{y}-\frac{y^2}{z}\right),
\]
где $\alpha=\text{const}$. Тогда
\[
\vec F=-\nabla U=-\left(\frac{\partial U}{\partial x}\vec i+\frac{\partial U}{\partial y}\vec j+\frac{\partial U}{\partial z}\vec k\right)=-\alpha\cdot\frac{2x}{y}\vec i+\alpha\cdot\frac{zx^2+2y^3}{zy^2}\vec j-\alpha\cdot\frac{y^2}{z^2}\vec k.
\]

Частица перешла из точки $M_1(3;2;1)$ в $M_2(1;2;3)$. Тогда
\[
U_1=\alpha\cdot\left(\frac{3^2}{2}-\frac{2^2}{1}\right)=\frac{\alpha}{2}
\]
и
\[
U_2=\alpha\cdot\left(\frac{1^2}{2}-\frac{2^2}{3}\right)=-\frac{5\alpha}{6}.
\]
В таком случае,
\[
A=U_1-U_2=\frac{4\alpha}{3}.
\]