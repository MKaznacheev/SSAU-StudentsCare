\subsection{9}

Равномерно возрастающая сила в начале пути длиной $S_{12}=12\;\text{м}$ равна $F_1=10\;\text{Н}$, а в его конце --- $F_2=46\;\text{Н}$. Очевидно (но не особо), что
\[
F=aS+b,
\]
где $a$ и $b$ --- некоторые постоянные, а $t$ --- время. При этом,
\[
\left\{
\begin{NiceArray}{l}
F(0)=F_1 \\
F(S_{12})=F_2
\end{NiceArray}
\right.
\LR
\left\{
\begin{NiceArray}{l}
b=10 \\
12a+b=46
\end{NiceArray}
\right.
\LR
\left\{
\begin{NiceArray}{l}
b=10 \\
a=3
\end{NiceArray}
\right.\;.
\]
Тогда
\[
F=3S+10.
\]

Из предположения (вновь не особо очевидного) о том, что векторы $\vec S$ и $\vec F$ сонаправленны и по определению, заключим, что
\[
\delta A=\vec F\cdot d\vec S=F\cdot dS=(3S+10)\,dS.
\]
Тогда
\[
A=\int\limits_0^{S_{12}}(3S+10)\,dS=\left.\frac{3}{2}S^2+10S\right\vert_0^{S_{12}}=\frac{3}{2}S_{12}^2+10S_{12}=336\;\text{Дж}.
\]