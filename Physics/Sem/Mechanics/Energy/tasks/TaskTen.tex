\subsection{17}

Уклон горы составляет $h=3\;\text{м}$ на каждые $l=100\;\text{м}$ пути. Значит, синус угла наклона равен
\[
\sin\alpha=\frac{h}{l}=0{,}03.
\]
В гору движется автомобиль массой
\[
m=1{,}8\;\text{т}=1800\;\text{кг}.
\]
При этом коэффициент трения равен $\mu=0{,}1$.

Запишем уравнение движения машины:
\[
\vec F+\vec F_\text{тр}+m\vec g+\vec N=m\vec a,
\]
где сила $\vec F$ --- результат работы двигателя. Будем считать, что машина движется равномерно. Зададим две оси: $Ox$, перпендикулярную плоскости горы, и $Oy$, направленную по движению машины. Тогда
\[
Ox:\quad -mg\cos\alpha+N=0\LR N=mg\cos\alpha=mg\sqrt{1-\sin^2\alpha}
\]
и
\[
Oy:\quad F-F_\text{тр}-mg\sin\alpha=0\LR F=F_\text{тр}+mg\sin\alpha.
\]
При этом
\[
F_\text{тр}=\mu N=\mu mg\sqrt{1-\sin^2\alpha}.
\]
В таком случае,
\[
F=\mu mg\sqrt{1-\sin^2\alpha}+mg\sin\alpha\approx2292{,}41\;\text{Н}.
\]
Машина проехала путь $S=5000\;\text{м}$. Тогда
\[
A=\int\limits_0^S\vec F\cdot d\vec s=\left.Fs\right\vert_0^S=FS\approx1{,}15\cdot10^7\;\text{Дж}.
\]