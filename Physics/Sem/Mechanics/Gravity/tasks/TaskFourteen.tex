\subsection{21}

Тело выпущено со скоростью $v_0=200\;\text{м/с}$ под углом $\alpha=60^\circ$ к горизонту. Как было показано в задаче 17,
\[
S=\frac{v_0^2\sin2\alpha}{g}\approx3534{,}8\;\text{м}
\]
и
\[
H=\frac{v_0^2\sin^2\alpha}{2g}\approx1530{,}61\;\text{м}.
\]
В наивысшей точке траектории нормальное ускорение равно ускорению свободного падания, а скорость --- её начальной горизонтальной составляющей,то есть
\[
g=\frac{(v_0\cos\alpha)^2}{R}\LR R=\frac{(v_0\cos\alpha)^2}{g}\approx1020{,}41\;\text{м}.
\]