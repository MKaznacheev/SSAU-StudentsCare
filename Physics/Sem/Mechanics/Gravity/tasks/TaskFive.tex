\subsection{12}

Проекции скорости тела на горизонтальную и вертикальную оси (ось $Ox$ направлена в сторону движения тела, а $Oy$ вверх) изменяются по законам
\begin{equation}\label{eq:12.1}
v_x(t)=v_0,\quad v_y(t)=-gt
\end{equation}
соответсвенно, причём $v_0=15\;\text{м/с}$. Высота, с которой оно было брошено, равна $H=25\;\text{м}$, а его координаты зависят от времени, как
\begin{equation}\label{eq:12.2}
x(t)=v_{0x}t=v_0t,\quad y(t)=H-\frac{gt^2}{2}
\end{equation}
(система отсчёта выбрана таким образом, что $x_0=0\;\text{м}$, $y_0=H$). Время движения $\tau$ является временем достижения точки с координатой $0$ по оси ординат:
\[
y(\tau)=0\LR H-\frac{g\tau^2}{2}=0\LR\tau=\sqrt\frac{2H}{g}\approx2{,}26\;\text{с}.
\]
Путь, пройденный телом в горизонтальном направлении составит
\[
S_x=x(\tau)=v_0\tau=33{,}9\;\text{м}.
\]
Скорость в момент падения будет равна
\[
v(\tau)=\sqrt{v_x^2(\tau)+v_y^2(\tau)}=\sqrt{v_0^2+g^2\tau^2}\approx26{,}75\;\text{м/с}.
\]
Угол $\varphi$ траектории с горизонтом (лучше сказать, угол между вектором скорости и осью абсцисс) в момент падения найдём из соотношения
\[
\cos\varphi=\frac{v_x(\tau)}{v(\tau)}=\frac{v_0}{v(\tau)}\LR\varphi=\arccos\frac{v_0}{v(\tau)}\approx55,89^\circ.
\]