\subsection{9}

Начальная скорость тела направлена вверх (как и ось $Oy$) и равна
\[
v_0=40\;\text{м/с}.
\]
Его скорость изменяется по формуле~\eqref{eq:8.1}. Времени $T$, через которое тело достигнет верхней точки траектории, соответствует нулевая скорость:
\begin{equation}\label{eq:9.1}
0=v_0-gT\LR T=\frac{v_0}{g}.
\end{equation}
Зависимость координаты тела от времени имеет вид~\eqref{eq:8.2}. Проекция перемещения тела вычисляется по формуле
\[
\Delta r_y(t)=y-y_0=v_0t-\frac{gt^2}{2},
\]
а путь ---
\[
S_1(t)=y-y_0=v_0t-\frac{gt^2}{2}
\]
до достижения верхней точки траектории (отсчёт времени ведётся с момента броска) и
\[
S_2(t)=\frac{gt^2}{2}
\]
--- после (отсчёт времени ведётся с момента достижения верхней точки траектории) из предположения о том, что начальная скорость движения из верхней точки равна нулю. Тогда за время $\tau=6$, тело совершило перемещение
\[
\Delta r_y(\tau)=63{,}6\;\text{м}
\]
и прошло путь
\[
S(6)=S_1(T)+S_2(\tau-T)=\frac{g\tau^2}{2}-\tau v_0+\frac{v_0^2}{g}\approx99{,}67\;\text{м}.
\]
Тогда
\[
\frac{S(6)}{\Delta r_y(\tau)}\approx1{,}57.
\]