\subsection{16}

Тело, брошенное под углом $\alpha$ к горизонту совершило путь в направлении оси абсцисс длиной $S=20\;\text{м}$. Проекции скорости тела на горизонтальное и вертикальное направление имеют вид
\begin{equation}\label{eq:16.1}
v_x=v_{0x}=v_0\cos\alpha,\quad v_y=v_{0y}-gt=v_0\sin\alpha-gt.
\end{equation}
При этом его координаты вычисляются по формулам
\begin{equation}\label{eq:16.2}
x=v_{0x}t=v_0\cos\alpha\cdot t,\quad y=v_{0y}t-\frac{gt^2}{2}=v_0\sin\alpha\cdot t-\frac{gt^2}{2}
\end{equation}
\big(считаем, что бросок произошёл с позиции $(0,0)$\big). Обозначим полное время движения через $T$. Тогда
\[
S=x(T)\LR S=v_0\cos\alpha\cdot T\LR T=\frac{S}{v_0\cos\alpha}.
\]
Также
\[
y(T)=0\LR \left(v_0\sin\alpha-\frac{gT}{2}\right)T=0\Rightarrow T=\frac{2v_0\sin\alpha}{g}.
\]
Тело поднялось на максимальную высоту спустя время $\tau$. При этом
\[
v_y(\tau)=0\LR v_0\sin\alpha-g\tau=0\LR\tau=\frac{v_0\sin\alpha}{g}.
\]
Заметим, что $T=2\tau$. Известно, что в этот момент горизонтальная проекция его скорости равнялась
\[
v_x(\tau)=10\;\text{м/с}.
\] 
То есть
\[
v_0\cos\alpha=10.
\]
Подставляя это в выражение для $T$, найдём, что
\[
T=2\;\text{с},
\]
а значит,
\[
\tau=1\;\text{с}.
\]