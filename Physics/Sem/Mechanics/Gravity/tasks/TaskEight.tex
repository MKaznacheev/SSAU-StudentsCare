\subsection{15}

Пусть частицы начали движение из одной точки с координатами $(0,H)$ в противоположные стороны. При этом проекции их скоростей вычисляются по формуам
\begin{gather*}
v_x=-v_0=-3\;\text{м/с},\quad v_y=-gt, \\
V_x=V_0=4\;\text{м/с},\quad V_y=-gt.
\end{gather*}
Значит,
\[
v=\sqrt{v_x^2+v_y^2}=\sqrt{v_0^2+g^2t^2}
\]
и
\[
V=\sqrt{V_x^2+V_y^2}=\sqrt{V_0^2+g^2t^2}.
\]
Найдём момент времени $\tau$, когда векторы $\vec v$ и $\vec V$ будут перпендикулярны:
\[
(\vec v,\vec V)=0\LR v_xV_x+v_yV_y=0\LR -v_0V_0+g^2\tau^2=0\LR\tau=\frac{\sqrt{v_0V_0}}{g}\approx0{,}35\;\text{с}.
\]