\subsection{22}

С высоты $H=15\;\text{м}$ ($x_0=0\;\text{м}$) со скоростью $v_0=20\;\text{м/с}$ под углом $\alpha=30^\circ$ к горизонту было брошено тело. Оно будет двигаться по закону
\begin{equation}\label{eq:22.1}
\left\{
\begin{NiceArray}{l}
	x=v_{0x}t=v_0\cos\alpha\cdot t \\
	y=H+v_{0y}t-\dfrac{gt^2}{2}=H+v_0\sin\alpha\cdot t-\dfrac{gt^2}{2}
\end{NiceArray}
\right.\;.
\end{equation}
Так,
\[
x(t)=10\sqrt3\cdot t\approx17{,}32t
\]
и
\[
y(t)=15+10t-4{,}9t^2.
\]
Через время $t_1=2\;\text{с}$ тело окажется в точке с координатами $(34{,}64;15{,}4)$.

Тело упадёт на землю, когда
\[
y=0\LR4{,}9\tau^2-10\tau-15=0\Rightarrow\tau=\frac{50+5\sqrt{394}}{49}\approx3{,}05\;\text{с}.
\]