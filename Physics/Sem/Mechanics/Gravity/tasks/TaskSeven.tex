\subsection{14}

Начальная скорость камня, брошенного горизонтально равна
\[
v_0=30\;\text{м/с}.
\]
Проекции скорости камня изменяются по закону~\eqref{eq:12.1}. В конце второй секунды (то есть через $\tau=2\;\text{с}$) скорость камня составит
\[
v(\tau)=\sqrt{v_x^2(\tau)+v_y^2(\tau)}=\sqrt{v_0^2+g^2\tau^2}\approx35{,}84\;\text{м/с}.
\]
Его тангенциальное ускорение равно
\[
a_\tau=g\cos\varphi,
\]
где $\varphi$ --- угол между векторами $\vec a_\tau$ и $\vec g$. При этом,
\[
\cos\varphi=\frac{|v_y|}{v}=\frac{gt}{\sqrt{v_0^2+g^2t^2}}=\frac{1}{\sqrt{\left(\dfrac{v_0}{gt}\right)^2+1}}.
\]
Тогда
\[
a_\tau=\frac{g}{\sqrt{\left(\dfrac{v_0}{gt}\right)^2+1}}=\frac{1}{\sqrt{\left(\dfrac{v_0}{g^2t}\right)^2+\dfrac{1}{g^2}}}.
\]
Значит,
\[
a_\tau(\tau)\approx5{,}36\;\text{м/с}^2.
\]
Наконец,
\[
a_n(\tau)=\sqrt{g^2-a_\tau^2(\tau)}\approx8{,}2\;\text{м/с}^2.
\]