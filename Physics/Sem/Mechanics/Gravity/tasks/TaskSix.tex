\subsection{13}

Пусть дальность полёта тела, брошенного горизонтально, равняется высоте, с которой его бросили. Его координаты изменяются по закону~\eqref{eq:12.2}. Как было показано в предыдущей задаче, время движения тела составит
\[
\tau=\sqrt\frac{2H}{g}.
\]
Тогда начальную скорость найдём из соотношения
\[
S_x(\tau)=x(\tau)=H.
\]
Так,
\[
v_0\tau=H\LR v_0=\frac{H}{\tau}=H\cdot\sqrt\frac{g}{2H}=\sqrt\frac{gH}{2}.
\]