\subsection{23}

Тело, брошенное с высоты $H=2\;\text{м}$ под углом $\alpha=45^\circ$ к горизонту, приземлилось на расстоянии $S=43\;\text{м}$ от места броска. Его координаты изменяются по закону~\eqref{eq:22.1}. Заметим, что
\[
y(\tau)=0\LR H+v_0\sin\alpha\cdot\tau-\frac{g\tau^2}{2}=0\LR v_0\sin\alpha=\frac{g\tau}{2}-\frac{H}{\tau}.
\]
В нашем случае $\sin\alpha=\cos\alpha$, а значит,
\[
S=x(\tau)=v_0\cos\alpha\cdot\tau=v_0\sin\alpha\cdot\tau=v_0\sin\alpha=\frac{g\tau^2}{2}-H.
\]
Тогда
\[
\tau=\sqrt\frac{2(S+H)}{g}\approx3{,}03\;\text{с}.
\]