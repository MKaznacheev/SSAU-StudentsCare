\subsection{8}

Начальная скорость тела направлена вверх (как и ось ординат) и равна
\[
v_0=9\;\text{м/с}.
\]
Скорость тела в проекции на ось $Oy$ изменяется по закону
\begin{equation}\label{eq:8.1}
v_y(t)=v_{0y}-gt=v_0-gt,
\end{equation}
а его координата ---
\begin{equation}\label{eq:8.2}
y(t)=y_0+v_{0y}t-\frac{gt^2}{2}=y_0+v_0t-\frac{gt^2}{2}.
\end{equation}
Найдём время $\tau$, по прошествию которого скорость тела уменьшится в три раза:
\[
3v_y(\tau)=v_0\LR 3v_0-3g\tau=v_0\LR\tau=\frac{2v_0}{3g}.
\]
Поскольку $\tau$ явно меньше времени подъёма тела (вычисляется по формуле~\eqref{eq:9.1}), то путь, пройденный им за это время, равен
\[
S=y(\tau)-y_0=v_0\tau-\frac{g\tau^2}{2}=\frac{2v_0^2}{3g}-\frac{2v_0^2}{9g}=\frac{4v_0^2}{9g}\approx3{,}67\;\text{м}.
\]