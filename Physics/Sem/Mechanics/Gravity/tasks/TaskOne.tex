\subsection{6}

Тело падает свободно, а значит его начальная скорость равна нулю:
\[
v_0=0\;\text{м/с}.
\]
Тогда путь зависит от времени, как
\begin{equation}\label{eq:6.1}
S(t)=\frac{gt^2}{2}.
\end{equation}
Обозначим время его движения через $\tau$. Известно, что
\[
S(\tau)-S(\tau-1)=\frac{S(\tau)}{2},
\]
то есть
\[
S(\tau)=2S(\tau-1).
\]
Учитывая выражение~\eqref{eq:6.1},
\[
\frac{g\tau^2}{2}=\frac{2g(\tau-1)^2}{2}\LR \tau^2=2(\tau-1)^2\LR |\tau|=\sqrt2|\tau-1|.
\]
Поскольку время является положительной величиной, то единственным подходящим раскрытием модулей будет
\[
\tau=\sqrt2(\tau-1)\LR\tau=\frac{1}{\sqrt2-1}=\sqrt2+1\approx2{,}41\;\text{с}.
\]
Высота падения тела равна
\[
h=S(\tau)\approx28{,}56\;\text{м}.
\]