\subsection{24}

Тело брошено под углом $\alpha$ к горизонту. При этом оно достигает максимальной высоты $H=20\;\text{м}$ при значении координаты $x_1=1000\;\text{м}$. Как было показано в задаче~17,
\[
H=\frac{v_0^2\sin^2\alpha}{2g}\LR\sin\alpha=\frac{\sqrt{2gH}}{v_0},
\]
а в задаче~16 ---
\[
\tau=\frac{v_0\sin\alpha}{g}=\frac{\sqrt{2gH}}{g}.
\]
Также известно, что
\[
x_1=x(\tau)=v_{0x}\tau=\frac{v_{0x}\sqrt{2gH}}{g}\LR v_{0x}=\frac{gx_1}{\sqrt{2gH}}.
\]
При этом
\[
v(\tau)=v_{0x}\approx479{,}82\;\text{м/с}.
\]