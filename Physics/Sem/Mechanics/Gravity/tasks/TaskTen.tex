\subsection{17}

Тело брошено со скоростью $v_0$ под углом $\alpha$ к горизонту. Проекции его скорости и его координаты меняются по законам~\eqref{eq:16.1} и~\eqref{eq:16.2} соответсвенно. Будем считать, что тело начало движение из точки с координатами $(0,0)$. Пусть тело поднялось на максимальную высоту за время $\tau$. Как было показано в предыдущей задаче,
\[
\tau=\frac{v_0\sin\alpha}{g}
\]
и
\[
T=2\tau=\frac{2v_0\sin\alpha}{g},
\]
где $T$ --- полное время полёта. Дальность полёта в таком случае равна
\[
S=x(T)=v_0\cos\alpha\cdot T=\frac{v_0^2\sin2\alpha}{g},
\]
а высота полёта ---
\[
H=y(\tau)=v_0\sin\alpha\cdot\tau-\frac{g\tau^2}{2}=\frac{v_0^2\sin^2\alpha}{2g}.
\]
Как видно, дальность полёта максимальна при наибольшем значении $\sin2\alpha$. Очевидно,
\[
\sin2\alpha=1\Rightarrow2\alpha=90^\circ\LR\alpha=45^\circ.
\]