\subsection{16}

Разность между максимальным и минимальным натяжением верёвки, к который привязан камень, составляет
\[
\Delta T=T_2-T_1=9{,}8\;\text{Н},
\]
где $T_2$ --- натяжение верёвки в нижней точки траектории (максимальное натяжение), а $T_1$ --- в верхней (минимальное натяжение). При этом, в нижней точке проекция уравнения движения камня на вертикальную ось, направленную вверх, имеет вид
\[
T_2-mg=ma_n\LR T_2=ma_n+mg,
\]
а в верхней ---
\[
-T_1-mg=-ma_n\LR T_1=ma_n-mg.
\]
Так,
\[
\Delta T=(ma_n+mg)-(ma_n-mg)=2mg\LR m=\frac{\Delta T}{2g}=0{,}5\;\text{кг}.
\]