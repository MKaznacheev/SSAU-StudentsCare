\subsection{4}

Сила тяготения, действующая на ракету на поверхности земли равна
\[
F_\text{п}=G\cdot\frac{m_\text{р}\cdot M_\text{з}}{R_\text{з}^2},
\]
где $m_\text{р}$ --- масса ракеты. Известно, что ракета поднялась на высоту, равную $R_\text{з}$. Теперь на неё действует сила
\[
F=G\cdot\frac{m_\text{р}\cdot M_\text{з}}{(R_\text{з}+R_\text{з})^2}=G\cdot\frac{m_\text{р}\cdot M_\text{з}}{4R_\text{з}^2}.
\]
Тогда
\[
\frac{F_\text{п}}{F}=\cfrac{G\cdot\cfrac{m_\text{р}\cdot M_\text{з}}{R_\text{з}^2}}{G\cdot\cfrac{m_\text{р}\cdot M_\text{з}}{4R_\text{з}^2}}=\frac{G\cdot m_\text{р}\cdot M_\text{з}}{R_\text{з}^2}\cdot\frac{4R_\text{з}^2}{G\cdot m_\text{р}\cdot M_\text{з}}=4.
\]