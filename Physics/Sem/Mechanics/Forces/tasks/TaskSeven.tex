\subsection{17}

За время $\tau=0{,}8\;\text{с}$ скорость пули изменилась от $v_0=800\;\text{м/с}$ до $v_1=200\;\text{м/с}$. Её масса равна $m=10\;\text{г}=0{,}01\;\text{кг}$. Сила сопротивления воздуха равна
\[
F=kv^2.
\]
Пренебрегая силой тяжести, запишем уравнение движения пули в проекции на направление, противоположное направлению её движения:
\[
F=ma\LR kv^2=m\frac{dv}{dt}\LR \frac{k}{m}\,dt=\frac{dv}{v^2}.
\]
Проинтегрируем полученное:
\[
\int\limits_0^\tau\frac{k}{m}\,dt=\int\limits_{v_0}^{v_1}\frac{dv}{v^2}\LR\left.\frac{kt}{m}\right\vert_0^\tau=\left.\frac{1}{v}\right\vert_{v_0}^{v_1}\LR\frac{k\tau}{m}-\frac{k\cdot0}{m}=\frac{1}{v_1}-\frac{1}{v_0}\LR k=\frac{m(v_0-v_1)}{\tau v_0v_1}.
\]
Так,
\[
k=4{,}6875\cdot10^{-5}\;\frac{\text{кг}}{\text{м}}.
\]