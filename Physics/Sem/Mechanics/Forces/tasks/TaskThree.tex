\subsection{13}

Жёсткости пружин, соединённых последовательно, равны
\[
k_1=0{,}3\;\frac{\text{кН}}{\text{м}}=300\;\frac{\text{Н}}{\text{м}}
\]
и
\[
k_2=0{,}8\;\frac{\text{кН}}{\text{м}}=800\;\frac{\text{Н}}{\text{м}}.
\]
Вторая пружина деформирована на $x_2=1{,}5\;\text{см}=0{,}015\;\text{м}$. Силы, с которыми пружины действуют друг на друга равны. Тогда, с учётом закона Гука,
\[
-k_1x_1=-k_2x_2\LR x_1=\frac{k_2}{k_1}\cdot x_2=0{,}04\;\text{м}.
\]