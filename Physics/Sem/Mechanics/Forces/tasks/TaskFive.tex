\subsection{15}

Радиус петли, как манёвра, совершаемого самолётом, равен $R=500\;\text{м}$. Линейная скорость самолёта равна
\[
v=360\;\text{км/ч}=100\;\text{м/с}.
\]
Нормальное ускорение равно
\[
a_n=\frac{v^2}{R}=20\;\text{м/с}^2.
\]
Масса лётчика составляет $m=70\;\text{кг}$. Направим ось $Oy$ вертикально вверх. Уравнение движения лётчика в проекции на эту ось в верхней точке траектории имеет вид
\[
-N_\text{в}-mg=-ma_n\LR N_\text{в}=m(a_n-g)=714\;\text{Н},
\]
где $N$ --- сила реакции кресла, численно равная весу:
\[
P_\text{в}=N_\text{в}=714\;\text{Н}.
\]

В нижней точке траектории проекция примет вид
\[
N_\text{н}-mg=ma_n\LR N_\text{н}=m(a_n+g)=2086\;\text{Н}.
\]
Значит,
\[
P_\text{н}=N_\text{н}=2086\;\text{Н}.
\]