\subsection{14}

Известно, что первая пружина имеет жёсткость
\[
k_1=2\;\frac{\text{кН}}{\text{м}}=2000\;\frac{\text{Н}}{\text{м}},
\]
а вторая ---
\[
k_2=6\;\frac{\text{кН}}{\text{м}}=6000\;\frac{\text{Н}}{\text{м}}.
\]
Пусть $x_1$ и $x_2$ --- изменение длин первой и второй пружин соответственно, $x$ --- изменение длины системы пружин, как целого, а $k$ --- её жёсткость. Рассмотрим случай, когда пружины соединены последовательно:
\[
x=x_1+x_2\LR\frac{F}{k}=\frac{F}{k_1}+\frac{F}{k_2}\LR k=\frac{k_1\cdot k_2}{k_1+k_2}=\frac{2000\cdot6000}{2000+6000}=1500\;\frac{\text{Н}}{\text{м}},
\]
поскольку на пружины и их систему в целом действуют одинаковые силы $F$.

Рассмотрим случай, когда пружины соединены параллельно. Тогда сила, действующая на систему, как целое, равна сумме сил, действующих на пружины по отдельности:
\[
F=F_1+F_2\LR kx=k_1x_1+k_2x_2.
\]
Заметим, что $x=x_1=x_2$ и, как следствие,
\[
kx=(k_1+k_2)x\LR k=k_1+k_2=2000+6000=8000\;\frac{\text{Н}}{\text{м}}.
\]