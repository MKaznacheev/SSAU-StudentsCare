\subsection{12}

Пусть масса спутника равна $m$. Тогда, если он движется у поверхности Земли, на него действует сила тяготения, равная по модулю
\[
F=G\cdot\frac{m\cdot M_\text{з}}{R_\text{з}^2}=mg.
\]
Модуль нормальной компоненты ускорения равен
\[
a_n=\frac{v^2}{R_\text{з}}.
\]
Очевидно, в проекции на ось, на которой лежат центры Земли и спутника и которая направлена от Земли к спутнику, уравнение движения спутника принимает вид
\begin{equation}\label{EqOne}
-F=-ma_n.
\end{equation}
Так,
\[
ma_n=F\LR m\frac{v^2}{R_\text{з}}=mg\LR v=\sqrt{gR_\text{з}}=\sqrt{9{,}8\cdot6{,}371\cdot10^6}\approx7901,6\;\text{м/с}.
\]
При этом,
\[
\omega=\frac{v}{R_\text{з}}=\sqrt{\frac{g}{R_\text{з}}}
\]
и
\[
T=\frac{2\pi}{\omega}=2\pi\sqrt{\frac{R_\text{з}}{g}}\approx5066{,}1\;\text{с}.
\]

В том случае, если спутник находится на высоте $H=7000\;\text{км}=7\cdot10^6\;\text{м}$ от земли, на него действует сила
\[
F=G\cdot\frac{m\cdot M_\text{з}}{(R_\text{з}+H)^2},
\]
а модуль нормального ускорения равен
\[
a_n=\frac{v^2}{R_\text{з}+H}.
\]
Тогда, руководствуясь уравнением~\eqref{EqOne}, заключим, что
\[
ma_n=F\LR m\frac{v^2}{R_\text{з}+H}=G\cdot\frac{m\cdot M_\text{з}}{(R_\text{з}+H)^2}\LR v=\sqrt{G\cdot\frac{M_\text{з}}{R_\text{з}+H}}.
\]
Так,
\[
v=\sqrt{6{,}67\cdot10^{-11}\cdot\frac{5{,}97\cdot10^{24}}{6{,}371\cdot10^6+7\cdot10^6}}\approx5457{,}2\;\text{м/с}.
\]
При этом,
\[
\omega=\frac{v}{R_\text{з}}=\sqrt{G\cdot\frac{M_\text{з}}{R_\text{з}^2(R_\text{з}+H)}}
\]
и
\[
T=\frac{2\pi}{\omega}=2\pi\sqrt{\frac{R_\text{з}^2(R_\text{з}+H)}{G\cdot M_\text{з}}}\approx3667{,}7\;\text{с}.
\]