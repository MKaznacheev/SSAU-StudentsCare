\subsection{14}

Материальная точка совершает движение по окружности ($v_0=0\;\text{м/с}$) радиусом $R=0{,}3\;\text{м}$, причём $a_\tau=5\;\text{м/с}^2$. Поскольку $a_\tau=\text{const}$, то
\[
a_\tau=\frac{\Delta v}{\Delta t}=\frac{v-v_0}{t-t_0}=\frac{v}{t}\Rightarrow v=a_\tau t.
\]
Тогда
\[
a_n=\frac{v^2}{R}=\frac{a^2_\tau t^2}{R}
\]
и
\[
a=\sqrt{a^2_\tau+a^2_n}=\sqrt{a^2_\tau+\frac{a^4_\tau t^4}{R^2}}=a_\tau\sqrt{1+\frac{a^2_\tau t^4}{R^2}}.
\]
Значит,
\[
a(3)=5\sqrt{1+\frac{5^2\cdot3^4}{0{,}3^2}}\approx750{,}02\;\text{м/с}^2.
\]