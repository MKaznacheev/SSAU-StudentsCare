\subsection{11}

Пусть радиус колеса равен $R=0{,}1\;\text{м}$, а его угловое ускорение --- $\varepsilon=3{,}14\;\text{рад/с}^2$. Рассмотрим момент времени $t=1\;\text{с}$. Будем считать, что $\omega_0=0\;\text{рад/с}$. Тогда
\[
\omega=\omega_0+\varepsilon t=0+3{,}14\cdot1=3{,}14\;\text{рад/с}.
\]
Значит,
\[
v=\omega R=1{,}57\cdot0{,}1=0{,}314\;\text{м/с}.
\]
Тангенциальное ускорение в таком случае равно
\[
a_\tau=\frac{v-v_0}{t}=\frac{0{,}314-0}{1}=0{,}314\;\text{м/с}^2,
\]
а нормальное ---
\[
a_n=\frac{v^2}{R}=\frac{0{,}314^2}{0{,}1}\approx0{,}986\;\text{м/с}^2.
\]
Полное ускорение найдём по формуле
\[
a=\sqrt{a_\tau^2+a_n^2}=\sqrt{0{,}314^2+0{,}986^2}\approx1{,}035\;\text{м/с}^2.
\]
Угол между вектором полного ускорения точки на ободе колеса и её радиус-вектором равен
\begin{multline*}
\alpha=\arccos\frac{(\vec a,\vec r)}{aR}=\arccos\frac{a_\tau r_\tau+a_nr_n}{aR}= \\
\arccos\frac{a_nR}{aR}=\arccos\frac{a_n}{a}= \\
=\arccos\frac{0{,}986}{1{,}035}\approx0{,}309\;\text{рад}\approx17^\circ42'.
\end{multline*}