\subsection{15}

По окружности радиусом $R=5\;\text{м}$ движется материальная точка. В некоторый момент времени нормальное ускорение равно $a_n=3{,}2\;\text{м/с}^2$, а угол между векторами полного и нормального ускорений --- $\varphi=60^\circ$. Тогда скорость равна
\[
v=\sqrt{a_nR}=\sqrt{3{,}2\cdot5}=4\;\text{м/с}.
\]
Полное ускорение вычисляется по формуле
\[
a=\sqrt{a^2_\tau+a^2_n}.
\]
Как видно из задачи 11,
\[
\cos\alpha=\frac{a_n}{a}=\frac{a_n}{\sqrt{a^2_\tau+a^2_n}}\Rightarrow a_n=\frac{a_\tau\cos\alpha}{\sin\alpha}=a_\tau\tg\alpha=3{,}2\cdot\tg60^\circ\approx5{,}5\;\text{м/с}^2.
\]