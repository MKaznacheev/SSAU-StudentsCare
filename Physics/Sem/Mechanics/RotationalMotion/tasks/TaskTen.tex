\subsection{18}

Точки на окружности диска радиусом $R=0{,}2\;\text{м}$ вращаются по закону
\[
\varphi=3-2t+4t^2.
\]
Тогда
\[
\omega=\frac{d\varphi}{dt}=-2+2\cdot4\cdot t=-2+8t
\]
и
\[
v=\omega R=-2R+8Rt=-2\cdot0{,}2+8\cdot0{,}2\cdot t=-0{,}4+1{,}6t.
\]
Найдём ускорения:
\begin{gather*}
a_\tau=\frac{dv}{dt}=1{,}6\;\text{м/с}^2, \\
a_n=\frac{v^2}{R}=\frac{(-0{,}4+1{,}6t)^2}{R}=\frac{(-0{,}4+1{,}6t)^2}{0{,}2}, \\
a=\sqrt{a^2_\tau+a^2_n}.
\end{gather*}
Наконец,
\[
a_n(10)=\frac{(-0{,}4+1{,}6\cdot10)^2}{0{,}2}=1216{,}8\;\text{м/с}^2
\]
и
\[
a(10)=\sqrt{1{,}6^2+1216{,}8^2}\approx1216{,}8\;\text{м/с}^2.
\]