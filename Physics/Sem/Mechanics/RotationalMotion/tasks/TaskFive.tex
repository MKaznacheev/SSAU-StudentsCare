\subsection{13}

Скорость движения точек на ободе колеса радиусом $R=0{,}1\;\text{м}$ задаётся уравнением
\[
v=At+Bt^2,
\]
где $A=0{,}03\;\text{м/с}^2$, $B=0{,}01\;\text{м/с}^2$. Тангенциальное ускорение равно
\[
a_\tau=\frac{dv}{dt}=A+2Bt,
\]
а нормальное ---
\[
a_n=\frac{v^2}{R}=\frac{\left(At+Bt^2\right)^2}{R}.
\]
Полное ускорение в таком случае равно
\[
a=\sqrt{a_\tau^2+a_n^2}=\sqrt{(A+2Bt)^2+\frac{\left(At+Bt^2\right)^4}{R^2}}. 
\]
Угол $\alpha$ между вектором полного ускорения точки на ободе и её радиус-вектором может быть вычислен по формуле
\[
\alpha=\arccos\frac{a_n}{a}=\arccos\cfrac{\cfrac{\left(At+Bt^2\right)^2}{R}}{\sqrt{(A+2Bt)^2+\cfrac{\left(At+Bt^2\right)^4}{R^2}}}
\]
(см. задачу 11).

Так,
\begin{gather*}
\alpha(0)=90^\circ\approx1{,}57\;\text{рад},\quad\alpha(1)\approx1{,}26\;\text{рад}\approx72^\circ15',\quad\alpha(2)\approx0{,}61\;\text{рад}\approx34^\circ59' \\
\alpha(3)\approx0{,}27\;\text{рад}\approx15^\circ31',\quad\alpha(4)\approx0{,}14\;\text{рад}\approx7^\circ59',\quad\alpha(5)\approx0{,}08\;\text{рад}\approx4^\circ38'.
\end{gather*}