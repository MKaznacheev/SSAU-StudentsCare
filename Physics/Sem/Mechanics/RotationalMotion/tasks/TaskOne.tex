\subsection{9}

Тело при равноускоренном вращении достигло угловой скорости $\omega=20\;\text{рад/с}$ спустя $N=10$ оборотов от начала движения. Угловое ускорение найдём из соотношения
\begin{equation}\label{Eq9.1}
\varphi=\varphi_0+\omega_0t+\frac{\varepsilon t^2}{2}\LR\varphi=\frac{\varepsilon t^2}{2}
\end{equation}
с учётом того, что $\varphi_0=0\;\text{рад}$ и $\omega_0=0\;\text{рад/с}$ по условию.
Из уравнения
\[
\varepsilon=\frac{\omega-\omega_0}{t}\LR\varepsilon=\frac{\omega}{t}
\]
найдём время:
\[
t=\frac{\omega}{\varepsilon},
\]
а угол --- из соотношения $\varphi=2\pi N$. Подставим полученное в уравнение~\eqref{Eq9.1}:
\[
2\pi N=\cfrac{\varepsilon\left(\cfrac{\omega}{\varepsilon}\right)^2}{2}\LR\varepsilon=\frac{\omega^2}{4\pi N}.
\]
Так,
\[
\varepsilon=\frac{20^2}{4\pi\cdot10}\approx3{,}18\;\text{рад/с}^2.
\]
При этом направления векторов $\vec\varepsilon$ и $\vec\omega$ совпадают.