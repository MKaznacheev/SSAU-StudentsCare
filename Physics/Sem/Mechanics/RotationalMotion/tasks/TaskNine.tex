\subsection{17}

Материальная точка совершает равномерное движение ($v=5\;\text{м/с}$) по окружности радиусом $R=5\;\text{м}$. При этом $\Delta r=0\;\text{м}$ в начальный момент времени. Путь точки вычисляется по по формуле
\[
S=vt=5t.
\]
На рисунке~\ref{PicOne} показан соответствующий график.

\begin{figure}[h!]
	\begin{center}
		\begin{tikzpicture}
			\draw [arrows = {-Stealth[scale=1]}, thick] (0,0) -- (8,0);
			\draw [arrows = {-Stealth[scale=1]}, thick] (0,0) -- (0,6);
			\draw (8,-0.2) node {$t$};
			\draw (-0.2,6) node {$S$};
			\draw (-0.15,-0.15) node {$0$};
			\draw (-0.1,5) -- (0.1,5);
			\draw (-0.3,5) node {$5$};
			\draw (1,-0.1) -- (1,0.1);
			\draw (1,-0.3) node {$1$};
			\draw (0,0) -- (1.2,6);
			\draw [densely dashed] (0,5) -- (1,5) -- (1,0);
		\end{tikzpicture}
		\caption{}\label{PicOne}
	\end{center}
\end{figure}

Угол поворота радиус вектора вычисляется, как
\[
\varphi=\varphi_0+\omega_0t.
\]
Поскольку $\varphi=0^\circ$, а так же
\[
\omega_0=\frac{v}{R},
\]
то
\[
\varphi=\frac{v}{R}\cdot t=\frac{5}{5}\cdot t=t.
\]
Заметим, что
\[
\sin\frac{\varphi}{2}=\frac{\Delta r}{2R}\LR\Delta r=2R\sin\frac{t}{2}=2\cdot5\sin\frac{t}{2}=10\sin\frac{t}{2}
\]
при $0\Leq\varphi\Leq\pi$. Это вытекает из взаимного расположения векторов $\vec r$, $\vec r_0$ и $\Delta\vec r$. Чтобы избавиться от ограничения на угол, запишем
\[
\Delta r=10\left|\sin\frac{t}{2}\right|,
\]
поскольку $\sin\varphi=-\sin(2\pi-\varphi)$. На рисунке~\ref{PicTwo} показан соответствующий график.

\begin{figure}[h!]
	\begin{center}
		\begin{tikzpicture}
			\draw [arrows = {-Stealth[scale=1]}, thick] (0,0) -- (8,0);
			\draw [arrows = {-Stealth[scale=1]}, thick] (0,0) -- (0,10);
			\draw (8,-0.2) node {$t$};
			\draw (-0.3,10.1) node {$\Delta r$};
			\draw (-0.15,-0.15) node {$0$};
			 \draw [smooth, domain=0:8] plot (\x,{abs(10*sin(\x/2 r))});
		\end{tikzpicture}
		\caption{}\label{PicTwo}
	\end{center}
\end{figure}

