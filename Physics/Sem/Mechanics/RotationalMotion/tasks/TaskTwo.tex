\subsection{10}

Скорость тела, движущегося равнозамедленно, за время $t=1\;\text{мин}$ снизило скорость с $\omega_1=300\;\text{об/мин}$ до $\omega_2=180\;\text{об/мин}$. Тогда
\[
\varepsilon=\frac{|\omega_2-\omega_1|}{t}=\frac{|180-300|}{1}=120\;\text{об/мин}^2
\]
или
\[
\varepsilon=120\cdot\frac{2\pi}{60^2}\approx0{,}21\;\text{рад/с}^2.
\]
Зависимость числа оборотов от времени (с учётом $n_0=0$) имеет вид:
\[
n=\omega_1t-\frac{\varepsilon t^2}{2}=300\cdot1-\frac{120\cdot1^2}{2}=240.
\]