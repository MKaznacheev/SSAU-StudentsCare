\subsection{12}

Точка движется по окружности так, что $s=A+Bt+Ct^2$, где $A=5\;\text{м}$, $B=-2\;\text{м/с}$, $C=1\;\text{м/с}^2$. Рассмотрим момент времени $\tau=3\;\text{с}$. При этом в момент $t_1=2\;\text{с}$ нормальное ускорение было равно $a_{n1}=0{,}5\;\text{м/с}^2$.

Заметим, что
\[
v=\frac{ds}{dt}=B+2Ct\Rightarrow v(\tau)=B+2C\tau=-2+2\cdot1\cdot3=4\;\text{м/с},
\]
и
\[
a_\tau=\frac{dv}{dt}=2C=2\;\text{м/с}^2.
\]
При этом
\[
a_n=\frac{v^2}{R}\Rightarrow R=\frac{v^2(t_1)}{a_{n1}}=\frac{(B+2Ct_1)^2}{a_{n1}}=\frac{(-2+2\cdot1\cdot2)^2}{0{,}5}=8\;\text{м}.
\]
Значит,
\[
a_n(\tau)=\frac{v^2(\tau)}{R}=\frac{4^2}{8}=2\;\text{м/с}^2.
\]
Полное ускорение составляет
\[
a(\tau)=\sqrt{a_\tau^2+a^2_n(\tau)}=\sqrt{2^2+2^2}\approx2{,}83\;\text{м/с}^2.
\]
