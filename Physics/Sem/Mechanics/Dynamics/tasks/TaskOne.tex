\subsection{8}

При начальной скорости $v_0=10\;\text{м/с}$ тормозной путь машины составил $S=8\;\text{м}$. Найдём ускорение автомобиля из формул
\[
S=v_0t+\frac{at^2}{2}
\]
и
\[
v=v_0+at,
\]
где $v$ --- конечная скорость, равная $0\;\text{м/с}$. Так,
\[
t=\frac{v-v_0}{a}
\]
и
\[
S=v_0\frac{v-v_0}{a}+\frac{(v-v_0)^2}{2a}=\frac{v^2-v_0^2}{2a}\LR a=\frac{v^2-v_0^2}{2S}
\]
(имеется ввиду проекция ускорения на направление движения). Сила реакции опоры (дороги) численно равна весу машины:
\[
N=mg.
\]
При этом
\[
F_\text{тр}=\mu N=\mu mg
\]
и
\[
-F_\text{тр}=ma,
\]
поскольку сила трения --- единственная сила, действующая на автомобиль в горизонтальном направлении (ось направлена в сторону движения). В таком случае,
\[
-\mu mg=ma\LR\mu=-\frac{a}{g}=\frac{v_0^2-v^2}{2gS}\approx0{,}64.
\]