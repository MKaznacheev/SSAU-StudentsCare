\subsection{11}

Направим ось $Oy$ вертикально вверх. Пусть масса левого груза машины Атвуда равна $m_2$, а правого --- $m_1$, причём
\[
m_1>m_2.
\]
Уравнения движения грузов принимают вид
\[
\left\{
\begin{NiceArray}{l}
m_1\vec g+\vec T=m_1\vec a_1 \\
m_2\vec g+\vec T=m_2\vec a_2
\end{NiceArray}
\right.\;,
\]
где $a_1=a_2=a$, или в проекции на ось $Oy$ ---
\begin{equation}\label{EqOne}
\left\{
\begin{NiceArray}{l}
-m_1g+T=-m_1a \\
-m_2g+T=m_2a
\end{NiceArray}
\right.\;.
\end{equation}
Вычтем из второго уравнения системы~\eqref{EqOne} первое и выразим ускорение:
\[
-m_2g+m_1g=m_2a+m_1a\LR a=\frac{(m_1-m_2)g}{m_1+m_2}.
\]
Теперь в первое уравнение системы~\eqref{EqOne} подставим найденное выражение для ускорение и выразим силу натяжения нити:
\[
-m_1g+T=-m_1\frac{(m_1-m_2)g}{m_1+m_2}\LR T=\frac{2m_1m_2g}{m_1+m_2}.
\]
На ось блока действует сила
\[
F=2T=\frac{4m_1m_2g}{m_1+m_2}.
\]