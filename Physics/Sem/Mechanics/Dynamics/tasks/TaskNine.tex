\subsection{17}

Высота клина равна $h=1\;\text{м}$, а длина --- $l=2\text{м}$. Тогда угол между наклонной плоскостью и горизонталью равен
\[
\alpha=\arcsin\frac{h}{l}.
\]
Расположим ось $Ox$ по направлению движения тела, а ось $Oy$ --- вверх, перпендикулярно наклонной плоскости. Уравнение движения тела принимает вид
\[
m\vec g+\vec N +\vec F_\text{тр}=m\vec a,
\]
или в проекции на оси абсцисс и ординат соответсвенно,
\begin{gather*}
mg\cos(90^\circ-\alpha)-F_\text{тр}=ma, \\
-mg\sin(90^\circ-\alpha)+N=0,
\end{gather*}
то есть
\begin{gather*}
mg\sin\alpha-F_\text{тр}=ma, \\
N=mg\cos\alpha.
\end{gather*}
Значит,
\[
F_\text{тр}=\mu N=\mu mg\cos\alpha
\]
и
\begin{multline*}
mg\sin\alpha-\mu mg\cos\alpha=ma\LR \\
\LR a=g\left(\sin\arcsin\frac{h}{l}-\mu\cos\arcsin\frac{h}{l}\right)= \\
=\frac{g\left(h-\mu\sqrt{l^2-h^2}\right)}{l}.
\end{multline*}
Поскольку $\mu=0{,}15$, то
\[
a\approx3{,}63\;\text{м/с}^2.
\]
При этом
\[
x=x_0+v_0t+\frac{at^2}{2}
\]
и $v_0=0\;\text{м/с}$, а значит,
\[
l=x-x_0=\frac{at^2}{2}.
\]
Тогда
\[
t=\sqrt{\frac{2l}{a}}\approx1{,}05\;\text{с}.
\]