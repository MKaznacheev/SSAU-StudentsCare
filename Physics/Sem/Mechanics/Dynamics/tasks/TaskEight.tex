\subsection{16}

Массы грузов, связанных нитью, имеют массы $m_1=0{,}5\;\text{кг}$ и $m_2=0{,}7\;\text{кг}$. На груз массой $m_1$ действует сила $F=6\;\text{Н}$. Запишем уравнения движения грузов в проекциях на направление движения:
\[
\left\{
\begin{NiceArray}{l}
	-T+F=m_1a \\
	T=m_2a
\end{NiceArray}
\right.\;,
\]
где $T$ --- сила натяжения нити. Так,
\[
a=\frac{T}{m_2}
\]
и, как следствтие,
\[
-T+F=m_1\frac{T}{m_2}\LR T=\frac{m_2F}{m_1+m_2}=3{,}5\;\text{Н}.
\]
При этом
\[
a=\frac{m_2F}{m_2(m_1+m_2)}=\frac{F}{m_1+m_2}=5\;\text{м/с}^2.
\]