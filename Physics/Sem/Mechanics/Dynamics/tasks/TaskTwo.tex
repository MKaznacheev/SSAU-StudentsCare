\subsection{10}

Известно, что масса тела, лежащего на наклонной плоскости под углом $\alpha=20^\circ$ к горизонтали, равна $m=10\;\text{кг}$. Также на него действует горизонтально направленная сила $F=8\;\text{Н}$. Уравнение движения тела имеет вид
\[
\vec F+m\vec g+\vec N=m\vec a.
\]
Направим ось абсцисс вверх, перпендикулярно плоскости, на которой лежит тело, а ось ординат --- по направлению движения тела. Ось $Oy$ образует с вектором $\vec F$ угол $\alpha$, а с вектором $m\vec g$ --- $90^\circ-\alpha$. Спроецируем уравнение движения на заданные оси:
\[
\begin{array}{ll}
Ox:\quad & F\sin\alpha-mg\sin(90^\circ-\alpha)+N=0, \\
Oy:\quad & F\cos\alpha+mg\cos(90^\circ-\alpha)=ma.
\end{array}
\]
Сила $P$, с которой тело давит на плоскость, по модулю равна силе реакции опоры, а по направлению противоположна ей:
\begin{gather*}
P=N=mg\sin(90^\circ-\alpha)-F\sin\alpha=mg\cos\alpha-F\sin\alpha\approx89{,}4\;\text{Н}, \\
\vec P=-\vec N.
\end{gather*}
Ускорение тела равно
\[
a=\frac{F\cos\alpha}{m}+g\cos(90^\circ-\alpha)=\frac{F\cos\alpha}{m}+g\sin\alpha\approx4{,}1\;\text{м/с}^2.
\]