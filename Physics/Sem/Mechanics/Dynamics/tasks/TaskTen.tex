\subsection{18}

Масса тела, находящегося на горизонтальном столе равна $M=2\;\text{кг}$, а тел, связанных с ним нитью --- $m_1=0{,}5\;\text{кг}$ (справа) и $m_2=0{,}3\;\text{кг}$. Направим ось $Ox$ горизонтально вправо, ось $Oy$ --- вертикально вверх. Уравнение движения тела массой $M$ на ось абсцисс имеет вид
\[
-T_2+T_1=Ma,
\]
тела массой $m_2$ на ось ординат ---
\[
-m_2g+T_2=m_2a,
\]
тела массой $m_1$ на ту же ось ---
\[
-m_1g+T_1=-m_1a.
\]
Преобразуем последние два уравнения:
\[
T_2=m_2g+m_2a,\quad T_1=m_1g-m_1a,
\]
и подставим их в первое:
\[
a=\frac{m_1-m_2}{M+m_1+m_2}g=0{,}7\;\text{м/с}^2.
\]