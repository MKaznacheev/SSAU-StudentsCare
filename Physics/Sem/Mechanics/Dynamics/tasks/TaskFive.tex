\subsection{13}

Ускорение системы грузов массами $m_1=0{,}5\;\text{кг}$ и $m_2=0{,}6\;\text{кг}$ равно $a=4{,}9\;\text{м/с}^2$ и направлено вверх. Пусть $a'$ --- ускорение грузов относительно стола. Запишем уравнения движения:
\[
\left\{
\begin{NiceArray}{l}
	m_1\vec g+\vec N+\vec T_1+\vec F_\text{тр}-m_1\vec a=m_1\vec a'_1 \\
	m_2\vec g+\vec T_2-m_2\vec a=m_2\vec a'_2
\end{NiceArray}
\right.\;
\]
(под векторами $-m_1\vec a$ и $-m_2\vec a$ имеются ввиду силы, вызванные движением лифта), где $\vec T_1$ и $\vec T_2$ --- силы натяжения нити, приложенные к телам массы $m_1$ и $m_2$ соответственно. При этом
\[
T_1=T_2=T.
\]
Направим ось $Ox$ горизонтально вправо, а ось $Oy$ --- вертикально вверх. Запишем уравнения движения в проекциях на эти оси (первое --- на ось $Ox$ и $Oy$, а второе --- на ось $Oy$):
\[
\left\{
\begin{NiceArray}{l}
	-T+F_\text{тр}=-m_1a' \\
	-m_1g+N-m_1a=0 \\
	-m_2g+T-m_2a=-m_2a'
\end{NiceArray}
\right.\;
\LR
\left\{
\begin{NiceArray}{l}
	a'=\dfrac{T-F_\text{тр}}{m_1} \\
	N=m_1(a+g) \\
	a'=\dfrac{m_2g-T+m_2a}{m_2}
\end{NiceArray}
\right.\;.
\]
Приравняем первое и третье выражения:
\[
\frac{T-F_\text{тр}}{m_1}=\frac{m_2g-T+m_2a}{m_2}\LR T=\frac{m_1m_2(a+g)+m_2F_\text{тр}}{m_1+m_2}.
\]
При этом
\[
F_\text{тр}=\mu N=\mu m_1(a+g),
\]
где $\mu=0{,}1$. Тогда
\[
T=\frac{m_1m_2(a+g)(1+\mu)}{m_1+m_2}=4{,}41\;\text{Н}.
\]