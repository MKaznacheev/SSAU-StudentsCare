\subsection{15}

Сила
\[
F=F_0\cos\omega t
\]
($F_0=\text{const}$, $\omega=\text{const}$) приводит в движение тело массой $m$. При этом известно, что
\[
r(0)=0\;\text{м},\quad v(0)=0\;\text{м/с}.
\]
Поскольку
\[
F=ma\LR F_0\cos\omega t=m\frac{dv}{dt}\LR \frac{F_0}{m}\cos\omega t\,dt=\,dv,
\]
то
\[
\frac{F_0}{m}\int\cos\omega t\,dt=\int dv\LR \frac{F_0}{m\omega}\sin\omega t=v+v(0)\LR \frac{F_0}{m\omega}\sin\omega t=v.
\]
С учетом того, что
\[
v=\frac{dr}{dt},
\]
имеем
\[
\frac{F_0}{m\omega}\sin\omega t=\frac{dr}{dt}\LR \frac{F_0}{m\omega}\sin\omega t\,dt=\,dr.
\]
Тогда
\[
\frac{F_0}{m\omega}\int\sin\omega t\,dt=\int dr \LR -\frac{F_0}{m\omega^2}\cos\omega t=r+r(0) \LR r=-\frac{F_0}{m\omega^2}\cos\omega t.
\]