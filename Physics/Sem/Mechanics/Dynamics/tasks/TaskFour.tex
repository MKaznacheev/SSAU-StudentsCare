\subsection{12}

Угол, который составляет наклонная плоскость с горизонталью, равен $\alpha=20^\circ$. Масса груза, который на ней лежит, равна $m_1=0{,}2\;\text{кг}$, а масса груза, связанного с ним --- $m_2=0{,}15\;\text{кг}$. Уравнения движения тел принимают вид
\[
\left\{
\begin{NiceArray}{l}
	m_1\vec g+\vec N+\vec T_1=m_1\vec a_1 \\
	m_2\vec g+\vec T_2=m_2\vec a_2
\end{NiceArray}
\right.\;,
\]
где $\vec T_1$ и $\vec T_2$ --- силы натяжения нити, приложенные к телам массы $m_1$ и $m_2$ соответственно, а $\vec a_1$ и $\vec a_2$ --- их ускорения. При этом
\[
T_1=T_2=T
\]
и
\[
a_1=a_2=a.
\]
Направим ось $Oy$ по направлению движения груза массой $m_1$, а ось $Ox$ --- вертикально вниз. Запишем уравнения движения в проекциях на эти оси (первое --- на ось $Oy$, а второе --- на ось $Ox$):
\[
\left\{
\begin{NiceArray}{l}
	-m_1g\sin\alpha+T=m_1a\\
	m_2g-T=m_2a
\end{NiceArray}
\right.\;\LR
\left\{
\begin{NiceArray}{l}
	T=m_1a+m_1g\sin\alpha \\
	T=m_2g-m_2a
\end{NiceArray}
\right.\,.
\]
Приравняем полученные выражения:
\[
m_1a+m_1g\sin\alpha=m_2g-m_2a\LR a=\frac{(m_2-m_1\sin\alpha)g}{m_1+m_2}\approx2{,}28\;\text{м/с}^2.
\]