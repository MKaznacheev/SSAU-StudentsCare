\section{Теоретические сведения}

Чувствительность весов $\delta$ определяется, как
\[
\delta=\frac{\Delta\alpha}{\Delta P},
\]
где $\Delta\alpha$ --- угол отклонения стрелки, а $\Delta P$ --- величина перегрузка. Обозначим через $L$ длину плеча кооамысла, через $l$ --- расстояние от его точки опоры до центра тяжести, а через $G$ --- его вес. Расчёты показывают, что
\[
\delta=\frac{L}{Gl}
\]
при $\alpha=0$, то есть чувствительность весов не зависит ни от нагрузки $P$, ни от величины перегрузка $\Delta P$.

В действительности, под действием нагрузки коромысло деформируется, отчего меняется чувствительность весов. Для уменьшения влияния этого эффекта на результаты измерений, применяются методы точного взвешивания. Мы будем использовать метод Менделеева. Положим на левую чашку весов гирю предельной массы. Уравновесим весы разновесками, положив их на правую чашку. Затем на неё положим взвешиваемое тело и будем убирать разновески, пока не восстановится равновесие. Масса снятых разновесков окажется равной массе тела.

Трение опорных призм о подушки приводит к тому, что при неизменной нагрузке весов, стрелка может останавливаться напротив разных делений шкалы. Так появляется полоса застоя. Нулевой точкой весов называют положение равновесия коромысла при отсутствии нагрузки.

Величину $\Delta$ определим следующим образом:
\[
\delta=\frac{|n-n_0|}{10},
\]
где $n_0$ --- отсчёт по шкале, соответствующий начальному положению равновесия весов, $n$ --- отсчёт при нагрузке в $10\;\text{мг}$. Цена деления при этом окажется равна
\[
h=\delta^{-1}.
\]

При взвешивании на левую чашу будем помещать гирю массой $100\;\text{г}$. Необходимо вновь определить нулевую точку $n_0'$, теперь --- при полной нагрузке. Через $m$ обозначим массу взвешиваемого тела, через $m_1$ --- массу разновеса, находящегося на правой чаше, через $\delta'$ --- чувствительность полностью нагруженных весов. Тогда
\[
m=100-m_1-\frac{n_1-n_0'}{\delta'}, \quad [m]=\text{г}.
\]

Из-за действия на тела выталкивающей силы воздуха, равновесие весов наступает при равенстве разностей веса и архимедовой силы для тела и разновеса. Таким образом,
\[
m_0=m+m\cdot0{,}0012\left(\frac{1}{\rho}-0{,}12\right),
\]
где $m$ --- масса тела, полученная по измерениям в воздухе, $m_0$ --- истинная масса тела. При этом, учтено, что плотность воздуха составляет $0{,}0012\;\text{г/см}^3$, а плотность латуни (вещества разновесков) --- $8{,}4\;\text{г/см}^3$.