\section{Контрольные вопросы}

\begin{Enumerate}
	\item Инертной массой тела называют меру его инертности, то есть сопротивления при попытках привести его в движение или изменить модуль или направление его скорости (Сивухин Д.В. Общий курс физики. Т.1: Механика). Так же выделяют гравитационную массу --- меру гравитационного взаимодействия тел. Весом тела называют силу, с которой оно действует на опору или подвес.
	
	\item Взвешивание тел на обыкновенных весах даёт результат, который в тех или иных случаях не может считаться удовлетворительным. Более точное взвешивание требует исключения ошибок разного рода. Для этого применяются методы Гаусса, Борда и Менделеева.
	
	\item При перемещении с полюса на экватор результат взвешивания на рычажных весах изменится, поскольку они прямо измеряют массу тел. Это будет связано с изменением рельефа Земли и влиянием сил инерции. Пружинные весы измеряют вес тел, а не их массу. Их измерения от перемещения не зависят, поскольку жёсткость пружины остаётся постоянной.
	
	\item Полоса застоя --- это промежуток на шкале, в котором может остановится стрелка весов при их равновесии. Её появление вызвано трением призмы об агатовую подушку.
	
	\item Специальной методикой при точном взвешивании следует пользоваться для минимизации влияния прогиба коромысла и, как следствие, изменения чувствительности весов при увеличении нагрузки.
	
	\item В методе постоянной нагрузки исключаются ошибки, связанные с неодинаковостью плеч коромысла и изменением чувствительности весов.
\end{Enumerate}