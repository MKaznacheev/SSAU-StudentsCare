\section{Вывод}

В упражнении по измерению толщины пластинки систематическая погрешность измерения толщины настолько мала, благодаря малой приборной погрешности микрометра, что соответствующая полная погрешность в некотором приближении определяется только случайной. При соблюдении методики измерений, удаётся добиться значения относительной погрешности измерения толщины микрометром меньше процента, что несомненно можно считать хорошим результатом.

В упражнении по определению объёма трубки и плотности её материала, так же получены удовлетворяющие результаты: относительные погрешности измерения длины трубки, внутреннего и внешнего диаметров не превышают одного процента, а относительные погрешности вычисления объёма и плотности --- двух. При этом систематическая ошибка, вносимая штангенциркулем в десять раз превышает таковую для микрометра. Последнее значительно влияет на точность измерений и вычислений: абсолютные полные погрешности величин $L$, $D$, $d$ превышают погрешность величины $l$ в первом упражнении не менее, чем в два раза. Однако в рамках упражнения такая точность вполне допустима. Полученное значение плотности с хорошей точностью соответствует плотности железа, но не исключено, что материалом трубки является один из видов стали.