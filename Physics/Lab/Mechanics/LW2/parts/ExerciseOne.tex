\section{Результаты измерений и обработка данных}

\subsection{Измерение толщины металлической пластинки микрометром}

Измерим толщину пластинки во всех её углах по три раза. Занесём данные в таблицу~\ref{TabOne}. При этом, $l_{ik}$ --- измеренная толщина, где $i$ --- номер угла, а $k$ --- номер измерения. Среднее значение вычисляется, как
\[
\bar l=\frac{\displaystyle\sum_{i=1}^{4}\displaystyle\sum_{k=1}^{3}l_{ik}}{4\cdot3}=\frac{\displaystyle\sum_{i=1}^{4}\displaystyle\sum_{k=1}^{3}l_{ik}}{12},
\]
а расхождение $ik$-о измерения со среднем --- как
\[
\Delta l_{ik}=l_{ik}-\bar l.
\]

Доверительную вероятность примем равной $95\%$:
\[
P=0{,}95.
\]
Коэффициент Стьюдента при количестве измерений $n=12$ равен
\[
t_P(n)=2{,}2.
\]

Приборная погрешность микрометра равна половине цены его деления:
\[
\Delta l_\text{пр}=\frac{0{,}01}{2}=0{,}005\;\text{мм}.
\]
Его погрешность округления в таком случае составит
\[
\Delta l_\text{окр}=P\cdot\Delta l_\text{пр}\approx0{,}005\;\text{мм}.
\]
Следовательно, систематическая погрешность будет равна
\[
\Delta l_\text{сист}=\sqrt{(\Delta l_\text{пр})^2+(\Delta l_\text{окр})^2}\approx0{,}007\;\text{мм}.
\]
Случайная ошибка вычисляется по формуле
\[
\Delta l_\text{случ}=t_P(n)\cdot S\approx0{,}04\;\text{мм},
\]
где
\[
S=\sqrt{\frac{\displaystyle\sum_{i=1}^{4}\displaystyle\sum_{k=1}^{3}(\Delta l_{ik})^2}{n(n-1)}}
\]
--- средняя квадратичная погрешность. Полная абсолютная ошибка измерения толщины пластинки равна
\[
\Delta l=\sqrt{(\Delta l_\text{случ})^2+(\Delta l_\text{сист})^2}\approx0{,}04.
\]
Наконец, относительная погрешность вычисляется по формуле
\[
\varepsilon_l=\frac{\Delta l}{\bar l}\cdot100\%\approx0{,}49\%.
\]

\begin{table}[h]
	\begin{center}
		\begin{tabular}{|c|c|c|c|c|c|c|c|c|}
			\hline
			\parbox[c][3em]{3em}{Номер угла} & \parbox[c][3em]{3em}{Номер опыта} & \parbox[c][3em]{2em}{$l_{ik}$, мм} & \parbox[c][3em]{2.5em}{$\Delta l_{ik}$, мм} & \parbox[c][3em]{3.5em}{$(\Delta l_{ik})^2$, мм$^2$} & \parbox[c][3em]{2em}{$S$, мм} & \parbox[c][3em]{2em}{$\Delta l$, мм} & \parbox[c][3em]{2em}{$\varepsilon_l$, \%} & \parbox[c][3em]{3em}{$l\pm\Delta l$, мм} \\
			\hline
			\multirow{3}{*}{I} & 1 & 8{,}09& $-0{,}09$ & 0{,}0081 & \multirow{14}{*}{0{,}02} & \multirow{14}{*}{0{,}04} & \multirow{14}{*}{0{,}49} & \multirow{14}{*}{$8{,}18\pm0{,}04$} \\
			\cline{2-5}
			& 2 & 8{,}10& $-0{,}08$ & 0{,}0064 & & & & \\
			\cline{2-5}
			& 3 & 8{,}10 & $-0{,}08$ & 0{,}0064 & & & & \\
			\cline{1-5}
			\multirow{3}{*}{II} & 1 & 8{,}10 & $-0{,}08$ & 0{,}0064 & & & & \\
			\cline{2-5}
			& 2 & 8{,}10 & $-0{,}08$ & 0{,}0064 & & & & \\
			\cline{2-5}
			& 3 & 8{,}10 & $-0{,}08$ & 0{,}0064 & & & & \\
			\cline{1-5}
			\multirow{3}{*}{III} & 1 & 8{,}24 & 0{,}06 & 0{,}0036 & & & & \\
			\cline{2-5}
			& 2 & 8{,}25 & 0{,}07 & 0{,}0049 & & & & \\
			\cline{2-5}
			& 3 & 8{,}24 & 0{,}06  & 0{,}0036 & & & & \\
			\cline{1-5}
			\multirow{3}{*}{IV} & 1 & 8{,}27 & 0{,}09 & 0{,}0081 & & & & \\
			\cline{2-5}
			& 2 & 8{,}26 & 0{,}08 & 0{,}0064 & & & & \\
			\cline{2-5}
			& 3 & 8{,}26 & 0{,}08 & 0{,}0064 & & & & \\
			\cline{1-5}
			& & \parbox[c][3em]{2em}{$\bar l$, мм} & & \parbox[c][5em]{7em}{$\displaystyle\sum_{i=1}^{4}\displaystyle\sum_{k=1}^{3}(\Delta l_{ik})^2$, мм$^2$} & & & & \\
			\cline{3-3}\cline{5-5}
			& & 8{,}18 & & 0{,}0731 & & & & \\
			\hline
		\end{tabular}
		\caption{Измерение толщины пластинки}\label{TabOne}
	\end{center}
\end{table}