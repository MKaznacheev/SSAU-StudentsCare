\section{Контрольные вопросы}

\begin{Enumerate}
	\item Нониусом называют дополнение к основному (обычному) масштабу, повышающее точность измерений. Линейный нониус представляет собой дополнительную линейку, скользящую вдоль основной. При этом дополнительный масштаб содержит $m$ делений, последнее из которых совпадает с $n$-м делением основного.
	
	\item Заметим, что для любого линейного нониуса истинно соотношение
	\[
	mx=ny\LR x=\frac{ny}{m},
	\]
	где $x$ --- цена деления дополнительного масштаба, а $y$ --- основного. Буквами $m$ и $n$ обозначены совпадающие деления дополнительного и основного масштабов соответсвенно. При этом число $m$ совпадает с полным количеством делений нониуса. Точность штангенциркуля вычисляется по формуле
	\[
	\Delta x=\frac{y}{m}.
	\]
	Погрешность штангенциркуля определяется, как
	\[
	\Delta l_\text{пр}=\frac{\Delta x}{2}.
	\]
	
	В нашем случае, $m=10$ и $n=19$, $y=1\;\text{мм}$. Тогда
	\begin{gather*}
	x=1{,}9\;\text{мм}, \\
	\Delta x=0{,}1\;\text{мм}, \\
	\Delta l_\text{пр}=0{,}05\;\text{мм}.
	\end{gather*}
	
	\item В нашей работе шаг микрометрического винта (смещение барабана относительно основной шкалы при полном его обороте) равен $0{,}5\;\text{мм}$. Цена деления деления основной шкалы микрометра совпадает с шагом его винта, а цена деления дополнительной --- равна шагу винта, делёному на количество делений дополнительной шкалы. В нашем случае, цена деления дополнительной шкалы составляет $0{,}01\;\text{мм}$. Погрешность равна половине цены этого деления: $0{,}005\;\text{мм}$.
	
	\item Прямыми называются измерения длины, произведённые некоторым прибором непосредственно. Косвенными будут считаться измерения длины, когда она является функцией других величин, измеренных прямо или так же косвенно. Например, измерив острый угол $\alpha $ прямоугольного треугольника и длину его гипотенузы $c$, можно найти длины его катетов $a$ и $b$ по формулам
	\[
	a=c\sin\alpha,\quad b=c\cos\alpha.
	\]
	При этом, $a=a(c,\alpha)$ и $b=b(c,\alpha)$ --- косвенно измеренные величины.
		
	В данной работе прямыми являются измерения всех длин: толщины пластинки в первом упражнении, а так же внутреннего и внешнего диаметров и длины трубки во втором. Во втором упражнении косвенно измерены объём и плотность трубки. В первом подобных измерений не проводилось.
	
	\item Схема обработки прямых измерений приведена третьем разделе отчёта. Ошибки косвенного измерения величины $u=f(x,y,x,\ldots)$ находятся по формулам
	\[
	\Delta u=\sqrt{\left(\frac{\partial f}{\partial x}\Delta x\right)^2+\left(\frac{\partial f}{\partial y}\Delta y\right)^2+\left(\frac{\partial f}{\partial z}\Delta z\right)^2+\ldots}
	\]
	и
	\[
	\varepsilon_u=\frac{\Delta u}{u}\cdot100\%.
	\]
\end{Enumerate}