\section{Результаты измерений и обработка данных}

\subsection{Проверка формулы для пути, пройденного телом при прямолинейном равноускоренном движении}

После проведения ряда подготовительных действий, перейдём к измерениям. В таблице~\ref{TbOne} приведены массы перегрузков ($m_1'$, $m_2'$, $m_3'$), а так же левого и правого грузов ($m_b$ и $m_a$ соответсвенно). Доверительную вероятность будем считать равной~$95\%$:
\[
P=0{,}95.
\]

\begin{table}[h!]
	\begin{center}
	\begin{tabular}{|c|c|}
		\hline
		Величина & Значение \\
		\hline
		\hline
		$m_a$ & $80{,}720\pm0{,}005\;\text{г}$ \\
		\hline
		$m_b$ & $80{,}950\pm0{,}005\;\text{г}$ \\
		\hline
		$m_1'$ & $2{,}56\pm0{,}01\;\text{г}$ \\
		\hline
		$m_2'$ & $4{,}00\pm0{,}01\;\text{г}$ \\
		\hline
		$m_3'$ & $4{,}12\pm0{,}01\;\text{г}$ \\
		\hline
	\end{tabular}
	\caption{Массы грузов и перегрузков}\label{TbOne}
	\end{center}
\end{table}

На левый груз положим перегрузок массой $m_1'$, а на правый --- перегрузки массами $m_2'$ и $m_3'$. Расстояние от нижнего среза правого
груза до индекса среднего кронштейна составляет
\[
h_1=0{,}3500\pm0{,}0007\;\text{м},
\]
учитывая, что приборная погрешность линейки равна половине цены деления:
\[
\Delta h_\text{пр}=\frac{0{,}001}{2}=0{,}0005\;\text{м},
\]
а так же
\[
\Delta h_\text{случ}=0\;\text{м},\quad\Delta h_\text{окр}=P\cdot\Delta h_\text{пр}\approx0{,}0005\;\text{м}.
\]
и, как следствие,
\[
\Delta h=\sqrt{\Delta h_\text{случ}^2+\Delta h_\text{пр}^2+\Delta h_\text{окр}^2}\approx0{,}0007\;\text{м}.
\]

После сброса груза, он преодолел расстояние $h_1$ за
\[
t_{11}=1{,}590\pm0{,}001\;\text{с},
\]
с учётом погрешностей электронного миллисекундомера:
\begin{gather*}
\Delta t_\text{случ}=0\;\text{с},\quad\Delta t_\text{пр}=0{,}0010\;\text{с},\quad\Delta t_\text{пр}=P\cdot\frac{\Delta t_\text{пр}}{2}\approx0{,}0005\;\text{с}, \\
\Delta t=\sqrt{\Delta t_\text{случ}^2+\Delta t_\text{пр}^2+\Delta t_\text{окр}^2}\approx0{,}001\;\text{с}
\end{gather*}
(для одного измерения полная погрешность совпадает с систематической). Повторим этот опыт ещё два раза. При этом величина $\bar t_k$, где $k$ --- номер серии измерений вычисляется по формуле
\[
\bar t_k=\frac{\displaystyle\sum_{i=1}^nt_{ki}}{3},
\]
а $\Delta t_{ki}$ --- по формуле
\[
\Delta t_{ki}=\left|\bar t_k- t_{ki}\right|
\]
(эта величина не является погрешностью, а лишь расхождением $i$-го измерения времени со средним значением для $k$-й серии).

Затем, дважды поменяем положение фотодатчика и проведём аналогичные серии измерений для каждого из положений. Запишем результаты в таблицу~\ref{TbTwo}.

\begin{table}
	\begin{center}
	\begin{tabular}{|c|c|c|c|c|c|c|c|c|c|c|c|c|c|}
		\hline
		$k$ & \makecell{$h_k$,\\м} & \makecell{$\varepsilon_{h_k}$,\\\%} & $i$ & \makecell{$t_{ki}$,\\с} & \makecell{$\bar t_k$,\\с} & \makecell{$\Delta t_{ki}$,\\с} & \makecell{$S_{\bar t}$,\\с} & \makecell{$\Delta t_k$,\\с} & \makecell{$\varepsilon_{t_k}$,\\\%} & \makecell{$a_k$,\\м/с$^2$} & \makecell{$\varepsilon_{a_k}$,\\\%} & \makecell{$\Delta a_k$,\\м/с$^2$} \\
		\hline
		\hline
		\multirow{3}{*}{1} & \multirow{3}{*}{0{,}35} & \multirow{3}{*}{0{,}20} & 1 & 1{,}590 & \multirow{3}{*}{1{,}622} & 0{,}032 & \multirow{3}{*}{0{,}029} & \multirow{3}{*}{0{,}125} & \multirow{3}{*}{7{,}71} & \multirow{3}{*}{0{,}266} & \multirow{3}{*}{15{,}42} & \multirow{3}{*}{0{,}041}\\
		\cline{4-5}\cline{7-7}
		& & & 2 & 1{,}597 & & 0{,}025 & & & & & & \\
		\cline{4-5}\cline{7-7}
		& & & 3 & 1{,}679 & & 0{,}057 & & & & & & \\
		\hline
		\multirow{3}{*}{2} & \multirow{3}{*}{0{,}40} & \multirow{3}{*}{0{,}18} & 1 & 1{,}772 & \multirow{3}{*}{1{,}811} & 0{,}039 & \multirow{3}{*}{0{,}023} & \multirow{3}{*}{0{,}099} & \multirow{3}{*}{5{,}47} & \multirow{3}{*}{0{,}244} & \multirow{3}{*}{10{,}94} & \multirow{3}{*}{0{,}027} \\
		\cline{4-5}\cline{7-7}
		& & & 2 & 1{,}811 & & 0 & & & & & & \\
		\cline{4-5}\cline{7-7}
		& & & 3 & 1{,}850 & & 0{,}039 & & & & & & \\
		\hline
		\multirow{3}{*}{3} & \multirow{3}{*}{0{,}45} & \multirow{3}{*}{0{,}15} & 1 & 1{,}865 & \multirow{3}{*}{1{,}842} & 0{,}023 & \multirow{3}{*}{0{,}012} & \multirow{3}{*}{0{,}052} & \multirow{3}{*}{2{,}82} & \multirow{3}{*}{0{,}265} & \multirow{3}{*}{5{,}64} & \multirow{3}{*}{0{,}015} \\
		\cline{4-5}\cline{7-7}
		& & & 2 & 1{,}824 & & 0{,}018 & & & & & & \\
		\cline{4-5}\cline{7-7}
		& & & 3 & 1{,}836 & & 0{,}006 & & & & & & \\
		\hline
	\end{tabular}
	\caption{Результаты первого опыта}\label{TbTwo}
	\end{center}
\end{table}

Для каждой серии измерений относительная погрешность вычислений высоты падения груза составляет
\[
\varepsilon_{h_k}=\frac{\Delta h}{h_k}=\frac{0{,}0007}{h_k}\cdot100\%.
\]
Средняя ошибка для времени составляет
\[
S_{\bar t}=\sqrt{\frac{\displaystyle\sum_{i=1}^3\Delta t_{ki}^2}{3(3-1)}}=\sqrt{\frac{\displaystyle\sum_{i=1}^3\Delta t_{ki}^2}{6}}.
\]
В таком случае, полная погрешность трёх измерений составляет
\[
\Delta t_k=\sqrt{\bigl(t_P(n)\cdot S_{\bar t}\bigr)^2+\Delta t^2},
\]
где $t_P(n)$ --- коэффициент Стьюдента:
\[
t_P(n)=4{,}30,
\]
при количестве измерений $n=3$. Полная абсолютная погрешность для времени вычисляется по формуле
\[
\varepsilon_{t_k}=\frac{\Delta t_k}{\bar t_k}\cdot100\%.
\]

Ускорение выражается через уже известные нам величины:
\[
a_k=\frac{2h_k}{\bar t_k^2},
\]
а относительная и абсолютная ошибки высчитываются по формулам
\[
\varepsilon_{a_k}=\sqrt{\varepsilon_{h_k}^2+4\varepsilon_{t_k}^2},\quad\Delta a_k= \frac{\varepsilon_{a_k}}{100}\cdot a_k.
\]

Повторим всё вышеописанное, но теперь на левый груз не станем класть перегрузки, а на правый --- пложим все у нас имеющиеся. Результаты занесём в таблицу~\ref{TbThree}.

\begin{table}
	\begin{center}
	\begin{tabular}{|c|c|c|c|c|c|c|c|c|c|c|c|c|c|}
		\hline
		$k$ & \makecell{$h_k$,\\м} & \makecell{$\varepsilon_{h_k}$,\\\%} & $i$ & \makecell{$t_{ki}$,\\с} & \makecell{$\bar t_k$,\\с} & \makecell{$\Delta t_{ki}$,\\с} & \makecell{$S_{\bar t}$,\\с} & \makecell{$\Delta t_k$,\\с} & \makecell{$\varepsilon_{t_k}$,\\\%} & \makecell{$a'_k$,\\м/с$^2$} & \makecell{$\varepsilon_{a'_k}$,\\\%} & \makecell{$\Delta a'_k$,\\м/с$^2$} \\
		\hline
		\hline
		\multirow{3}{*}{1} & \multirow{3}{*}{0{,}35} & \multirow{3}{*}{0{,}20} & 1 & 1{,}157 & \multirow{3}{*}{1{,}172} & 0{,}015 & \multirow{3}{*}{0{,}008} & \multirow{3}{*}{0{,}034} & \multirow{3}{*}{2{,}90} & \multirow{3}{*}{0{,}510} & \multirow{3}{*}{5{,}80} & \multirow{3}{*}{0{,}030} \\
		\cline{4-5}\cline{7-7}
		& & & 2 & 1{,}176 & & 0{,}004 & & & & & & \\
		\cline{4-5}\cline{7-7}
		& & & 3 & 1{,}183 & & 0{,}011 & & & & & & \\
		\hline
		\multirow{3}{*}{2} & \multirow{3}{*}{0{,}40} & \multirow{3}{*}{0{,}18} & 1 & 1{,}284 & \multirow{3}{*}{1{,}270} & 0{,}014 & \multirow{3}{*}{0{,}007} & \multirow{3}{*}{0{,}030} & \multirow{3}{*}{2{,}36} & \multirow{3}{*}{0{,}496} & \multirow{3}{*}{4{,}72} & \multirow{3}{*}{0{,}023} \\
		\cline{4-5}\cline{7-7}
		& & & 2 & 1{,}266 & & 0{,}004 & & & & & & \\
		\cline{4-5}\cline{7-7}
		& & & 3 & 1{,}259 & & 0{,}011 & & & & & & \\
		\hline
		\multirow{3}{*}{3} & \multirow{3}{*}{0{,}45} & \multirow{3}{*}{0{,}15} & 1 & 1{,}321 & \multirow{3}{*}{1{,}340} & 0{,}019 & \multirow{3}{*}{0{,}010} & \multirow{3}{*}{0{,}043} & \multirow{3}{*}{3{,}21} & \multirow{3}{*}{0{,}501} & \multirow{3}{*}{6{,}42} & \multirow{3}{*}{0{,}032} \\
		\cline{4-5}\cline{7-7}
		& & & 2 & 1{,}357 & & 0{,}017 & & & & & & \\
		\cline{4-5}\cline{7-7}
		& & & 3 & 1{,}341 & & 0{,}001 & & & & & & \\
		\hline
	\end{tabular}
	\caption{Результаты второго опыта}\label{TbThree}
	\end{center}
\end{table}

Интервалы $(a_k-\Delta a_k;a_k+\Delta a_k)$ для $k=1$, $k=2$, $k=3$ соотвественно имеют следующий вид:
\[
(0{,}225;0{,}307),\quad(0{,}217;0{,}271),\quad(0{,}250;0{,}280),
\]
а $\left(a'_k-\Delta a'_k;a'_k+\Delta a'_k\right)$ ---
\[
(0{,}480;0{,}540),\quad(0{,}473;0{,}519),\quad(0{,}469;0{,}533).
\]
Как в первом, так и во втором случае общие точки имеют все три интервала.