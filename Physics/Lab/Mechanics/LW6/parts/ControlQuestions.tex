\section{Контрольные вопросы}

\begin{Enumerate}
	\item Оба груза в машине Атвуда будут двигаться с равными по модулю ускорениями, если нить, на которой они подвешены, является нерастяжимой.
	
	\item При выводе системы уравнений~\eqref{EqFifteen} мы условились считать силу сопротивления воздуха пренебрежимо малой, нить --- нерастяжимой и невесомой,  а проскальзывание нити по блоку --- исключённым.
	
	\item Выведем пошагово уравнение~\eqref{EqSeventeen} из системы~\eqref{EqFifteen} и уравнения~\eqref{EqSixteen}:
	\begin{multline*}
	\left\{
	\begin{NiceArray}{l}
		\eqref{EqFifteen} \\
		\eqref{EqSixteen}
	\end{NiceArray}
	\right.
	\Rightarrow
	\left\{
	\begin{NiceArray}{l}
		T_1=m_1(g+a) \\
		T_2=m_2(g-a) \\
		\alpha m_0r^2\cdot\dfrac{a}{r}=r(T_2-T_1)-M_\text{тр}
	\end{NiceArray}
	\right.
	\Rightarrow \\
	\Rightarrow
	a\alpha m_0r=r\bigl(m_2(g-a)-m_1(g+a)\bigr)-M_\text{тр}\LR \\
	\LR a(m_1+m_2+\alpha m_0)r=(m_2-m_1)gr-M_\text{тр}\LR \\
	a=\cfrac{(m_2-m_1)g-\cfrac{M_\text{тр}}{r}}{m_1+m_2+\alpha m_0}\LR\eqref{EqSeventeen}.
	\end{multline*}
	Так, в первую очередь в третьем уравнении системы~\eqref{EqFifteen} избавляются от величины~$\beta$, а в первых двух выражают силы натяжения нитей. Результат последних действий подставляют в третье уравнение, после чего все члены, содержащие ускорение~$a$, переносят в одну сторону. Последним действием выражают величину $a$.
	Формулы для сил натяжения нитей принимают следующий вид:
	\begin{gather*}
		T_1=m_1\left(g+\cfrac{(m_2-m_1)g-\cfrac{M_\text{тр}}{r}}{m_1+m_2+\alpha m_0}\right)=m_1\cdot\cfrac{(2m_2+\alpha m_0)g-\cfrac{M_\text{тр}}{r}}{m_1+m_2+\alpha m_0}, \\
		T_2=m_2\left(g-\cfrac{(m_2-m_1)g-\cfrac{M_\text{тр}}{r}}{m_1+m_2+\alpha m_0}\right)=m_2\cdot\cfrac{(2m_1+\alpha m_0)g+\cfrac{M_\text{тр}}{r}}{m_1+m_2+\alpha m_0}.
	\end{gather*}
	
	\item Массой блока можно пренебречь в том случае, если она существенно меньше суммы масс грузов. Если момент сил трения значительно меньше величины
	\[
	(m_2-m_1)rg,
	\]
	где $m_1$ и $m_2$ --- массы левого и правого груза соотвественно, то можно пренебречь и трением в оси блока.
	
	\item Различием сил натяжения нитей, действующих на грузы, можно пренебречь, если в условиях эксперимента нить считается невесомой.
	
	\item Уравнения движения грузов в машине Атвуда выглядят следующим образом:
	\[
	\left\{
	\begin{NiceArray}{l}
		m_1\vec a_1=m_1\vec g+\vec T_1 \\
		m_2\vec a_2=m_2\vec g+\vec T_2
	\end{NiceArray}
	\right.\;.
	\]
	Проецируя последнее на ось $Ox$, получим
	\[
	\left\{
	\begin{NiceArray}{l}
		m_1a_{1x}=m_1g-T_1 \\
		m_2a_{2x}=m_2g-T_2
	\end{NiceArray}
	\right.\;.
	\]
	Так как нить не растяжима, то $a=-a_{1x}=a_{2x}$, а значит,
	\[
	\left\{
	\begin{NiceArray}{l}
		-m_1a=m_1g-T_1 \\
		m_2a=m_2g-T_2
	\end{NiceArray}
	\right.\;.
	\]
	Далее, так же, как описано в теоретических сведениях, прийдём к системе уравнений~\eqref{EqFifteen}, а затем, повторяя третий контрольный вопрос --- к уравнению \eqref{EqSeventeen}. Ввиду невесомости блока и отсутствием трения в его оси (см. контрольный вопрос 4, а так же уравнения~$\eqref{EqEighteen}\div\eqref{EqNeinteen}$), заключим, что
	\[
	a\approx\frac{(m_2-m_1)g}{m_1+m_2}.
	\]
	
	Модули сил натяжения нитей также вычисляются аналогично контрольному вопросу 3 и, благодаря тем же допущениям, принимают вид
	\begin{gather*}
		T_1=\cfrac{2m_1m_2}{m_1+m_2}\cdot g, \\
		T_2=\cfrac{2m_1m_2}{m_1+m_2}\cdot g,
	\end{gather*}
	поскольку
	\[
	\left\{
	\begin{NiceArray}{l}
		2m_2\approx m_1+m_2\gg\alpha m_0\\
		2m_1\approx m_1+m_2\gg\alpha m_0\\
		2m_2rg\gg(m_2-m_1)rg\gg M_\text{тр}
	\end{NiceArray}
	\right.\;.
	\]
	
	\item Получим из формулы~\eqref{EqSix} и уравнения~\eqref{EqSeven} уравнение~\eqref{EqEight}. Обозначим угол между векторами $\vec N$ и $\vec r_2$ через $\alpha$, а между векторами $\vec F_\text{тр}$ и $\vec r_1$ --- через $\beta$. Запишем уравнение~\eqref{EqSeven} в проекциях на оси $Ox$ и $Oy$:
	\[
	\begin{array}{ll}
	Ox:\quad & m_0g+T'_1+T'_2-N\sin\alpha-F_\text{тр}\sin\beta=0, \\
	Oy:\quad & -N\cos\alpha+F_\text{тр}\cos\beta=0.
	\end{array}
	\]
	Заметим, что векторы $\vec F_\text{тр}$ и $\vec N$ ортогональны, а значит
	\[
	\alpha+\beta+\frac{\pi}{2}=\pi\LR\beta=\frac{\pi}{2}-\alpha.
	\]
	С учётом этого, а так же формулы~\eqref{EqSix}, преобразуем проекцию уравнения~\eqref{EqSeven} на ось ординат:
	\[
	\eqref{EqSeven}_{Oy}\LR-N\cos\alpha+\mu N\sin\alpha=0\LR \ctg\alpha=\mu\LR\alpha=\arcctg\mu.
	\]
	Теперь займёмся проекцией на ось абсцисс:
	\begin{multline*}
	\eqref{EqSeven}_{Ox}\LR m_0g+T'_1+T'_2-\frac{F_\text{тр}}{\mu}\sin\alpha-F_\text{тр}\cos\alpha=0\LR \\
	\LR F_\text{тр}\left(\frac{1}{\mu}\sin\alpha+\cos\alpha\right)=T'_1+T'_2+m_0g\LR \\
	\LR F_\text{тр}\left(\frac{1}{\mu}\sin\arcctg\mu+\cos\arcctg\mu\right)=T'_1+T'_2+m_0g\LR \\
	\LR F_\text{тр}\left(\frac{1}{\mu\sqrt{1+\mu^2}}+\frac{\mu}{\sqrt{1+\mu^2}}\right)=T'_1+T'_2+m_0g\LR \\
	\LR F_\text{тр}=\frac{\mu}{\sqrt{1+\mu^2}}\left(T'_1+T'_2+m_0g\right)\approx\mu\left(T'_1+T'_2+m_0g\right)
	\end{multline*}
	(учтено, что $\mu\ll1$).
	
	\item Выведем формулу пути $h$, пройденного телом при свободном падении за время~$t$. Пусть $x$ --- координата положения тела по оси, сонаправленной движению:
	\[
	x=x(t).
	\]
	Нам известно, что
	\[
	\frac{dx}{dt}=v
	\]
	и
	\[
	\frac{d^2x}{dt^2}=\frac{dv}{dt}=a,
	\]
	где $v$ --- скорость. тела, $a$ --- ускорение тела. Выразим скорость через ускорение:
	\[
	dv=a\,dt\Rightarrow\int dv=\int a\,dt\Rightarrow v=at+v_0.
	\]
	Аналогично выразим координату через скорость:
	\[
	dx=v\,dt=(at+v_0)\,dt\Rightarrow\int dx=\int at\,dt+\int v_0\,dt\Rightarrow x=\frac{at^2}{2}+v_0t+x_0.
	\]
	Очевидно, что
	\[
	h=x-x_0=\frac{at^2}{2}+v_0t,
	\]
	а поскольку $v_0=0$, то
	\[
	h=\frac{at^2}{2}. 
	\]
	
	\item Второй закон Ньютона формулируется следующим образом: в инерциальной системе отсчёта производная импульса материальной точки по времени равна действующей на неё силе (Сивухин Д.В. Общий курс физики. Т.1: Механика). Так, уравнение движения материальной точки в соответсвии с этим законом принимает вид
	\[
	\frac{d\vec p}{dt}=\vec F.
	\]
	
	\item Описание методик содержится в разделе "Результаты измерений и обработка данных"\ и "Вывод"{}.
	
	\item В том случае, если трение в оси блока незначительно, однако мы всё равно не можем пренебречь его моментом инерции, то формула~\eqref{EqSeventeen} примет вид
	\[
	a=\frac{(m_2-m_1)g}{m_1+m_2+\alpha m_0}.
	\]
	Как следствие, формула~\eqref{EqTwentysix} будет справедлива. Вспоминим, что для вычисления величины  $a$ мы используем значения $m_1=m+m_1'$ и $m_2=m+m_2'$:
	\[
	a=\frac{(m_2'-m_1')g}{2m+m_1'+m_2'+\alpha m_0},
	\]
	а для вычисления $a'$ --- значения $m_1=m$, $m_2=m+m_1'+m_2'$:
	\[
	a'=\frac{(m_1'+m_2')g}{2m+m_1'+m_2'+\alpha m_0}.
	\]
	Тогда
	\[
	\frac{a}{a'}=\frac{(m_2'-m_1')g}{(m_1'+m_2')g}.
	\]
	
	\item Ответ на вопрос о том, почему результаты эксперимента можно рассматривать, как доказательство справедливости второго закона Ньютона, дан в выводе.
\end{Enumerate}