\subsection{Проверка второго закона Ньютона}

Вычислим средние экспериментальные значения ускорений грузов в машине Атвуда:
\[
a=\frac{a_1+a_2+a_3}{3}\approx0{,}258\;\text{м/с}^2,\quad a'=\frac{a_1'+a_2'+a_3'}{3}\approx0{,}502\;\text{м/с}^2.
\]
При этом
\[
\Delta a=\frac{1}{3}\cdot\sqrt{\Delta a_1^2+\Delta a_2^2+\Delta a_3^2}\approx0{,}017\;\text{м/с}^2,\quad\varepsilon_{a}=\frac{\Delta a}{a}\cdot100\%,\approx6{,}59\%,
\]
а также
\[
\Delta a'=\frac{1}{3}\cdot\sqrt{\Delta a_1^{'2}+\Delta a_2^{'2}+\Delta a_3^{'2}}\approx0{,}017\;\text{м/с}^2,\quad\varepsilon_{a'}=\frac{\Delta a'}{a'}\cdot100\%,\approx3{,}39\%.
\]

Отношение вычисленных значений ускорений составляет
\[
\frac{a}{a'}\approx0{,}514,
\]
а абсолютная ошибка ---
\[
\Delta\left(\frac{a}{a'}\right)=\frac{a}{a'}\cdot\frac{\sqrt{\varepsilon_a^2+\varepsilon_{a'}^2}}{100}\approx0{,}038.
\]

По формуле~\eqref{EqTwentysix} вычислим теоретическое значение отношения ускорений:
\[
\left(\frac{a}{a'}\right)_\text{теор}=\frac{m_2'+m_3'-m_1'}{m_1'+m_2'+m_3'}\approx0{,}521.
\]
При этом последняя величина попадает в интервал
\[
\biggl(\frac{a}{a'}-\Delta\left(\frac{a}{a'}\right);\frac{a}{a'}+\Delta\left(\frac{a}{a'}\right)\biggr)=(0{,}476;0{,}552).
\]