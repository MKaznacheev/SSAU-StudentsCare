\section{Схема установки}
На рисунке~\ref{PicTwo} приведена схема машины Атвуда. На основании 2 машины закреплен миллисекундомер 10 и стойка 1. На последней расположены три кронштейна: нижний 3, средний 4, верхний 5, на котором крепится блок с узлом подшипников, через который перекинута нить с одинаковыми грузами 6. Так же там расположен электромагнит 7, удерживающий систему в состоянии покоя.

На среднем кронштейне 4 укреплён фотоэлектрический датчик 8, который останавливает счёт времени при падении груза с достаточной высоты. Нижний кронштейн 3 является площадкой с резиновым амортизатором. На стойке укреплена металлическая линейка 9.