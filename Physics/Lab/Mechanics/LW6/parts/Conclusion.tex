\section{Вывод}

Полученные в первом упражнении результаты свидетельствуют о том, что движение грузов в машине Атвуда при любом расположении перегрузков с неравными массами является прямолинейным и равноускоренным, а путь, пройденный каждым грузом из состояния покоя за любое время $t$, вычисляется по формуле
\[
h=\frac{at^2}{2}.
\]
Об этом говорит совпадение вычисленных по этой формуле значений ускорений в пределах погрешностей с хорошей точностью.

Во втором упражнении мы замечаем совпадение теоретического и экспериментального значений величины 
\[
\frac{a}{a'}
\]
в пределах погрешности. Поскольку посылкой к формуле~\eqref{EqTwentysix}, является второй закон Ньютона, мы можем интерпретировать результаты этого упражнения, как одно из экспериментальных подтверждений справедливости второго закона Ньютона.