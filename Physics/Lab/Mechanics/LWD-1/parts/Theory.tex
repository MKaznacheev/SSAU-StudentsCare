\section{Теоретические сведения}

Пусть масса вращающейся платформы $D$ (см. рисунок~\ref{PicOne}) равна $m$ и при вращении она поднимается на высоту $h$. Тогда приращение потенциальной энергии определяется формулой
\[
U=mgh,
\]
где $g$ --- ускорение свободного падения. При прохождении положения равновесия во время вращения в другом направлении, кинетическая энергия платформы равна
\[
K=\frac{I\omega_0^2}{2},
\]
где $I$ --- момент инерции платформы, а $\omega_0$ --- её угловая скорость в этот момент. Исключая работу сил трения, запишем закон сохранения энергии:
\begin{equation}\label{eq:1}
K_i+U_i=\text{const}.
\end{equation}
При достижении высоты $h$, платформа имеет только потенциальную энергию, а в положении равновесия --- только кинетическую. Тогда
\begin{equation}\label{eq:2}
K=U\LR\frac{I\omega_0^2}{2}=mgh.
\end{equation}

Поскольку платформа совершает гармонические колебания, то имеет место запись
\[
\varphi=\varphi_0\sin\frac{2\pi}{T}t,
\]
где $\varphi$ --- угловое смещение платформы от положения равновесия, $\varphi_0$ --- амплитуда смещения, $T$ --- период колебаний, $t$ --- время. Следовательно,
\[
\omega=\frac{d\varphi}{dt}=\frac{2\pi}{T}\varphi_0\cos\frac{2\pi}{T}t.
\]
Отсюда видно, что в момент прохождения положения равновесия угловая скорость максимальна и равна
\[
\omega_0=\frac{2\pi}{T}\varphi_0.
\]
Тогда, на основании уравнения~\eqref{eq:2},
\begin{equation}\label{eq:3}
mgh=\frac{I}{2}\left(\frac{2\pi}{T}\varphi_0\right)^2
\end{equation}

Будем считать нити нерастяжимыми. В таком случае, нетрудно получить выражение
\[
h=\frac{Rr}{2l}\varphi_0^2
\]
(см. схему установки).
Скомпонируем полученное с уравнением~\eqref{eq:3}:
\begin{equation}\label{eq:4}
I=\frac{mgRr}{4\pi^2l}T^2.
\end{equation}
Этой формулой можно пользоваться и для определения момента инерции системы "платформа $+$ тело"{}.