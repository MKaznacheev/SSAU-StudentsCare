\section{Контрольные вопросы}

\begin{Enumerate}
	\item Моментом инерции твёрдого тела относительно оси называется величина
	\[
	I=\int r^2\,dm,
	\]
	где $r$ --- расстояние элемента массы $dm$ тела до оси вращения. Эта величина является мерой инертности тела во вращательном движении вокруг оси. Она измеряется в кг$\cdot$м$^2$.
	
	\item Проще всего найти момент инерции тела относительно оси его симметрии. Моменты относительно прочих осей можно найти по теореме Гюйгенса-Штейнера. Другой метод состоит в нахождении момента инерции тела относительно некоторого полюса $O$. Это используется, когда нужна найти моменты инерции относительно трёх взаимно перпендикулярных осей, пересекающихся в точке $O$. Найдём моменты инерции некоторых тел.
	\begin{enumerate}
		\item \textbf{Момент инерции однородного стержня длины $l$ и массы $m$.} Будем считать, что длина стержня значительно превосходит его поперечные размеры (иначе говоря, стержень является бесконечно тонким и представим в виде математической прямой). Его центр масс лежит в середине его длины. Ось $C$ направим через центр масс перпендикулярно стержню. Ввиду однородности стержня,
		\[
		\frac{dm}{dr}=\frac{m}{l}\LR dm=\frac{m}{l}\,dr,
		\]
		где $dr$ --- длина элемента стержня массы $dm$. Тогда
		\[
		I_C=\int r^2\,dm=\frac{m}{l}\int\limits_{-l/2}^{l/2}r^2dr=\left.\frac{mr^3}{3l}\right\vert_{-l/2}^{l/2}=\frac{ml^2}{12}.
		\]
		
		\item \textbf{Момент инерции однородного кольца радиуса $r$ и массы $m$.} Будем считать, что длина кольца значительно превосходит его толщину (иначе говоря, кольцо является бесконечно тонким и представимо в виде математической окружности). Его центр масс лежит в его геометрической  середине. Ось $C$ направим через центр масс перпендикулярно плоскости, в которой лежит кольцо. Заметим, что расстояния всех точек кольца до оси $C$ совпадают. Тогда
		\[
		I_C=\int r^2\,dm=r^2\int dm=mr^2.
		\]
		
		\item \textbf{Момент инерции однородного диска радиуса $R$ и массы $m$.} Будем считать, что диаметр диска значительно превосходит его толщину (иначе говоря, диск является бесконечно тонким и представим в виде математического круга). Его центр масс лежит в его геометрической  середине. Ось $C$ направим через центр масс перпендикулярно плоскости, в которой лежит диск. Ввиду однородности диска,
		\[
		\frac{dm}{dS}=\frac{m}{S}\LR dm=\frac{m}{S}\,dS,
		\]
		где $dS$ --- площадь элемента кольца массы $dm$. Заметим, что
		\[
		S=\pi R^2
		\]
		и
		\[
		dS=\pi dr^2=2\pi r\,dr.
		\]
		Тогда
		\[
		dm=\frac{2mr}{R^2}\,dr
		\]
		и
		\[
		I_C=\int r^2\,dm=\frac{2m}{R^2}\int\limits_{0}^{R}r^3\,dr=\frac{mR^2}{2}.
		\]
		
		\item \textbf{Момент инерции однородного цилиндра радиуса $R$ и массы $M$.} Момент инерции этой фигуры задаётся той же формулой, что и момент инерции плоского диска. Это можно объяснить следующим образом: цилиндр может быть представлен совокупностью бесконечного числа дисков радиусов $R$, моменты инерции которых относительно оси симметрии $C$ цилиндра, строго говоря, совпадают. Так
		\[
		I_C=\int d\frac{mR^2}{2}=\frac{R^2}{2}\int\limits_0^M dm=\frac{MR^2}{2}.
		\]
		
		Иначе, можно исходить только из однородности цилиндра:
		\[
		\frac{dm}{dV}=\frac{m}{V}\LR dm=\frac{m}{V}\,dV=\frac{m}{\pi R^2h}\,d\pi r^2h=\frac{2mr}{R^2}\,dr
		\]
		и
		\[
		I_C=\int r^2\,dm=\frac{2m}{R^2}\int\limits_0^R r^3\,dr=\frac{MR^2}{2}.
		\]
		
		\item \textbf{Момент инерции однородной втулки внутреннего радиуса $R$, внешнего радиуса $R_1$ и массы $M$.} В этом случае рассуждения практически аналогичны предшествующим:
		\[
		\frac{dm}{dV}=\frac{m}{V}\LR dm=\frac{m}{V}\,dV=\frac{m}{\pi (R_1^2-R^2)h}\,d\pi \left(r^2-R^2\right)h=\frac{2mr}{R_1^2-R^2}\,dr
		\]
		и
		\[
		I_C=\int r^2\,dm=\frac{2m}{R_1^2-R^2}\int\limits_R^{R_1} r^3\,dr=\frac{M\left(R_1^4-R^4\right)}{2\left(R_1^2-R^2\right)}=\frac{M\left(R_1^2+R^2\right)}{2}.
		\]
	\end{enumerate}
	
	\item Закон сохранения энергии выражается уравнениями~$\eqref{eq:1}\div\eqref{eq:2}$ и пояснён в теоретических сведениях.
	
	\item При выведении формулы мы полагаем, что можно исключить работу сил трения, а так же нити считать нерастяжимыми.
	
	\item В том случае, если ось вращения платформы не проходит через центр масс тела, методом трифилярного подвеса пользоваться нельзя, поскольку могут возникать колебания по нежелательным направлениям, что значительно усложнило бы вычисления, отчего метод потерял бы свою ценность.
	
	\item Получение формул для вычисления ошибок описано в третьем разделе "Результаты измерений и обработка данных"{}.
\end{Enumerate}