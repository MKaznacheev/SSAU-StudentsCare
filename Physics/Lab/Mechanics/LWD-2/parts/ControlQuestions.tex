\section{Контрольные вопросы}

\begin{Enumerate}
	\item Моментом инерции твёрдого тела относительно оси называется величина
	\[
	I=\int r^2\,dm,
	\]
	где $r$ --- расстояние элемента массы $dm$ тела до оси вращения. Эта величина является мерой инертности тела во вращательном движении вокруг оси. Она измеряется в кг$\cdot$м$^2$.
	
	\item Пусть имеется некоторое тело. Выберем в нём точку, расстояние от которой до оси $O$ определяется радиус-вектором $\vec r_O$ (при этом за точку $O$ будем считать проекцию выбранной точки тела на ось $O$), а до оси $A$ --- $\vec r_A$ (оси $O$ и $A$ параллельны). При этом расстояние между этими осями определяется радиус-вектором $\vec a$ так, что
	\[
	\vec r_A=\vec r_O-\vec a.
	\]
	Возведём это уравнение в квадрат и умножим на массу точки $dm$:
	\[
	r_A^2\,dm=r_O^2\,dm+a^2\,dm-2\left(\vec a,\vec r_O\,dm\right).
	\]
	Далее проинтегрируем результат:
	\[
	\int r_A^2\,dm=\int r_O^2\,dm+a^2\int dm-2\left(\vec a,\int\vec r_O\,dm\right).
	\]
	Заметим, что
	\[
	\int r_A^2\,dm=I_A,\quad \int r_O^2\,dm=I_O,\quad a^2\int dm=ma^2.
	\]
	Через $\vec R_C$ обозначим компоненту вектора центра масс относительно точки $O$, которая перпендикулярна оси $O$. Тогда
	\[
	\int\vec r_O\,dm=m\vec R_C
	\]
	и
	\[
	I_A=I_O+ma^2-2m\left(\vec a,\vec R_C\right).
	\]
	Если точка $O$ является центром масс $C$ тела, то $\vec R_C=\vec 0$ и
	\[
	I_A=I_C+ma^2.
	\]
	Это и есть теорема Гюйгенса-Штейнера.
	
	\item Закон сохранения энергии выражается уравнениями~$\eqref{eq:1}\div\eqref{eq:2}$ и пояснён в теоретических сведениях.
	
	\item При выведении формулы мы полагаем, что можно исключить работу сил трения, а так же нити считать нерастяжимыми.
	
	\item Получение формул для вычисления ошибок описано в третьем разделе "Результаты измерений и обработка данных"{}.
\end{Enumerate}