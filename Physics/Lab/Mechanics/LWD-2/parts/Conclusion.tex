\section{Вывод}

Эксперимент проведён с хорошей точностью: относительные ошибки измерений не превышают $2\%$. Вычисленное теоретически по теореме Гюйгенса-Штейнера значение момента инерции $I$ исследуемого тела практически лежит на отрезке $[I_T-\Delta I_T,I_T+\Delta I_T]$ (до попадания в интервал не хватает лишь $10^{-6}\;\text{кг}\cdot\text{м}^2$). Таким образом, теорему Гюйгенса-Штейнера можно считать экспериментально подтверждённой.