\section{Результаты измерений и обработка данных}

Определим время $N=10$ полных колебаний платформы, сообщая ей вращательный импульс. Измерения повторим 6 раз и вычислим период колебаний ненагруженной платформы по формуле
\[
T=\frac{\langle t\rangle}{N},
\]
где $\langle t\rangle$ --- среднее время $N$ колебаний. Результаты (эти и дальнейшие) занесём в таблицу~\ref{TbOne}.

В используемой нами установке радиус нижней платформы составляет
\[
R=78{,}05\pm0{,}05\;\text{мм},
\]
а верхней ---
\[
r=28{,}85\pm0{,}05\;\text{мм}.
\]
Масса нижней платформы равна
\[
m_\text{пл}=51{,}6\pm0{,}01\;\text{г}.
\]
Нити, которыми нижняя платформа крепится к верхней, имеют длину
\[
l=70{,}0\pm0{,}2\;\text{см}.
\]
При данных значениях величин формула~\eqref{eq:4} принимает вид
\[
I=0{,}000799\cdot mT^2.
\]

Вычислим момент инерции платформы $I_\text{пл}$ по формуле~\eqref{eq:4}:
\[
I_\text{пл}=0{,}000799\cdot m_\text{пл}T_\text{пл}^2\approx1{,}70\cdot10^{-4}\;\text{кг$\cdot$м$^2$}.
\]

Далее возьмём два одинаковых цилиндра массами
\[
m=500\;\text{г}
\]
и радиусами
\[
R=2{,}05\pm0{,}01\;\text{см}.
\]
Поставим их друг на друга в центр платформы. Вычислим момент инерции этой системы и занесём сведения в таблицу~\ref{TbOne}:
\[
I_{1}=0{,}000799\cdot(2m+m_\text{пл})T_{1}^2\approx3{,}89\cdot10^{-4}\;\text{кг$\cdot$м$^2$}.
\]
Тогда момент инерции одного цилиндра относительно оси, проходящей через центр масс равен
\[
I_0=\frac{I_{1}-I_\text{пл}}{2}\approx1{,}10\cdot10^{-4}\;\text{кг$\cdot$м$^2$}.
\]

\begin{table}[h]
	\begin{center}
	\begin{tabular}{|c|c|c|c|c|c|c|c|c|c|}
		\hline
		Тело & \parbox[c][3.5em]{3em}{Номер опыта} & $t_i$, с & $\Delta t$, с & $T$, с & $\Delta T$, с & $\varepsilon_T$, \% & \parbox[c]{2.6em}{$I$, кг$\cdot$м$^2\cdot$ $\cdot10^{-4}$} & \parbox[c]{2.6em}{$\Delta I$, кг$\cdot$м$^2\cdot$ $\cdot10^{-4}$} & $\varepsilon_I$, \%\\
		\hline
		\hline
		\multirow{7}{*}{\parbox[c]{4.1em}{Пустая платформа}} & 1 & $20{,}20$ & \multirow{7}{*}{$0{,}05$} & \multirow{7}{*}{$2{,}030$} & \multirow{7}{*}{$0{,}005$} & \multirow{7}{*}{$0{,}25$} & \multirow{7}{*}{$1{,}70$} & \multirow{7}{*}{$0{,}01$} & \multirow{7}{*}{$0{,}61$} \\
		\cline{2-3}
		& 2 & $20{,}36$ & & & & & & & \\
		\cline{2-3}
		& 3 & $20{,}23$ & & & & & & & \\
		\cline{2-3}
		& 4 & $20{,}28$ & & & & & & & \\
		\cline{2-3}
		& 5 & $20{,}25$ & & & & & & & \\
		\cline{2-3}
		& 6 & $20{,}32$ & & & & & & & \\
		\cline{2-3}
		& $\langle t\rangle$ & $20{,}27$ & & & & & & & \\
		\hline
		\hline
		\multirow{7}{*}{\parbox[c]{4em}{Два тела в центре платформы}} & 1 & $6{,}79$ & \multirow{7}{*}{$0{,}03$} & \multirow{7}{*}{$0{,}680$} & \multirow{7}{*}{$0{,}003$} & \multirow{7}{*}{$0{,}44$} & \multirow{7}{*}{$3{,}89$} & \multirow{7}{*}{$0{,}04$} & \multirow{7}{*}{$0{,}94$} \\
		\cline{2-3}
		& 2 & $6{,}72$ & & & & & & & \\
		\cline{2-3}
		& 3 & $6{,}76$ & & & & & & & \\
		\cline{2-3}
		& 4 & $6{,}75$ & & & & & & & \\
		\cline{2-3}
		& 5 & $6{,}75$ & & & & & & & \\
		\cline{2-3}
		& 6 & $6{,}76$ & & & & & & & \\
		\cline{2-3}
		& $\langle t\rangle$ & $6{,}76$ & & & & & & & \\
		\hline
		\hline
		\multirow{7}{*}{\parbox[c]{4em}{Два тела на расстоянии $a$ от центра платформы}} & 1 & $22{,}83$ & \multirow{7}{*}{$0{,}15$} & \multirow{7}{*}{$2{,}310$} & \multirow{7}{*}{$0{,}015$} & \multirow{7}{*}{$0{,}65$} & \multirow{7}{*}{$44{,}84$} & \multirow{7}{*}{$0{,}60$} & \multirow{7}{*}{$1{,}34$} \\
		\cline{2-3}
		& 2 & $23{,}06$ & & & & & & & \\
		\cline{2-3}
		& 3 & $23{,}08$ & & & & & & & \\
		\cline{2-3}
		& 4 & $23{,}11$ & & & & & & & \\
		\cline{2-3}
		& 5 & $23{,}17$ & & & & & & & \\
		\cline{2-3}
		& 6 & $23{,}22$ & & & & & & & \\
		\cline{2-3}
		& $\langle t\rangle$ & $23{,}08$ & & & & & & & \\
		\hline
	\end{tabular}
	\caption{Результаты измерений и вычислений}\label{TbOne}
	\end{center}
\end{table}

Теперь положим цилиндры на платформу симметрично относительно оси вращения, так, чтобы каждый из них лежал на расстоянии
\[
a=6{,}35\pm0{,}07\;\text{см}
\]
от неё. Аналогичным образом определим момент инерции системы тел
\[
I_2=0{,}000799\cdot(2m+m_\text{пл})T_{2}^2\approx44{,}84\cdot10^{-4}\;\text{кг$\cdot$м$^2$}
\]
и каждого тела в отдельности
\[
I_T=\frac{I_{2}-I_\text{пл}}{2}=21{,}57\cdot10^{-4}\;\text{кг$\cdot$м$^2$}.
\]

Перейдём к вычислению погрешностей. Доверительную вероятность примем равной $P=95\%$. Тогда коэффициент Стьюдента для $n=6$ измерений оказывается равным
\[
t_P(n)=2{,}57.
\]

Для измерения времени систематическая погрешность высчитывается следующим образом:
\begin{gather*}
\Delta t_\text{пр}=0{,}01\;\text{с},\quad\Delta t_\text{окр}=P\cdot\frac{\Delta t_\text{пр}}{2}\approx0{,}005\;\text{с}, \\
\Delta t_\text{сист}=\sqrt{(\Delta t_\text{пр})^2+(\Delta t_\text{окр})^2}\approx0{,}01\;\text{с}.
\end{gather*}
Средние ошибки измерений времени вычисляются по формуле
\[
S=\sqrt{\frac{\sum\limits_{i=1}^n\bigl(t_i-\langle t\rangle\bigr)^2}{n(n-1)}}.
\]
Так,
\[
S_\text{пл}\approx0{,}02\;\text{с},\quad S_1\approx0{,}01\;\text{с},\quad S_2\approx0{,}06\;\text{с}.
\]
Случайные ошибки вычислим по формуле
\[
\Delta t_\text{случ}=t_P(n)\cdot S.
\]
Тогда
\[
\Delta t_\text{случ, пл}\approx0{,}05\;\text{с},\quad\Delta t_\text{случ, 1}\approx0{,}03\;\text{с}\quad\Delta t_\text{случ, 2}\approx0{,}15\;\text{с}.
\]
Наконец, по формуле
\[
\Delta t=\sqrt{(\Delta t_\text{случ})^2+(\Delta t_\text{сист})^2}
\]
вычислим абсолютные полные погрешности и запишем результаты в таблицу~\ref{TbOne}.

Значения периодов получены в результате косвенных измерений, а значит полная абсолютная погрешность их измерения вычисляется по формуле
\[
\Delta T=\sqrt{\left(\frac{\partial T}{\partial\langle t\rangle}\,\Delta t\right)^2}=\left|\frac{dT}{d\langle t\rangle}\,\Delta t\right|=\frac{\Delta t}{N},
\]
а относительная --- по формуле
\[
\varepsilon_T=\frac{\Delta T}{T}\cdot100\%.
\]

Найдём погрешность нахождения момента инерции платформы:
\[
\Delta I=\sqrt{\left(\frac{\partial I}{\partial m}\,\Delta m\right)^2+\left(\frac{\partial I}{\partial R}\,\Delta R\right)^2+\left(\frac{\partial I}{\partial r}\,\Delta r\right)^2+\left(\frac{\partial I}{\partial l}\,\Delta l\right)^2+\left(\frac{\partial I}{\partial T}\,\Delta T\right)^2}.
\]
В результате имеем
\begin{multline*}
\Delta I=\frac{g}{4\pi^2l}T^2\cdot \\
\cdot\sqrt{\left(Rr\,\Delta m\right)^2+\left(mr\,\Delta R\right)^2+\left(mR\,\Delta r\right)^2+\left(\frac{mRr}{l}\,\Delta l\right)^2+\left(\frac{2mRr}{T}\,\Delta T\right)^2}.
\end{multline*}
Разделим полученное на $I$ и умножим на $100\%$:
\[
\varepsilon_I=\frac{\Delta I}{I}\cdot100\%=\sqrt{\varepsilon_m^2+\varepsilon_R^2+\varepsilon_r^2+\varepsilon_l^2+4\varepsilon^2_T}.
\]
По той же формуле вычисляется относительная погрешность измерения момента инерции системы "платформа $+$ тело"{}, где
\[
\varepsilon_m=\frac{0{,}001}{m},\;\%;\quad \varepsilon_R\approx0{,}06\%;\quad\varepsilon_r\approx0{,}17\%;\quad\varepsilon_l\approx0{,}29\%.
\]
Тогда
\[
\varepsilon_{I_\text{С}}=\sqrt{\frac{10^{-6}}{m^2}+4\varepsilon^2_T+0{,}1166}.
\]
При этом,
\[
\Delta I_\text{С}=\varepsilon_{I_\text{С}}\cdot I_\text{С}\cdot10^{-2},
\]
как и для пустой платформы. 

Однако, нас больше интересуют погрешности для тел в отдельности от платформы:
\[
\Delta I=\frac{\sqrt{(\Delta I_\text{С})^2+(\Delta I_\text{пл})^2}}{2}
\]
и
\[
\varepsilon_I=\frac{\Delta I}{I}\cdot 100\%.
\]

По теореме Гюйгенса-Штейнера, момент инерции тела, расположенного на расстоянии $a$ от оси, относительно этой оси равен
\[
I=I_0+ma^2.
\]
Вычислим его и занесём данные в таблицу~\ref{TbTwo}.

\begin{table}[h]
	\begin{center}
	\begin{tabular}{|c|c|c|c|c|c|}
		\hline
		\parbox[c][3.5em]{5em}{$I_0\pm\Delta I_0$, кг$\cdot$м$^2$$\cdot$$10^{-4}$} & $\varepsilon_{I_0}$, \% & \parbox[c][3.5em]{5em}{$I_T\pm\Delta I_T$, кг$\cdot$м$^2$$\cdot$$10^{-4}$} & $\varepsilon_{I_T}$, \% & $a$, м & \parbox[c][3.5em]{5em}{$I$, кг$\cdot$м$^2$$\cdot$$10^{-4}$} \\
		\hline
		\hline
		$1{,}10\pm0{,}02$ & $1{,}82\%$ & $21{,}57\pm0{,}30$ & $1{,}39\%$ & $0{,}0635$ & $21{,}26$ \\
		\hline
	\end{tabular}
	\caption{Сравнение экспериментальных и теоритических}\label{TbTwo}
	\end{center}
\end{table}