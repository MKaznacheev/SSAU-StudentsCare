\subsection{Определение плотности твердого тела гидростатическим взвешиванием}
Определим массу $m$ взвешиваемого тела в воздухе. Затем определим его массу $m_1$ вместе с проволокой. Затем опустим тело на проволоке  в воду и снова измерим его массу $m_2$. Результаты занесём в таблицу~\eqref{TbThree}.

\begin{table}[h!]
	\begin{center}
	\begin{tabular}{|c|c|c|c|}
		\hline
		$m$, г & $m_1$, г & $m_2$, г & $W=m_1-m_2$, г \\
		\hline
		$28{,}810$ & $28{,}890$ & $17{,}510$ & $11{,}380$ \\
		\hline
	\end{tabular}
	\caption{Результаты измерения массы}\label{TbThree}
	\end{center}
\end{table}

По формуле~\eqref{eq:2} вычислим плотность тела:
\[
\rho\approx2{,}5230\;\text{г/см}^3.
\]
Абсолютная ошибка измерения плотности вычисляется, как
\[
\begin{split}
	\Delta\rho & =\sqrt{\left(\frac{\partial\rho}{\partial m}\,\Delta m\right)^2+\left(\frac{\partial\rho}{\partial m_1}\,\Delta m_1\right)^2+\left(\frac{\partial\rho}{\partial m_2}\,\Delta m_2\right)^2}= \\
	& =\Delta m\sqrt{\left(\frac{\partial\rho}{\partial m}\right)^2+\left(\frac{\partial\rho}{\partial m_1}\right)^2+\left(\frac{\partial\rho}{\partial m_2}\right)^2},
\end{split}
\]
поскольку $\Delta m_1=\Delta m_2=\Delta m$. При этом,
\begin{gather*}
\frac{\partial\rho}{\partial m}=\frac{1}{m_1-m_2}(\delta-\lambda), \\
\frac{\partial\rho}{\partial m_1}=-\frac{m}{(m_1-m_2)^2}(\delta-\lambda), \\
\frac{\partial\rho}{\partial m_2}=\frac{m}{(m_1-m_2)^2}(\delta-\lambda),
\end{gather*}
а значит,
\[
\Delta\rho=\frac{\sqrt{W^2+2m^2}}{W^2}(\delta-\lambda)\,\Delta m\approx0{,}0023\;\text{г/см}^3
\]
и
\[
\varepsilon_\rho=\frac{\Delta\rho}{\rho}\cdot100\%\approx0{,}09\%.
\]
Результаты запишем в таблицу~\ref{TbFour}.

\begin{table}[h!]
	\begin{center}
	\begin{tabular}{|c|c|c|c|}
		\hline
		$\rho$, г/см$^3$ & $\Delta\rho$, г/см$^3$ & $\varepsilon_\rho$, \% & $\rho\pm\Delta\rho$, г/см$^3$ \\
		\hline
		$2{,}5230$ & $0{,}0023$ & $0{,}09$ & $2{,}5230\pm0{,}0023$ \\
		\hline
	\end{tabular}
	\caption{Результаты вычисления плотности}\label{TbFour}
	\end{center}
\end{table}