\section{Вывод}

Плотность тела, вычисленная в первом упражнении, приблизительно соответствует плотности свинца (около $11{,}4\;\text{г/см}^3$). Плотность тела во втором опыте близка к плотности стекла (около $2{,}5\;\text{г/см}^3$). При этом относительная ошибка измерения плотности в первом опыте примерно в $4$ раза превосходит таковую во втором, что говорит о превосходстве метода гидростатического взвешивания над методом определения плотности при помощи пикнометра.