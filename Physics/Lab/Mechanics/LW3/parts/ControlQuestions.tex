\section{Контрольные вопросы}

\begin{Enumerate}
	\item Плотностью вещества называется величина, численно равная массе этого вещества, содержащейся в единице объема:
	\[
	\rho=\frac{dm}{dV}.
	\]
	Она измеряется в $\text{кг}/\text{м}^3$.
	
	\item Закон Архимеда имеет следующую формулировку: на тело, погружённое в жидкость или газ, действует выталкивающая сила, численно равная весу объёма жидкости или газа, вытесненного телом. Иначе говоря, истинно соотношение
	\[
	F_A=\rho gV,
	\]
	где $\rho$ --- плотность жидкости или газа, $g$ --- ускорение свободного падения, $V$ --- объём части тела, погружённой в жидкость или газ, $F_A$ --- сила Архимеда.
	
	\item При уменьшении температуры плотность твёрдого тела увеличивается.
	
	\item Плотность воды имеет максимальное значение при $4\;^\circ C$ и уменьшается как с повышением, так и с понижением температуры относительно этого значения. Особенности изменения плотности воды связаны с перестройкой её молекулярной структуры.
	
	\item Будем считать, что величины $\Delta m$, $\Delta M$, $\Delta M_0$, $\Delta m_1$ и $\Delta m_2$ не обязательно равны попарно. Для удобства введём обозначение $\mu=m+M-M_0$. Заметим следующее:
	\[
	\frac{\partial\rho}{\partial m}=\left(\frac{1}{\mu}-\frac{m}{\mu^2}\right)\cdot(\delta-\lambda).
	\]
	Тогда в случае первого упражнения погрешность измерения плотности выразится формулой
	\[
	\Delta\rho=(\delta-\lambda)\sqrt{\left(\frac{1}{\mu}-\frac{m}{\mu^2}\right)^2(\Delta m)^2+\left(\frac{m}{\mu^2}\,\Delta M\right)^2+\left(\frac{m}{\mu^2}\,\Delta M_0\right)^2}.
	\]
	Поскольку, в нашем случае, $\lambda\ll\rho$, будем считать, что
	\[
	\rho=\frac{m}{\mu}(\delta-\lambda).
	\]
	Тогда
	\[
	\varepsilon_\rho=\frac{\Delta\rho}{\rho}\cdot100\%=\sqrt{\left(\frac{\Delta m}{m}\right)^2-2\frac{(\Delta m)^2}{m\mu}+\frac{(\Delta m)^2+(\Delta M)^2+(\Delta M_0)^2}{\mu^2}}\cdot100\%. 
	\]
	Наконец, заметим, что в условиях нашего опыта дробь
	\[
	2\frac{(\Delta m)^2}{m\mu}
	\]
	не оказывает значительного влияния на результат, а потому исключим её. В конечном счёте, имеем
	\[
	\varepsilon_\rho=\sqrt{\left(\frac{\Delta m}{m}\right)^2+\frac{(\Delta m)^2+(\Delta M)^2+(\Delta M_0)^2}{\mu^2}}\cdot100\%
	\]
	и
	\[
	\Delta\rho=\frac{\rho\cdot\varepsilon_\rho}{100\%}.
	\]
	
	В случае второго упражнения, получается
	\[
	\Delta\rho=(\delta-\lambda)\sqrt{\left(\frac{1}{W}\,\Delta m\right)^2+\left(\frac{m}{W^2}\,\Delta m_1\right)^2+\left(\frac{m}{W^2}\,\Delta m_2\right)^2}.
	\]
	Аналогично видоизменим формулу для вычислсения плотности:
	\[
	\rho=\frac{m}{W}(\delta-\lambda).
	\]
	Тогда
	\[
	\varepsilon_\rho=\frac{\Delta\rho}{\rho}\cdot100\%=\sqrt{\left(\frac{\Delta m}{m}\right)^2+\frac{(\Delta m_1)^2+(\Delta m_2)^2}{W^2}}\cdot100\% 
	\]
	и
	\[
	\Delta\rho=\frac{\rho\cdot\varepsilon_\rho}{100\%}.
	\]
\end{Enumerate}