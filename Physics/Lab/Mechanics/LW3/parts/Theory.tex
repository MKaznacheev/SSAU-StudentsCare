\section{Теоретические сведения}

В методе определения плотности твердого тела при помощи пикнометра исследуемое твердое тело взвешивают в воздухе и затем погружают в пикнометр с водой. На пикнометре имеется метка, до уровня которой должна доходить налитая жидкость.

Согласно закону Архимеда, плотность тела равна
\[
\rho'=\frac{m}{m_0}\delta,
\]
где $M$ --- масса тела в воздухе, $m_0$ – масса воды, вытесненная телом, $\delta$ – плотность воды. Пусть $M$ – масса пикнометра с водой, $M_0$ – масса пикнометра с водой и твердым телом. Тогда
\[
m_0=M+m-M_0
\]
и
\[
\rho'=\frac{m}{M+m-M_0}\delta.
\]
С учётом кажущейся потери веса при взвешивании в воздухе, имеем
\begin{equation}\label{eq:1}
\rho=\frac{m}{m+M-M_0}(\delta-\lambda)+\lambda,
\end{equation}
где $\lambda$ --- плотность воздуха.

При измерении плотности методом гидростатического взвешивания исследуемое твердое тело сначала взвешивают в воздухе и затем, подвесив его на тонкой проволочке, взвешивают в воде. Масса вытесненной телом воды равна
\[
W=m_1-m_2,
\]
где $m_1$ --- масса тела с проволочкой в воздухе, $m2$ --- масса тела в воде. По закону Архимеда плотность тела равна
\[
\rho=\frac{m}{W}\delta.
\]
Учитывая действие выталкивающей силы при взвешивании тела и воды в воздухе, получим
\begin{equation}\label{eq:2}
\rho=\frac{m}{m_1-m_2}(\delta-\lambda)+\lambda.
\end{equation}