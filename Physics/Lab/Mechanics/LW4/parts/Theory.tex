\section{Теоретические сведения}

В физике единицей времени считается атомная секунда, то есть временной промежуток, за который совершается $9\;192\;631\;770$, колебаний электромагнитного излучения, которое соответствует переходу между двумя определенными сверхтонкими уровнями основного состояния атома $\mathrm{^{133}Ce}$ в отсутствие внешних полей.

Один из методов измерения продолжительных промежутков времени --- механический. Он предполагает использование таких приборов, как часы, метроном, секундомер или хронограф в качестве механических автоколебательных систем. 