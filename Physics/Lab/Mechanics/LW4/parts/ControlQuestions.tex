\section{Контрольные вопросы}

\begin{Enumerate}
	\item Для измерения промежутков времени применяются механические, стробоскопические и электрические методы.
	
	\item Система часового механизма является автоколебательной, поскольку колебания в ней некоторое время поддерживаются без вмешательства человека за счёт перехода кинетической энергии механизма в потенциальную и наоборот.
	
	\item Стробоскопический метод измерения времени используется, когда требуется измерить периоды или частоты периодических процессов. В основе данного метода лежит освещение отдельными короткими вспышками, следующими через равные промежутки времени.
	
	\item В системе СИ единицей измерения времени признаётся секунда. Так же используются такие единицы, как минута, час, сутки, неделя, месяц, год, век и так далее.
\end{Enumerate}