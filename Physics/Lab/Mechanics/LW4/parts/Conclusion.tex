\section{Вывод}

Периоды колебаний математического маятника вычислены с достаточной точностью: относительные ошибки не превышают $0{,}07\%$. Разница между теоретическеми и экспериментальными значениями периодов составляет не более $8\;\text{мс}$. При этом, чем меньше число колебаний, тем точнее измерения, ввиду уменьшения влияния эффекта затухания колебаний. В среднем, относительные ошибки для измерения периодов при 20 колебаниях превышают таковые для 10 в $1{,}8$ раз.

Точность метода так же подтверждается графиками соответствий длин математического маятника от квадратов периодов его колебаний. Для теоретических и экспериментальных значений они практически совпадают, а разница между наклонами "экспериментальных"{} прямых составляет всего лишь $0{,}01^\circ$.

Таким образом, механический метод измерения времени, исследуемый в настоящей работе, можно считать удовлетворительно точным.