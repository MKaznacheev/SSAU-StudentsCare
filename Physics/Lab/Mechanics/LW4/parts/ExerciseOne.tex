\section{Результаты измерений и обработка данных}

\begin{table}[h!]
	\begin{center}
	\begin{tabular}{|c|c|c|c|c|c|c|c|c|}
		\hline
		\multicolumn{9}{|c|}{$N_1=10$} \\
		\hline
		\hline
		$k$ & \parbox[c][2.5em]{1.1em}{$l_k$, см} & i & \parbox[c][2.5em]{1.3em}{$t_{ki}$, с} & \parbox[c][2.5em]{1.1em}{$\bar t_k$, с} & \parbox[c][2.5em]{2.2em}{$\Delta t_{ki}$, с} & \parbox[c][2.5em]{3.3em}{$(\Delta t_{ki})^2$, с$^2$} & \parbox[c][2.5em]{1.6em}{$S_{\bar t_k}$, с} & \parbox[c][2.5em]{1.9em}{$\Delta t_k$, с} \\
		\hline
		\hline
		\multirow{3}{*}{1} & \multirow{3}{*}{30} & 1 & $11{,}026$ & \multirow{3}{*}{$11{,}027$} & $-0{,}001$ & $10^{-6}$ & \multirow{3}{*}{$0{,}0007$} & \multirow{3}{*}{$0{,}003$} \\
		\cline{3-4}\cline{6-7}
		& & 2 & $11{,}026$ & & $-0{,}001$ & $10^{-6}$ & & \\
		\cline{3-4}\cline{6-7}
		& & 3 & $11{,}028$ & & $0{,}001$ & $10^{-6}$ & & \\
		\hline
		\multirow{3}{*}{2} & \multirow{3}{*}{35} & 1 & $11{,}947$ & \multirow{3}{*}{$11{,}945$} & $0{,}002$ & $4\cdot10^{-6}$ & \multirow{3}{*}{$0{,}0012$} & \multirow{3}{*}{$0{,}005$} \\
		\cline{3-4}\cline{6-7}
		& & 2 & $11{,}946$ & & $0{,}001$ & $10^{-6}$ & & \\
		\cline{3-4}\cline{6-7}
		& & 3 & $11{,}943$ & & $-0{,}002$ & $4\cdot10^{-6}$ & & \\
		\hline
		\multirow{3}{*}{3} & \multirow{3}{*}{40} & 1 & $12{,}649$ & \multirow{3}{*}{$12{,}650$} & $-0{,}001$ & $10^{-6}$ & \multirow{3}{*}{$0{,}0007$} &\multirow{3}{*}{$0{,}003$} \\
		\cline{3-4}\cline{6-7}
		& & 2 & $12{,}651$ & & $0{,}001$ & $10^{-6}$ & & \\
		\cline{3-4}\cline{6-7}
		& & 3 & $12{,}649$ & & $-0{,}001$ & $10^{-6}$ & & \\
		\hline
		\multirow{3}{*}{4} & \multirow{3}{*}{45} & 1 & $13{,}521$ & \multirow{3}{*}{$13{,}521$} & $0{,}000$ & 0 & \multirow{3}{*}{$0{,}0006$} &\multirow{3}{*}{$0{,}003$} \\
		\cline{3-4}\cline{6-7}
		& & 2 & $13{,}520$ & & $-0{,}001$ & $10^{-6}$ & & \\
		\cline{3-4}\cline{6-7}
		& & 3 & $13{,}522$ & & $0{,}001$ & $10^{-6}$ & & \\
		\hline
		\multirow{3}{*}{5} & \multirow{3}{*}{50} & 1 & $14{,}246$ & \multirow{3}{*}{$14{,}247$} & $-0{,}001$ & $10^{-6}$ & \multirow{3}{*}{$0{,}0014$} & \multirow{3}{*}{$0{,}006$} \\
		\cline{3-4}\cline{6-7}
		& & 2 & $14{,}246$ & & $-0{,}001$ & $10^{-6}$ & & \\
		\cline{3-4}\cline{6-7}
		& & 3 & $14{,}250$ & & $0{,}003$ & $9\cdot10^{-6}$ & & \\
		\hline
	\end{tabular}
	\caption{Результаты измерений и вычислений для $N_1=10$}\label{TbOne}
	\end{center}
\end{table}

\begin{table}[h!]
	\begin{center}
	\begin{tabular}{|c|c|c|c|c|c|c|c|c|}
		\hline
		\multicolumn{9}{|c|}{$N_2=20$} \\
		\hline
		\hline
		$k$ & \parbox[c][2.5em]{1.1em}{$l_k$, см} & i & \parbox[c][2.5em]{1.3em}{$t_{ki}$, с} & \parbox[c][2.5em]{1.1em}{$\bar t_k$, с} & \parbox[c][2.5em]{2.2em}{$\Delta t_{ki}$, с} & \parbox[c][2.5em]{3.3em}{$(\Delta t_{ki})^2$, с$^2$} & \parbox[c][2.5em]{1.6em}{$S_{\bar t_k}$, с} & \parbox[c][2.5em]{1.9em}{$\Delta t_k$, с} \\
		\hline
		\hline
		\multirow{3}{*}{1} & \multirow{3}{*}{30} & 1 & $22{,}062$ & \multirow{3}{*}{$22{,}057$} & $0{,}005$ & $25\cdot10^{-6}$ & \multirow{3}{*}{$0{,}0032$} & \multirow{3}{*}{$0{,}014$} \\
		\cline{3-4}\cline{6-7}
		& & 2 & $22{,}058$ & & $0{,}001$ & $10^{-6}$ & & \\
		\cline{3-4}\cline{6-7}
		& & 3 & $22{,}051$ & & $-0{,}006$ & $36\cdot10^{-6}$ & & \\
		\hline
		\multirow{3}{*}{2} & \multirow{3}{*}{35} & 1 & $23{,}889$ & \multirow{3}{*}{$23{,}892$} & $-0{,}003$ & $9\cdot10^{-6}$ & \multirow{3}{*}{$0{,}0035$} & \multirow{3}{*}{$0{,}015$} \\
		\cline{3-4}\cline{6-7}
		& & 2 & $23{,}888$ & & $-0{,}004$ & $16\cdot10^{-6}$ & & \\
		\cline{3-4}\cline{6-7}
		& & 3 & $23{,}899$ & & $0{,}007$ & $49\cdot10^{-6}$ & & \\
		\hline
		\multirow{3}{*}{3} & \multirow{3}{*}{40} & 1 & $25{,}298$ & \multirow{3}{*}{$25{,}294$} & $0{,}004$ & $16\cdot10^{-6}$ & \multirow{3}{*}{$0{,}0031$} & \multirow{3}{*}{$0{,}013$} \\
		\cline{3-4}\cline{6-7}
		& & 2 & $25{,}288$ & & $-0{,}006$ & $36\cdot10^{-6}$  & & \\
		\cline{3-4}\cline{6-7}
		& & 3 & $25{,}296$ & & $0{,}002$ & $4\cdot10^{-6}$  & & \\
		\hline
		\multirow{3}{*}{4} & \multirow{3}{*}{45} & 1 & $27{,}041$ & \multirow{3}{*}{$27{,}039$} & $0{,}002$ & $4\cdot10^{-6}$ & \multirow{3}{*}{$0{,}0025$} & \multirow{3}{*}{$0{,}011$} \\
		\cline{3-4}\cline{6-7}
		& & 2 & $27{,}042$ & & $0{,}003$ & $9\cdot10^{-6}$ & & \\
		\cline{3-4}\cline{6-7}
		& & 3 & $27{,}034$ & & $-0{,}005$ & $25\cdot10^{-6}$ & & \\
		\hline
		\multirow{3}{*}{5} & \multirow{3}{*}{50} & 1 & $28{,}506$ &\multirow{3}{*}{$28{,}506$} & $0{,}000$ & 0 & \multirow{3}{*}{$0{,}0000$} &  \multirow{3}{*}{$0{,}001$} \\
		\cline{3-4}\cline{6-7}
		& & 2 & $28{,}506$ & & $0{,}000$ & 0 & & \\
		\cline{3-4}\cline{6-7}
		& & 3 & $28{,}506$ & & $0{,}000$ & 0 & & \\
		\hline
	\end{tabular}
	\caption{Результаты измерений и вычислений для $N_2=20$}\label{TbTwo}
	\end{center}
\end{table}

Доверительную вероятность примем равной $P=95\%$. Тогда коэффициент Стьюдента при $n=3$ измерениях будет равен $t_P(n)=4{,}3$.

Цена деления шкалы секундомера составляет $0{,}001\;\text{с}$. Тогда
\begin{gather*}
\Delta t_\text{пр}=0{,}001\;\text{с},\quad\Delta t_\text{окр}=P\cdot\frac{\Delta t_\text{пр}}{2}\approx0{,}0005\;\text{с}, \\
\Delta t_\text{сист}=\sqrt{(\Delta t_\text{пр})^2+(\Delta t_\text{окр})^2}\approx0{,}001\;\text{с}.
\end{gather*}

Измерения будем проводить при длинах маятника $30\;\text{см}$, $35\;\text{см}$, $40\;\text{см}$, $45\;\text{см}$ и $50\;\text{см}$. При этом сначала измерения проведём для $N_1=10$ (таб.~\ref{TbOne}), а потом для $N_2=20$ (таб.~\ref{TbTwo}) колебаний.

Для каждой длины, которой условно присвоен номер $k$, среднее время вычисляется по формуле
\[
\bar t_k=\frac{\sum\limits_{i=1}^nt_{ki}}{n}.
\]
Величина $\Delta t_{ki}$ вычисляется, как
\[
\Delta t_{ki}=t_{ki}-\bar t_k.
\]
По формуле
\[
S_{\bar t_k}=\sqrt{\frac{\sum\limits_{i=1}^n(\Delta t_{ki})^2}{n(n-1)}}
\]
высчитывается средняя ошибка для времени, а по формуле
\[
\Delta t_{k,\text{случ}}=t_P(n)\cdot S_{\bar t_k}.
\]
--- случайная погрешность. Наконец, полную абсолютную погрешность найдём, как
\[
\Delta t_k=\sqrt{(\Delta t_{k,\text{случ}})^2+(\Delta t_\text{сист})^2}=\sqrt{\bigl(t_P(n)\cdot S_{\bar t_k}\bigr)^2+(\Delta t_\text{сист})^2}.
\]

\begin{table}[h!]
	\begin{center}
	\begin{tabular}{|c|c|c|c|c|}
		\hline
		\multicolumn{5}{|c|}{$N_1=10$} \\
		\hline
		\hline
		$k$ & \parbox[c][2.5em]{1.1em}{$l_k$, см} & \parbox[c][2.5em]{1.3em}{$T_k$, с} & \parbox[c][2.5em]{2.2em}{$\Delta T_k$, с} & \parbox[c][2.5em]{1.6em}{$\varepsilon_{T_k}$, \%} \\
		\hline
		\hline
		1 & 30 & $1{,}1027$ & $0{,}0003$ & $0{,}03$ \\
		\hline
		2 & 35 & $1{,}1945$ & $0{,}0005$ & $0{,}04$ \\
		\hline
		3 & 40 & $1{,}2650$ & $0{,}0003$ & $0{,}02$ \\
		\hline
		4 & 45 & $1{,}3521$ & $0{,}0003$ & $0{,}02$ \\
		\hline
		5 & 50 & $1{,}4247$ & $0{,}0006$ & $0{,}04$ \\
		\hline
	\end{tabular}
	\caption{Результаты вычисления периода для $N_1=10$}\label{TbThree}
	\end{center}
\end{table}

\begin{table}[h!]
	\begin{center}
	\begin{tabular}{|c|c|c|c|c|}
		\hline
		\multicolumn{5}{|c|}{$N_2=20$} \\
		\hline
		\hline
		$k$ & \parbox[c][2.5em]{1.1em}{$l_k$, см} & \parbox[c][2.5em]{1.3em}{$T_k$, с} & \parbox[c][2.5em]{2.2em}{$\Delta T_k$, с} & \parbox[c][2.5em]{1.6em}{$\varepsilon_{T_k}$, \%} \\
		\hline
		\hline
		1 & 30 & $1{,}1029$ & $0{,}0007$ & $0{,}06$ \\
		\hline
		2 & 35 & $1{,}1946$ & $0{,}0008$ & $0{,}07$ \\
		\hline
		3 & 40 & $1{,}2647$ & $0{,}0007$ & $0{,}06$ \\
		\hline
		4 & 45 & $1{,}3520$ & $0{,}0006$ & $0{,}04$ \\
		\hline
		5 & 50 & $1{,}4253$ & $0{,}0001$ & $0{,}01$ \\
		\hline
	\end{tabular}
	\caption{Результаты вычисления периода для $N_2=20$}\label{TbFour}
	\end{center}
\end{table}

Для каждой длины найдём период колебаний маятника:
\[
T_k=\frac{\bar t_k}{N}.
\]
Абсолютная ошибка этой величины равна
\[
\Delta T_k=\sqrt{\left(\frac{\partial T_k}{\partial\bar t_k}\,\Delta t_k\right)^2}=\left|\frac{dT_k}{d\bar t_k}\,\Delta t_k\right|=\frac{\Delta t_k}{N},
\]
а относительная ---
\[
\varepsilon_{T_k}=\frac{\Delta T_k}{T_k}\cdot100\%.
\]

Вычислим теоретические значения периодов $\Theta_l$ по формуле
\[
\Theta_l=2\pi\sqrt{\frac{l}{g}}.
\]
Так,
\[
\Theta_{30}\approx1{,}0993\;\text{с},\quad\Theta_{35}\approx1{,}1874\;\text{с},\quad\Theta_{40}\approx1{,}2694\;\text{с},\quad\Theta_{45}\approx1{,}3464\;\text{с},\quad\Theta_{50}\approx1{,}4192\;\text{с}.
\]

Рассмотрим функцию $T^2=f(l)$. Как показывает теория, зависимость между величинами $T^2$ и $l$ можно считать линейной и $f(l)=0$. Поскольку в опыте всегда присутствует некоторая неточность, точки графика функции $f(l)$ не будут лежать на одной прямой, а значит следует провести аппроксимацию. Воспользуемся методом наименьших квадратов. Очевидно, что
\[
f(l)=kl,
\]
где
\[
k=\frac{\left\langle lT^2\right\rangle}{\langle l^2\rangle}.
\]
Ошибка коэффициента вычисляется, как
\[
\Delta k=\frac{t_P(n'-2)}{5\sqrt{n'-1}}\sqrt{\frac{\sum\limits_{i=1}^nl_i^2\sum\limits_{i=1}^nT_i^4-\left(\sum\limits_{i=1}^nl_iT_i^2\right)^2}{\langle l^2\rangle^2}},
\]
где $n'=5$.
Поскольку
\begin{gather*}
t_P(n'-2)=t_P(n)=4{,}3,\quad\left\langle l^2\right\rangle=0{,}165\;\text{м}^2, \\
\left\langle lT_{10}^2\right\rangle\approx0{,}6684\;\text{м}\cdot\text{с}^2,\quad\left\langle lT_{20}^2\right\rangle\approx0{,}6685\;\text{м}\cdot\text{с}^2,
\end{gather*}
где $T_i$ --- значение периода при $i$ колебаниях, то
\begin{gather*}
k_{10}\approx4{,}0510\pm0{,}0547\;\text{с$^2$/м},\quad k_{20}\approx4{,}0515\pm0{,}0572\;\text{с$^2$/м}.
\end{gather*}
То есть
\[
T_{10}^2=4{,}0510\cdot l,\quad T_{20}^2=4{,}0515\cdot l.
\]
При этом для квадрата теоретического значения периода $\varphi(l)=\Theta^2$ имеем
\[
\Theta^2=\frac{4\pi^2}{g}\cdot l\approx4{,}0284\cdot l
\]
(далее угловой коэффициент этой прямой будет обозначен через $\kappa$).

	
\begin{figure}[h!]
	\centering
	\begin{tikzpicture}
		\begin{axis}[x=25cm, xmin=0, xmax=0.55, ymin=0, ymax=2.2, axis lines=middle, xlabel=$l$, ylabel=$T_{10}^2$, xtick={0.1, 0.15, ..., 0.5}, ytick={0.2, 0.4, ..., 2}, domain=0:1]
		\addplot[samples=2, red, thick]{4.0510*x};
		\addplot[only marks, black] table {
		0.3 1.2160
		0.35 1.4268
		0.4 1.6002
		0.45 1.8282
		0.5 2.0298
		};
		\end{axis}
	\end{tikzpicture}
	\caption{График засисмости $T_{10}^2(l)$}
\end{figure}

\begin{figure}[h!]
	\centering
	\begin{tikzpicture}
		\begin{axis}[x=25cm, xmin=0, xmax=0.55, ymin=0, ymax=2.2, axis lines=middle, xlabel=$l$, ylabel=$T_{20}^2$, xtick={0.1, 0.15, ..., 0.5}, ytick={0.2, 0.4, ..., 2}, domain=0:1]
		\addplot[samples=2, blue, thick]{4.0515*x};
		\addplot[only marks, black] table {
		0.3 1.2164
		0.35 1.4271
		0.4 1.5995
		0.45 1.8279
		0.5 2.0315
		};
		\end{axis}
	\end{tikzpicture}
	\caption{График засисмости $T_{20}^2(l)$}
\end{figure}

\begin{figure}[h!]
	\centering
	\begin{tikzpicture}
		\begin{axis}[x=25cm, xmin=0, xmax=0.55, ymin=0, ymax=2.2, axis lines=middle, xlabel=$l$, ylabel=$\Theta^2$, xtick={0.1, 0.15, ..., 0.5}, ytick={0.2, 0.4, ..., 2}, domain=0:1]
		\addplot[samples=2, black, thick]{4.0284*x};
		\end{axis}
	\end{tikzpicture}
	\caption{График засисмости $\Theta^2(l)$}
\end{figure}

Ввиду того, что обе прямые $f(l)$ и прямая $\varphi(l)$ практически совпадают, не имеет смысла пытаться изобразить их на одной координатной плоскости. Это подтверждается величинами углов, которые прямые составляют с осью $Ol$:
\begin{gather*}
\alpha=\arctg k_{10}\approx76{,}13^\circ, \\
\beta=\arctg k_{20}\approx76{,}14^\circ, \\
\gamma=\arctg\kappa\approx76{,}06^\circ.
\end{gather*}