\subsection{3}
Найдём число молекул и атомов водорода в объёме $V(\mathrm{H_2})=55{,}6\;\text{л}$. Так как
\[
\nu(\mathrm{H_2})=\frac{V(\mathrm{H_2})}{V_m},
\]
то
\[
N(\mathrm{H_2})=\nu(\mathrm{H_2})\cdot N_A=\frac{V(\mathrm{H_2})}{V_m}\cdot N_A=\frac{55{,}6}{22{,}4}\cdot6{,}02\cdot10^{23}\approx14{,}943\cdot10^{23}.
\]
Очевидно, что
\[
N(\mathrm{H})=2N(\mathrm{H_2})\approx29{,}886\cdot10^{23}.
\]