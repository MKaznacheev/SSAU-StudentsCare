\subsection{8}

Составим уравнение реакции
\[
\mathrm{K_2Cr_2O_7}+\mathrm{FeCl_2}+\mathrm{KOH}\longrightarrow\mathrm{[Cr(OH)_6]^{3-}}+\mathrm{Fe(OH)_3}.
\]
Полуреакции идут по следующим схемам:
\[
\begin{array}{r|l}
	1 & \mathrm{Cr_2O_7^{2-}}+7\mathrm{H_2O}+6e^-\longrightarrow2\mathrm{[Cr(OH)_6]^{3-}}+2\mathrm{OH^{-}} \\
	6 & \mathrm{Fe^{2+}}+3\mathrm{OH^-}-e^-\longrightarrow\mathrm{Fe(OH)_3}
\end{array}.
\]
Просуммируем полученное:
\[
\mathrm{Cr_2O_7^{2-}}+6\mathrm{Fe^{2+}}+16\mathrm{OH^-}+7\mathrm{H_2O}\longrightarrow2\mathrm{[Cr(OH)_6]^{3-}}+6\mathrm{Fe(OH)_3}.
\]
Допишем недостающие ионы:
\[
\mathrm{K_2Cr_2O_7}+6\mathrm{FeCl_2}+16\mathrm{KOH}+7\mathrm{H_2O}\longrightarrow2\mathrm{K_3[Cr(OH)_6]}+6\mathrm{Fe(OH)_3}+12\mathrm{KCl}.
\]