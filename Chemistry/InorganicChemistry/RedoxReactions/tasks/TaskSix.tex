\subsection{6}

Составим уравнение реакции
\[
\mathrm{KMnO_4}+\mathrm{FeCl_2}+\mathrm{KOH}\longrightarrow\mathrm{MnO_4^{2-}}+\mathrm{Fe(OH)_3}.
\]
Полуреакции идут по следующим схемам:
\[
\begin{array}{r|l}
	1 & \mathrm{MnO_4^{-}}+e^-\longrightarrow\mathrm{MnO_4^{2-}} \\
	1 & \mathrm{Fe^{2+}}+3\mathrm{OH^-}-e^-\longrightarrow\mathrm{Fe(OH)_3}
\end{array}.
\]
Просуммируем полученное:
\[
\mathrm{MnO_4^{-}}+\mathrm{Fe^{2+}}+3\mathrm{OH^-}\longrightarrow\mathrm{MnO_4^{2-}}+\mathrm{Fe(OH)_3}.
\]
Допишем недостающие ионы:
\[
\mathrm{KMnO_4}+\mathrm{FeCl_2}+3\mathrm{KOH}\longrightarrow\mathrm{K_2MnO_4}+\mathrm{Fe(OH)_3}+2\mathrm{KCl}.
\]