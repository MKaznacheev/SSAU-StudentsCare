\subsection{1}

Составим уравнение реакции
\[
\mathrm{KMnO_4}+\mathrm{FeCl_2}+\mathrm{H_2SO_4}\longrightarrow\mathrm{Mn^{2+}}+\mathrm{Fe^{3+}}.
\]
Полуреакции идут по следующим схемам:
\[
\begin{array}{r|l}
	1 & \mathrm{MnO_4^{-}}+8\mathrm{H^+}+5e^-\longrightarrow\mathrm{Mn^{2+}}+4\mathrm{H_2O} \\
	5 & \mathrm{Fe^{2+}}-e^-\longrightarrow\mathrm{Fe^{3+}}
\end{array}.
\]
Просуммируем полученное:
\[
\mathrm{MnO_4^{-}}+5\mathrm{Fe^{2+}}+8\mathrm{H^+}\longrightarrow\mathrm{Mn^{2+}}+5\mathrm{Fe^{3+}}
+4\mathrm{H_2O}.
\]
Допишем недостающие ионы:
\[
\mathrm{KMnO_4}+5\mathrm{FeCl_2}+4\mathrm{H_2SO_4}\longrightarrow\mathrm{MnSO_4}+3\mathrm{FeCl_3}+\mathrm{Fe_2(SO_4)_3}+\mathrm{KCl}
+4\mathrm{H_2O}.
\]