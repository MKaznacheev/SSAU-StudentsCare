\subsection{4}

Составим уравнение реакции
\[
\mathrm{K_2Cr_2O_7}+\mathrm{P}+\mathrm{H_2SO_4}\longrightarrow\mathrm{Cr^{3+}}+\mathrm{H_3PO_4}.
\]
Полуреакции идут по следующим схемам:
\[
\begin{array}{r|l}
	5 & \mathrm{Cr_2O_7^{2-}}+14\mathrm{H^+}+6e^-\longrightarrow2\mathrm{Cr^{3+}}+7\mathrm{H_2O} \\
	6 & \mathrm{P}+4\mathrm{H_2O}-5e^-\longrightarrow\mathrm{H_3PO_4}+5\mathrm{H^+}
\end{array}.
\]
Просуммируем полученное:
\[
5\mathrm{Cr_2O_7^{2-}}+6\mathrm{P}+40\mathrm{H^+}\longrightarrow10\mathrm{Cr^{3+}}+6\mathrm{H_3PO_4}+11\mathrm{H_2O}.
\]
Допишем недостающие ионы:
\[
5\mathrm{K_2Cr_2O_7}+6\mathrm{P}+20\mathrm{H_2SO_4}\longrightarrow5\mathrm{Cr_2(SO_4)_3}+6\mathrm{H_3PO_4}+\mathrm{5K_2SO_4}+11\mathrm{H_2O}.
\]