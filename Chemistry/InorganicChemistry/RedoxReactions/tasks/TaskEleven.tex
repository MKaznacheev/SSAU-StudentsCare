\subsection{11}

Составим уравнение реакции
\[
\mathrm{KMnO_4}+\mathrm{P}+\mathrm{H_2O}\longrightarrow\mathrm{MnO_2}+\mathrm{PO_4^{3-}}.
\]
Полуреакции идут по следующим схемам:
\[
\begin{array}{r|l}
	5 & \mathrm{MnO_4^-}+4\mathrm{H^+}+3e^-\longrightarrow\mathrm{MnO_2}+2\mathrm{H_2O} \\
	3 & \mathrm{P}+4\mathrm{H_2O}-5e^-\longrightarrow\mathrm{PO_4^{3-}}+8\mathrm{H^+}
\end{array}.
\]
Просуммируем полученное:
\[
5\mathrm{MnO_4^-}+3\mathrm{P}+2\mathrm{H_2O}\longrightarrow5\mathrm{MnO_2}+3\mathrm{PO_4^{3-}}+4\mathrm{H^+}.
\]
Допишем недостающие ионы:
\[
5\mathrm{KMnO_4}+3\mathrm{P}+2\mathrm{H_2O}\longrightarrow5\mathrm{MnO_2}+\mathrm{KH_2PO_4}+2\mathrm{K_2HPO_4}.
\]