\subsection{15}

Составим уравнение реакции
\[
\mathrm{K_2Cr_2O_7}+\mathrm{H_2O_2}\longrightarrow\mathrm{Cr(OH)_3}+\mathrm{O_2}.
\]
Полуреакции идут по следующим схемам:
\[
\begin{array}{r|l}
	1 & \mathrm{Cr_2O_7^{2-}}+8\mathrm{H^+}+6e^-\longrightarrow2\mathrm{Cr(OH)_3}+\mathrm{H_2O} \\
	3 & \mathrm{H_2O_2}-2e^-\longrightarrow\mathrm{O_2}+2\mathrm{H^+}
\end{array}.
\]
Просуммируем полученное:
\[
\mathrm{Cr_2O_7^{2-}}+3\mathrm{H_2O_2}+2\mathrm{H^+}\longrightarrow2\mathrm{Cr(OH)_3}+3\mathrm{O_2}+\mathrm{H_2O}.
\]
Поскольку в левой части уравнения физически не может быть катионов водорода, допишем в левую и правую части уравнения по $2$ гидроксид-иона:
\[
\mathrm{Cr_2O_7^{2-}}+3\mathrm{H_2O_2}+\mathrm{H_2O}\longrightarrow2\mathrm{Cr(OH)_3}+3\mathrm{O_2}+2\mathrm{OH^-}.
\]
Допишем недостающие ионы:
\[
\mathrm{K_2Cr_2O_7}+3\mathrm{H_2O_2}+\mathrm{H_2O}\longrightarrow2\mathrm{Cr(OH)_3}+3\mathrm{O_2}+2\mathrm{KOH}.
\]