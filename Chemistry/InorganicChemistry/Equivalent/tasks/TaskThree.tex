\subsection{3}

Известно, что из некоторого металла массой $m(\mathrm{Me})=3{,}24\;\text{г}$ можно получить сульфид массой $m(\mathrm{Me}_x\mathrm{S}_y)=3{,}72\;\text{г}$. Очевидно, что
\[
m(\mathrm{S})=m(\mathrm{Me}_x\mathrm{S}_y)-m(\mathrm{Me})=3{,}72-3{,}24=0{,}48\;\text{г}.
\]

По закону эквивалентных отношений,
\[
\frac{m(\mathrm{Me})}{m(\mathrm{S})}=\frac{M_\text{Э}(\mathrm{Me})}{M_\text{Э}(\mathrm{S})}\LR M_\text{Э}(\mathrm{Me})=\frac{m(\mathrm{Me})}{m(\mathrm{S})}\cdot M_\text{Э}(\mathrm{S}).
\]
Число атомов водорода, которым эквивалентна сера в реакции образования сульфида, равно~$z(\mathrm{S})=2$. Её фактор эквивалентности в таком случае равен
\[
f_\text{Э}(\mathrm{S})=\frac{1}{z(\mathrm{S})}=\frac{1}{2},
\]
а молярная масса эквивалента ---
\[
M_\text{Э}(\mathrm{S})=f_\text{Э}(\mathrm{S})\cdot M(\mathrm{S})=\frac{1}{2}\cdot32=16\;\text{г/экв}.
\]
Так,
\[
M_\text{Э}(\mathrm{Me})=\frac{3{,}24}{0{,}48}\cdot16\approx108\;\text{г/экв}.
\]