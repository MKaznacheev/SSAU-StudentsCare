\subsection{2}

Известно, что к образованию оксида массой $m(\mathrm{Me}_x\mathrm{O}_y)=9{,}44\;\text{г}$ приводит сгорание металла массой $m(\mathrm{Me})=5\;\text{г}$. Запишем уравнение реакции:
\[
2x\mathrm{Me}+y\mathrm{O_2}\longrightarrow2\mathrm{Me}_x\mathrm{O}_y.
\]
Очевидно, что
\[
m(\mathrm{O_2})=m(\mathrm{Me}_x\mathrm{O}_y)-m(\mathrm{Me})=9{,}44-5=4{,}44\;\text{г}.
\]

По закону эквивалентных отношений,
\[
\frac{m(\mathrm{Me})}{m(\mathrm{O_2})}=\frac{M_\text{Э}(\mathrm{Me})}{M_\text{Э}(\mathrm{O_2})}\LR M_\text{Э}(\mathrm{Me})=\frac{m(\mathrm{Me})}{m(\mathrm{O_2})}\cdot M_\text{Э}(\mathrm{O_2}).
\]
При этом число атомов водорода, которым эквивалентна молекула кислорода, равно~$z(\mathrm{O_2})=4$. Его фактор эквивалентности в таком случае равен
\[
f_\text{Э}(\mathrm{O_2})=\frac{1}{z(\mathrm{O_2})}=\frac{1}{4},
\]
а молярная масса эквивалента ---
\[
M_\text{Э}(\mathrm{O_2})=f_\text{Э}(\mathrm{O_2})\cdot M(\mathrm{O_2})=\frac{1}{4}\cdot32=8\;\text{г/экв}.
\]
Так,
\[
M_\text{Э}(\mathrm{Me})=\frac{5}{4{,}44}\cdot8\approx9\;\text{г/экв}.
\]