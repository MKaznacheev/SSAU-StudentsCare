\subsection{1}

Найдём фактор эквивалентности $f_\text{Э}$ и молярную массу эквивалента $M_\text{Э}$ в следующих реакциях.
\begin{Enumerate}
	\item Реакция полной нейтрализации азотной кислоты протекает по следующей схеме:
	\begin{equation}\label{eq1}
	\mathrm{HNO_3}+\mathrm{NaOH}\longrightarrow\mathrm{NaNO_3}+\mathrm{H_2O}.
	\end{equation}
	При этом число отданных кислотой катионов водорода $\mathrm{H^+}$ в ходе реакции составляет~$z=1$. Тогда
	\[
	f_\text{Э}(\mathrm{HNO_3})=\frac{1}{z}=\frac{1}{1}=1
	\]
	и
	\[
	M_\text{Э}(\mathrm{HNO_3})=f_\text{Э}(\mathrm{HNO_3})\cdot M(\mathrm{HNO_3})=1\cdot63=63\;\text{г/экв}.
	\]
	
	\item Реакция полной нейтрализации гидроксида натрия протекает по схеме~\eqref{eq1}.
	При этом число отданных основанием гидроксид-ионов $\mathrm{OH^-}$ в ходе реакции составляет~$z=1$. Тогда
	\[
	f_\text{Э}(\mathrm{NaOH})=\frac{1}{z}=\frac{1}{1}=1
	\]
	и
	\[
	M_\text{Э}(\mathrm{NaOH})=f_\text{Э}(\mathrm{NaOH})\cdot M(\mathrm{NaOH})=1\cdot40=40\;\text{г/экв}.
	\]
	
	\item Реакция полной нейтрализации серной кислоты протекает по следующей схеме:
	\[
	\mathrm{H_2SO_4}+2\mathrm{NaOH}\longrightarrow\mathrm{Na_2SO_4}+2\mathrm{H_2O}.
	\]
	При этом число отданных кислотой катионов водорода $\mathrm{H^+}$ в ходе реакции составляет~$z=2$. Тогда
	\[
	f_\text{Э}(\mathrm{H_2SO_4})=\frac{1}{z}=\frac{1}{2}=0{,}5
	\]
	и
	\[
	M_\text{Э}(\mathrm{H_2SO_4})=f_\text{Э}(\mathrm{H_2SO_4})\cdot M(\mathrm{H_2SO_4})=0{,}5\cdot98=49\;\text{г/экв}.
	\]
\end{Enumerate}