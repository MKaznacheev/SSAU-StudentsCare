\subsection{4}

При температуре $T=18\;^\circ\text{C}=291\;\text{K}$ и давлении $P=122\;\text{кПа}=122000\;\text{Па}$ некоторый газ массой $m=30{,}3\;\text{г}=0{,}0303\;\text{кг}$ имеет объём~$V=15\;\text{л}=0{,}015\;\text{м}^3$. Найдём молярную массу газа по уравнению Клапейрона-Менделеева:
\[
PV=\frac{m}{M}\cdot RT\Rightarrow M=\frac{mRT}{PV}=\frac{0{,}0303\cdot8{,}314\cdot291}{122000\cdot0{,}015}\approx0{,}04\;\text{кг/моль}=40\;\text{г/моль}.
\]