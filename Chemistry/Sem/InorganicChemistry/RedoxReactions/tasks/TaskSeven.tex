\subsection{7}

Составим уравнение реакции
\[
\mathrm{KMnO_4}+\mathrm{H_2O_2}+\mathrm{NaOH}\longrightarrow\mathrm{MnO_4^{2-}}+\mathrm{O_2}.
\]
Полуреакции идут по следующим схемам:
\[
\begin{array}{r|l}
	2 & \mathrm{MnO_4^-}+e^-\longrightarrow\mathrm{MnO_4^{2-}} \\
	1 & \mathrm{H_2O_2}+2\mathrm{OH^-}-2e^-\longrightarrow\mathrm{O_2}+2\mathrm{H_2O}
\end{array}.
\]
Просуммируем полученное:
\[
2\mathrm{MnO_4^-}+\mathrm{H_2O_2}+2\mathrm{OH^-}\longrightarrow2\mathrm{MnO_4^{2-}}+\mathrm{O_2}+2\mathrm{H_2O}.
\]
Допишем недостающие ионы:
\[
2\mathrm{KMnO_4}+\mathrm{H_2O_2}+2\mathrm{NaOH}\longrightarrow\mathrm{K_2MnO_4}+\mathrm{Na_2MnO_4}+\mathrm{O_2}+2\mathrm{H_2O}.
\]