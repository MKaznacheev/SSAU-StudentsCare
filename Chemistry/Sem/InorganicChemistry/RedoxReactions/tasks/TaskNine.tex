\subsection{9}

Составим уравнение реакции
\[
\mathrm{K_2Cr_2O_7}+\mathrm{P}+\mathrm{NaOH}\longrightarrow\mathrm{[Cr(OH)_6]^{3-}}+\mathrm{PO_4^{3-}}.
\]
Полуреакции идут по следующим схемам:
\[
\begin{array}{r|l}
	5 & \mathrm{Cr_2O_7^{2-}}+7\mathrm{H_2O}+6e^-\longrightarrow2\mathrm{[Cr(OH)_6]^{3-}}+2\mathrm{OH^{-}} \\
	6 & \mathrm{P}+8\mathrm{OH^-}-5e^-\longrightarrow\mathrm{PO_4^{3-}}+4\mathrm{H_2O}
\end{array}.
\]
Просуммируем полученное:
\[
5\mathrm{Cr_2O_7^{2-}}+6\mathrm{P}+38\mathrm{OH^-}+11\mathrm{H_2O}\longrightarrow10\mathrm{[Cr(OH)_6]^{3-}}+6\mathrm{PO_4^{3-}}.
\]
Допишем недостающие ионы:
\[
5\mathrm{K_2Cr_2O_7}+6\mathrm{P}+38\mathrm{NaOH}+11\mathrm{H_2O}\longrightarrow10\mathrm{Na_3[Cr(OH)_6]}+2\mathrm{Na_3PO_4}+3\mathrm{K_3PO_4}+\mathrm{Na_2KPO_4}.
\]