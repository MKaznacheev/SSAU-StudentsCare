\subsection{14}

Составим уравнение реакции
\[
\mathrm{K_2Cr_2O_7}+\mathrm{P}+\mathrm{H_2O}\longrightarrow\mathrm{Cr(OH)_3}+\mathrm{PO_4^{3-}}.
\]
Полуреакции идут по следующим схемам:
\[
\begin{array}{r|l}
	5 & \mathrm{Cr_2O_7^{2-}}+8\mathrm{H^+}+6e^-\longrightarrow2\mathrm{Cr(OH)_3}+\mathrm{H_2O} \\
	6 & \mathrm{P}+4\mathrm{H_2O}-5e^-\longrightarrow\mathrm{PO_4^{3-}}+8\mathrm{H^+}
\end{array}.
\]
Просуммируем полученное:
\[
5\mathrm{Cr_2O_7^{2-}}+6\mathrm{P}+19\mathrm{H_2O}\longrightarrow10\mathrm{Cr(OH)_3}+6\mathrm{PO_4^{3-}}+8\mathrm{H^+}.
\]
Допишем недостающие ионы:
\[
5\mathrm{K_2Cr_2O_7}+6\mathrm{P}+19\mathrm{H_2O}\longrightarrow10\mathrm{Cr(OH)_3}+4\mathrm{K_2HPO_4}+2\mathrm{KH_2PO_4}.
\]