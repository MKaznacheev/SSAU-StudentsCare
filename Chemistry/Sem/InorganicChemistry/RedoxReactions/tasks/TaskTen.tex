\subsection{10}

Составим уравнение реакции
\[
\mathrm{K_2Cr_2O_7}+\mathrm{H_2O_2}+\mathrm{NaOH}\longrightarrow\mathrm{[Cr(OH)_6]^{3-}}+\mathrm{O_2}.
\]
Полуреакции идут по следующим схемам:
\[
\begin{array}{r|l}
	1 & \mathrm{Cr_2O_7^{2-}}+7\mathrm{H_2O}+6e^-\longrightarrow2\mathrm{[Cr(OH)_6]^{3-}}+2\mathrm{OH^{-}} \\
	3 & \mathrm{H_2O_2}+2\mathrm{OH^-}-2e^-\longrightarrow\mathrm{O_2}+2\mathrm{H_2O}
\end{array}.
\]
Просуммируем полученное:
\[
\mathrm{Cr_2O_7^{2-}}+3\mathrm{H_2O_2}+4\mathrm{OH^-}+\mathrm{H_2O}\longrightarrow2\mathrm{[Cr(OH)_6]^{3-}}+3\mathrm{O_2}.
\]
Допишем недостающие ионы:
\[
\mathrm{K_2Cr_2O_7}+3\mathrm{H_2O_2}+4\mathrm{NaOH}+\mathrm{H_2O}\longrightarrow\mathrm{Na_3[Cr(OH)_6]}+\mathrm{K_2Na[Cr(OH)_6]}+3\mathrm{O_2}.
\]