\subsection{3}

Составим уравнение реакции
\[
\mathrm{K_2Cr_2O_7}+\mathrm{FeSO_4}+\mathrm{H_2SO_4}\longrightarrow\mathrm{Cr^{3+}}+\mathrm{Fe^{3+}}.
\]
Полуреакции идут по следующим схемам:
\[
\begin{array}{r|l}
	1 & \mathrm{Cr_2O_7^{2-}}+14\mathrm{H^+}+6e^-\longrightarrow2\mathrm{Cr^{3+}}+7\mathrm{H_2O} \\
	6 & \mathrm{Fe^{2+}}-e^-\longrightarrow\mathrm{Fe^{3+}}
\end{array}.
\]
Просуммируем полученное:
\[
\mathrm{Cr_2O_7^{2-}}+6\mathrm{Fe^{2+}}+14\mathrm{H^+}\longrightarrow2\mathrm{Cr^{3+}}+6\mathrm{Fe^{3+}}+7\mathrm{H_2O}.
\]
Допишем недостающие ионы:
\[
\mathrm{K_2Cr_2O_7}+6\mathrm{FeSO_4}+7\mathrm{H_2SO_4}\longrightarrow\mathrm{Cr_2(SO_4)_3}+3\mathrm{Fe_2(SO_4)_3}+\mathrm{K_2SO_4}+7\mathrm{H_2O}.
\]