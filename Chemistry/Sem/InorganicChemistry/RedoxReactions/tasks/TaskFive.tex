\subsection{5}

Составим уравнение реакции
\[
\mathrm{K_2Cr_2O_7}+\mathrm{H_2O_2}+\mathrm{H_2SO_4}\longrightarrow\mathrm{Cr^{3+}}+\mathrm{O_2}.
\]
Полуреакции идут по следующим схемам:
\[
\begin{array}{r|l}
	1 & \mathrm{Cr_2O_7^{2-}}+14\mathrm{H^+}+6e^-\longrightarrow2\mathrm{Cr^{3+}}+7\mathrm{H_2O} \\
	3 & \mathrm{H_2O_2}-2e^-\longrightarrow\mathrm{O_2} +2\mathrm{H^+}
\end{array}.
\]
Просуммируем полученное:
\[
\mathrm{Cr_2O_7^{2-}}+3\mathrm{H_2O_2}+8\mathrm{H^+}\longrightarrow2\mathrm{Cr^{3+}}+3\mathrm{O_2} +7\mathrm{H_2O}.
\]
Допишем недостающие ионы:
\[
\mathrm{K_2Cr_2O_7}+3\mathrm{H_2O_2}+4\mathrm{H_2SO_4}\longrightarrow\mathrm{Cr_2(SO_4)_3}+\mathrm{K_2SO_4}+3\mathrm{O_2} +7\mathrm{H_2O}.
\]