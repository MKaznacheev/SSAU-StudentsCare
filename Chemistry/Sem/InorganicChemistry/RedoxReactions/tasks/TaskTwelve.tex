\subsection{12}

Составим уравнение реакции
\[
\mathrm{KMnO_4}+\mathrm{H_2O_2}\longrightarrow\mathrm{MnO_2}+\mathrm{O_2}.
\]
Полуреакции идут по следующим схемам:
\[
\begin{array}{r|l}
	2 & \mathrm{MnO_4^-}+2\mathrm{H_2O}+3e^-\longrightarrow\mathrm{MnO_2}+4\mathrm{OH^-} \\
	3 & \mathrm{H_2O_2}+2\mathrm{OH^-}-2e^-\longrightarrow\mathrm{O_2}+2\mathrm{H_2O}
\end{array}.
\]
Просуммируем полученное:
\[
2\mathrm{MnO_4^-}+3\mathrm{H_2O_2}\longrightarrow2\mathrm{MnO_2}+3\mathrm{O_2}+2\mathrm{H_2O}+2\mathrm{OH^-}.
\]
Допишем недостающие ионы:
\[
2\mathrm{KMnO_4}+3\mathrm{H_2O_2}\longrightarrow2\mathrm{MnO_2}+3\mathrm{O_2}+2\mathrm{H_2O}+2\mathrm{KOH}.
\]