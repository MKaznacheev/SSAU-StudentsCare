\subsection{13}

Составим уравнение реакции
\[
\mathrm{K_2Cr_2O_7}+\mathrm{FeSO_4}+\mathrm{H_2O}\longrightarrow\mathrm{Cr(OH)_3}+\mathrm{Fe^{3+}}.
\]
Полуреакции идут по следующим схемам:
\[
\begin{array}{r|l}
	1 & \mathrm{Cr_2O_7^{2-}}+8\mathrm{H^+}+6e^-\longrightarrow2\mathrm{Cr(OH)_3}+\mathrm{H_2O} \\
	6 & \mathrm{Fe^{2+}}-e^-\longrightarrow\mathrm{Fe^{3+}}
\end{array}.
\]
Просуммируем полученное:
\[
\mathrm{Cr_2O_7^{2-}}+6\mathrm{Fe^{2+}}+8\mathrm{H^+}\longrightarrow2\mathrm{Cr(OH)_3}+6\mathrm{Fe^{3+}}+\mathrm{H_2O}.
\]
Поскольку в левой части уравнения физически не может быть катионов водорода, допишем в левую и правую части уравнения по $8$ гидроксид-ионов:
\[
\mathrm{Cr_2O_7^{2-}}+6\mathrm{Fe^{2+}}+7\mathrm{H_2O}\longrightarrow2\mathrm{Cr(OH)_3}+6\mathrm{Fe^{3+}}+8\mathrm{OH^-}.
\]
Допишем недостающие ионы:
\[
\mathrm{K_2Cr_2O_7}+6\mathrm{FeSO_4}+7\mathrm{H_2O}\longrightarrow2\mathrm{Cr(OH)_3}+2\mathrm{Fe_2(SO_4)_3}+2\mathrm{KOH}+2\mathrm{Fe(OH)_3}.
\]