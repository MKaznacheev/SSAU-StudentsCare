\subsection{2}

Составим уравнение реакции
\[
\mathrm{KMnO_4}+\mathrm{P}+\mathrm{H_2SO_4}\longrightarrow\mathrm{Mn^{2+}}+\mathrm{H_3PO_4}.
\]
Полуреакции идут по следующим схемам:
\[
\begin{array}{r|l}
	1 & \mathrm{MnO_4^{-}}+8\mathrm{H^+}+5e^-\longrightarrow\mathrm{Mn^{2+}}+4\mathrm{H_2O} \\
	1 & \mathrm{P}+4\mathrm{H_2O}-5e^-\longrightarrow\mathrm{H_3PO_4}+5\mathrm{H^+}
\end{array}.
\]
Просуммируем полученное и умножим на два все коэффициенты для удобства:
\[
2\mathrm{MnO_4^{-}}+2\mathrm{P}+6\mathrm{H^+}\longrightarrow2\mathrm{Mn^{2+}}+2\mathrm{H_3PO_4}.
\]
Допишем недостающие ионы:
\[
2\mathrm{KMnO_4}+2\mathrm{P}+3\mathrm{H_2SO_4}\longrightarrow2\mathrm{MnSO_4}+\mathrm{K_2SO_4}+2\mathrm{H_3PO_4}.
\]