\subsection{8}

Найдём количество атомов водорода в воде и аммиаке, в случае
\[
V(\mathrm{H_2O})=V(\mathrm{NH_3})=10\;\text{л}=10000\;\text{мл}.
\]
Поскольку $\rho(\mathrm{H_2O})=1\;\text{г/мл}$, то
\[
m(\mathrm{H_2O})=\rho(\mathrm{H_2O})\cdot V(\mathrm{H_2O})=10000\;\text{г}.
\]
Тогда
\begin{multline*}
N_\mathrm{H_2O}(\mathrm{H})=2N(\mathrm{H_2O})=2\cdot\nu(\mathrm{H_2O})\cdot N_A=2\cdot\frac{m(\mathrm{H_2O})}{M(\mathrm{H_2O})}\cdot N_A= \\
=2\cdot\frac{10000}{18}\cdot6{,}02\cdot10^{23}\approx6688{,}9\cdot10^{23},
\end{multline*}
ввиду формулы
\[
\nu(\mathrm{H_2O})=\frac{m(\mathrm{H_2O})}{M(\mathrm{H_2O})}.
\]
Поскольку при стандартных условиях аммиак является газом, используем формулу
\[
\nu(\mathrm{NH_3})=\frac{V(\mathrm{NH_3})}{V_m}
\]
для получения числа атомов водорода:
\begin{multline*}
N_\mathrm{NH_3}(\mathrm{H})=3N(\mathrm{NH_3})=3\cdot\nu(\mathrm{NH_3})\cdot N_A=3\cdot\frac{V(\mathrm{NH_3})}{V_m}\cdot N_A= \\
=3\cdot\frac{10}{22{,}4}\cdot6{,}02\cdot10^{23}\approx8{,}1\cdot10^{23}.
\end{multline*}
Таким образом, число атомов водорода в $10\;\text{л}$ воды превышает число атомов водорода в том же объёме аммиака в
\[
n=\frac{N_\mathrm{H_2O}(\mathrm{H})}{N_\mathrm{NH_3}(\mathrm{H})}\approx\frac{6688{,}9\cdot10^{23}}{8{,}1\cdot10^{23}}\approx825{,}8
\]
раз.