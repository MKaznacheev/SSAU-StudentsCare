\subsection{9}

Масса молекулы серы составляет $m(\mathrm{S_n})=3{,}19\cdot10^{-22}\;\text{г}$. Вычислим массу одного атома серы, для определения брутто-формулы молекулы:
\[
m(\mathrm{S})=\nu(\mathrm{S})\cdot M(\mathrm{S})=\frac{N(\mathrm{S})}{N_A}\cdot M(\mathrm{S})=\frac{1}{6{,}02\cdot10^{23}}\cdot32\approx5{,}316\cdot10^{-23}\;\text{г},
\]
так как
\[
\nu(\mathrm{S})=\frac{N(\mathrm{S})}{N_A}.
\]
Количество атомов в молекуле равно
\[
n=\frac{m(\mathrm{S_n})}{m(\mathrm{S})}\approx\frac{3{,}19\cdot10^{-22}}{5{,}32\cdot10^{-23}}\approx6.
\]
То есть формула имеет вид $\mathrm{S_6}$.