\subsection{7}

Найдём количество молекул воды в капле объёмом $V(\mathrm{H_2O})=0{,}01\;\text{мл}$. Поскольку плотность воды близка к одному грамму на миллилитр, то её масса в нашем случае численно совпадает с объёмом: $m(\mathrm{H_2O})=0{,}01\;\text{г}$. Поскольку количество вещества вычисляется по формуле
\[
\nu(\mathrm{H_2O})=\frac{m(\mathrm{H_2O})}{M(\mathrm{H_2O})},
\]
то
\[
N(\mathrm{H_2O})=\nu(\mathrm{H_2O})\cdot N_A=\frac{m(\mathrm{H_2O})}{M(\mathrm{H_2O})}\cdot N_A=\frac{0{,}01}{18}\cdot6{,}02\cdot10^{23}\approx 3{,}344\cdot10^{20}.
\]