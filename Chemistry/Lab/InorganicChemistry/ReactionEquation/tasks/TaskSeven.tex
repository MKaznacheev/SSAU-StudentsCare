\subsection{7}

Реакция уксусной кислоты с магнием протекает следующими образом:
\[
\rm 2CH_3COOH+Mg\longrightarrow (CH_3COO)_2Mg+H_2.
\]
При этом,
\[
m(\mathrm{Mg})=4{,}8\;\text{г},\quad\nu(\mathrm{Mg})=\frac{m(\mathrm{Mg})}{M(\mathrm{Mg})}=\frac{4{,}8}{24}=0{,}2\;\text{моль}.
\]
Так же известно, что
\[
V_\text{р}(\mathrm{CH_3COOH})=50\;\text{мл},\quad\omega(\mathrm{CH_3COOH})=15\%,\quad\rho_\text{р}(\mathrm{CH_3COOH})=1{,}02\;\text{г/мл}.
\]
Тогда
\[
m_\text{р}(\mathrm{CH_3COOH})=\rho_\text{р}(\mathrm{CH_3COOH})\cdot V_\text{р}(\mathrm{CH_3COOH})=51\;\text{г}
\]
и
\[
m(\mathrm{CH_3COOH})=\frac{\omega(\mathrm{CH_3COOH})}{100}\cdot m_\text{р}(\mathrm{CH_3COOH})=7{,}65\;\text{г}.
\]
Значит,
\[
\nu(\mathrm{CH_3COOH})=\frac{m(\mathrm{CH_3COOH})}{M(\mathrm{CH_3COOH})}=\frac{7{,}65}{60}\approx0{,}13\;\text{моль}. 
\]
Как видно из уравнения реакции, кислоты должно быть вдвое больше, чем металла, а значит, в нашем случае, её не хватит.