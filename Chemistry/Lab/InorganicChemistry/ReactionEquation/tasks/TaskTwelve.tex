\subsection{12}

Реакция прокаливания гидросульфита натрия протекает следующими образом:
\[
\rm 2NaHSO3\longrightarrow Na_2SO_3+SO_2+H_2O.
\]
Известно, что $m(\mathrm{NaHSO_3})=15{,}6\;\text{г}$ и масса твёрдого остатка равна $m=11{,}5\;\text{г}$. Изначальное количество гидросульфита равно
\[
\nu(\mathrm{NaHSO_3})=\frac{m(\mathrm{NaHSO_3})}{M(\mathrm{NaHSO_3})}=\frac{15{,}6}{104}\approx0{,}15\;\text{моль},
\]
а количество прореагировавшего --- $\nu_\text{п}(\mathrm{NaHSO_3})$.
Как видно из уравнения реакции,
\[
\nu(\mathrm{Na_2SO_3})=\frac{\nu_\text{п}(\mathrm{NaHSO_3})}{2}.
\]
Тогда
\begin{multline*}
m=\bigl(\nu(\mathrm{NaHSO_3})-\nu_\text{п}(\mathrm{NaHSO_3})\bigr)\cdot M(\mathrm{NaHSO_3})+\nu(\mathrm{Na_2SO_3})\cdot M(\mathrm{Na_2SO_3})= \\
=\bigl(\nu(\mathrm{NaHSO_3})-\nu_\text{п}(\mathrm{NaHSO_3})\bigr)\cdot M(\mathrm{NaHSO_3})+\frac{\nu_\text{п}(\mathrm{NaHSO_3})}{2}\cdot M(\mathrm{Na_2SO_3}).
\end{multline*}
То есть
\[
\nu_\text{п}(\mathrm{NaHSO_3})=2\cdot\frac{m-\nu(\mathrm{NaHSO_3})\cdot M(\mathrm{NaHSO_3})}{M(\mathrm{Na_2SO_3})-2M(\mathrm{NaHSO_3})}=2\cdot\frac{11{,}5-0{,}15\cdot104}{126-2\cdot104}=0{,}1\;\text{моль}
\]
и
\[
m_\text{п}(\mathrm{NaHSO_3})=\nu_\text{п}(\mathrm{NaHSO_3})\cdot M(\mathrm{NaHSO_3})=10{,}4\;\text{г}.
\]
Значит, прореагировало
\[
\frac{m_\text{п}(\mathrm{NaHSO_3})}{m(\mathrm{NaHSO_3})}\cdot100\%\approx66{,}7\%
\]
гидросульфита натрия.