\subsection{11}

На первой стадии получают хлорид кальция:
\[
\rm CaCO_3+2HCl\longrightarrow CaCl_2+H_2O+CO_2,
\]
а на второй проводят электролиз расплава:
\[
\rm CaCl_2\longrightarrow Ca+Cl_2.
\]
Известно, что $m(\mathrm{CaCO_3})=120\;\text{г}$. Тогда
\[
\nu(\mathrm{CaCO_3})=\frac{m(\mathrm{CaCO_3})}{M(\mathrm{CaCO_3})}=\frac{120}{100}=1{,}2\;\text{моль}.
\]
Выход на каждой стадии составляет $90\%$. Как видно из уравнений реакций,
\[
\nu(\mathrm{Ca})=0{,}9\nu(\mathrm{CaCl_2})=0{,}9\bigl(0{,}9\nu(\mathrm{CaCO_3})\bigr)=0{,}972\;\text{моль}.
\]
Значит,
\[
m(\mathrm{Ca})=\nu(\mathrm{Ca})\cdot M(\mathrm{Ca})=0{,}972\cdot40=38{,}88\;\text{г}.
\]