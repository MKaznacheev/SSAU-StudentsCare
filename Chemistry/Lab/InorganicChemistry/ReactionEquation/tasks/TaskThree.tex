\subsection{3}

Реакция нагревания железа с концентрированной серной кислотой протекает следующими образом:
\[
\rm 2Fe+6H_2SO_4\longrightarrow Fe_2(SO_4)_3+6H_2O+3SO_2.
\]
Известно, что масса железа равна $m(\mathrm{Fe})=6{,}16\;\text{г}$. Тогда
\[
\nu(\mathrm{Fe})=\frac{m(\mathrm{Fe})}{M(\mathrm{Fe})}=\frac{6{,}16}{56}=0{,}11\;\text{моль}.
\]
Исходя из уравнения реакции,
\[
\nu(\mathrm{SO_2})=\frac{3}{2}\nu(\mathrm{Fe})=0{,}165\;\text{моль}.
\]
Реакция проходила при $T=300\;^\circ\text{C}=573\;\text{К}$ и $P=101300\;\text{Па}$. По уравнению Клапейрона-Менделеева,
\[
V(\mathrm{SO_2})=\frac{\nu(\mathrm{SO_2})\cdot RT}{P}\cdot10^3=\frac{0{,}165\cdot8{,}314\cdot573}{101300}\cdot10^3\approx7{,}76\;\text{л}.
\]