\subsection{2}

Реакция разложения пероксида водорода протекает следующими образом:
\[
\rm 2H_2O_2\longrightarrow 2H_2O+O_2.
\]
Известно, что масса раствора, содержащего пероксид водорода в количестве $\omega(\mathrm{H_2O_2})=3{,}4\%$, равна
\[
m_\text{р}=100\;\text{г}.
\]
Тогда масса пероксида равна
\[
m(\mathrm{H_2O_2})=\frac{\omega(\mathrm{H_2O_2})}{100}\cdot m_\text{р}=3{,}4\;\text{г}.
\]

При этом, в результате реакции образовался кислород объёмом $V(\mathrm{O_2})=0{,}56\;\text{л}$, то есть
\[
\nu(\mathrm{O_2})=\frac{V(\mathrm{O_2})}{V_m}=\frac{0{,}56}{22{,}4}=0{,}025\;\text{моль}.
\]
Исходя из уравнения реакции,
\[
\nu(\mathrm{H_2O_2})=2\nu(\mathrm{O_2})=0{,}05\;\text{моль}
\]
и
\[
m_\text{п}(\mathrm{H_2O_2})=\nu(\mathrm{H_2O_2})\cdot M(\mathrm{H_2O_2})=0{,}05\cdot34=1{,}7\;\text{г}.
\]
А значит, разложению подверглось
\[
\frac{m_\text{п}(\mathrm{H_2O_2})}{m(\mathrm{H_2O_2})}\cdot100\%=50\%
\]
от изначальной массы пероксида водорода.