\subsection{1}

Реакция между карбонатом натрия и соляной кислотой протекает следующими образом:
\[
\rm Na_2CO_3+2HCl\longrightarrow2NaCl+H_2O+CO_2.
\]
Известно, что масса смеси, содержащей примеси в количестве $\omega_\text{п}=15\%$, равна
\[
m_\text{с}=15\;\text{г}.
\]
Тогда
\[
m(\mathrm{Na_2CO_3})=\frac{\omega(\mathrm{Na_2CO_3})}{100}\cdot m_\text{с}=\frac{100-\omega_\text{п}}{100}\cdot m_\text{с}=12{,}75\;\text{г}.
\]
Исходя из уравнения реакции,
\[
\nu(\mathrm{CO_2})=\nu(\mathrm{Na_2CO_3})=\frac{m(\mathrm{Na_2CO_3})}{M(\mathrm{Na_2CO_3})}.
\]
Значит,
\[
V(\mathrm{CO_2})=\nu(\mathrm{CO_2})\cdot V_m=\frac{m(\mathrm{Na_2CO_3})}{M(\mathrm{Na_2CO_3})}\cdot V_m=\frac{12{,}75}{106}\cdot22{,}4\approx2{,}7\;\text{л}.
\]