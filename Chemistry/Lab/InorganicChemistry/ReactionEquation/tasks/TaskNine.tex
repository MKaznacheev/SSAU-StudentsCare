\subsection{9}

Известно, что смесь массой $m=40\;\text{г}$ содержит вещества в следующих долях:
\[
\omega(\mathrm{MgO})=30\%,\quad\omega(\mathrm{ZnO})=20\%,\quad\omega(\mathrm{BaO})=50\%.
\]
Тогда
\begin{gather*}
m(\mathrm{MgO})=\frac{\omega(\mathrm{MgO})}{100}\cdot m=12\;\text{г}, \\
m(\mathrm{ZnO})=\frac{\omega(\mathrm{ZnO})}{100}\cdot m=8\;\text{г}, \\
m(\mathrm{BaO})=\frac{\omega(\mathrm{BaO})}{100}\cdot m=20\;\text{г}
\end{gather*}
и
\begin{gather*}
\nu(\mathrm{MgO})=\frac{m(\mathrm{MgO})}{M(\mathrm{MgO})}=\frac{12}{40}=0{,}3\;\text{моль}, \\
\nu(\mathrm{ZnO})=\frac{m(\mathrm{ZnO})}{M(\mathrm{ZnO})}=\frac{8}{81}\approx0{,}1\;\text{моль}, \\
\nu(\mathrm{BaO})=\frac{m(\mathrm{BaO})}{M(\mathrm{BaO})}=\frac{20}{153}\approx0{,}13\;\text{моль}.
\end{gather*}
К этой смеси добавили серную кислоту:
\[
V_\text{р}(\mathrm{H_2SO_4})=600\;\text{мл},\quad\omega(\mathrm{H_2SO_4})=12\%,\quad\rho_\text{р}(\mathrm{H_2SO_4})=1{,}08\;\text{г/мл}.
\]
Так,
\[
m_\text{р}(\mathrm{H_2SO_4})=\rho_\text{р}(\mathrm{H_2SO_4})\cdot V_\text{р}(\mathrm{H_2SO_4})=648\;\text{г}
\]
и
\begin{gather*}
m(\mathrm{H_2SO_4})=\frac{\omega(\mathrm{H_2SO_4})}{100}\cdot m_\text{р}(\mathrm{H_2SO_4})=77{,}76\;\text{г}, \\
\nu(\mathrm{H_2SO_4})=\frac{m(\mathrm{H_2SO_4})}{M(\mathrm{H_2SO_4})}=\frac{77{,}76}{98}\approx0{,}79\;\text{моль}.
\end{gather*}
Всего произойдёт три реакции вида
\[
\rm MeO+H_2SO_4\longrightarrow MeSO_4+H_2O.
\]
Серная кислота взята в избытке, а значит реакции пройдут полностью. Тогда, как видно из уравнения реакции, количество образовавшейся в них воды равно
\[
\nu(\mathrm{H_2O})=\nu(\mathrm{MgO})+\nu(\mathrm{ZnO})+\nu(\mathrm{BaO})=0{,}53\;\text{моль}.
\]
Значит, полная масса воды в растворе равна
\begin{multline*}
m(\mathrm{H_2O})=\nu(\mathrm{H_2O})\cdot M(\mathrm{H_2O})+\bigl(m_\text{р}(\mathrm{H_2SO_4})-m(\mathrm{H_2SO_4})\bigr)= \\
=0{,}53\cdot18+648-77{,}76=579{,}78\;\text{г}.
\end{multline*}