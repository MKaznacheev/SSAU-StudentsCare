\subsection{10}

Известно, что $m(\mathrm{NaOH})=10\;\text{г}$ и $m(\mathrm{H_2S})=10\;\text{г}$. Тогда
\[
\nu(\mathrm{NaOH})=\frac{m(\mathrm{NaOH})}{M(\mathrm{NaOH})}=\frac{10}{40}=0{,}25\;\text{моль}
\] 
и
\[
\nu(\mathrm{H_2S})=\frac{m(\mathrm{H_2S})}{M(\mathrm{H_2S})}=\frac{10}{34}\approx0{,}29\;\text{моль}.
\]
Реакция гидроксида натрия с сероводородом может протекать следующим двумя путями:
\[
\rm 2NaOH+H_2S\longrightarrow Na_2S+2H_2O
\]
и
\[
\rm NaOH+H_2S\longrightarrow NaHS+H_2O.
\]
Как видно из уравнений реакции, в нашем случае образуется кислая соль:
\[
\nu(\mathrm{NaHS})=\nu(\mathrm{NaOH})=0{,}25\;\text{моль}.
\]