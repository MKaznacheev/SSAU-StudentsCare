\subsection{5}

Реакция синтеза хлороводорода протекает следующими образом:
\[
\rm H_2+Cl_2\longrightarrow 2HCl.
\]
При этом водород взят с избытком по отношению к необходимому объёму:
\[
V(\mathrm{H_2})=1{,}1V_\text{н}(\mathrm{H_2}).
\]
Как видно из уравнения реакции,
\[
V_\text{н}(\mathrm{H_2})=V(\mathrm{Cl_2})\Rightarrow V(\mathrm{H_2})=1{,}1V(\mathrm{Cl_2}).
\]
При этом
\[
m(\mathrm{H_2})=\nu(\mathrm{H_2})\cdot M(\mathrm{H_2})=\frac{V(\mathrm{H_2})}{V_m}\cdot M(\mathrm{H_2})=\frac{1{,}1V(\mathrm{Cl_2})}{V_m}\cdot M(\mathrm{H_2}).
\]
и
\[
m(\mathrm{Cl_2})=\nu(\mathrm{Cl_2})\cdot M(\mathrm{Cl_2})=\frac{V(\mathrm{Cl_2})}{V_m}\cdot M(\mathrm{Cl_2}).
\]
Значит,
\begin{multline*}
\omega(\mathrm{H_2})=\frac{m(\mathrm{H_2})}{m(\mathrm{H_2})+m(\mathrm{Cl_2})}\cdot100\%= \\
=\cfrac{\cfrac{1{,}1V(\mathrm{Cl_2})}{V_m}\cdot M(\mathrm{H_2})}{\cfrac{1{,}1V(\mathrm{Cl_2})}{V_m}\cdot M(\mathrm{H_2})+\cfrac{V(\mathrm{Cl_2})}{V_m}\cdot M(\mathrm{Cl_2})}\cdot100\%= \\
=\frac{1{,}1M(\mathrm{H_2})}{1{,}1(\mathrm{H_2})+M(\mathrm{Cl_2})}\cdot100\%=\frac{1{,}1\cdot2}{1{,}1\cdot2+71}\cdot100\%\approx3\%.
\end{multline*}