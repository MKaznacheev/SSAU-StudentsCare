\subsection{4}

Реакция нитрата серебра с хлоридом натрия протекает следующими образом:
\[
\rm AgNO_3+HCl\longrightarrow AgCl+HNO_3.
\]
Известно, что
\[
m_\text{р}(\mathrm{AgNO_3})=250\;\text{г},\quad\omega(\mathrm{AgNO_3})=12\%
\]
и
\[
m_\text{р}(\mathrm{HCl})=300\;\text{г},\quad\omega(\mathrm{HCl})=4\%.
\]
Тогда
\[
m(\mathrm{AgNO_3})=\frac{\omega(\mathrm{AgNO_3})}{100}\cdot m_\text{р}(\mathrm{AgNO_3})=30\;\text{г}
\]
и
\[
m(\mathrm{HCl})=\frac{\omega(\mathrm{HCl})}{100}\cdot m_\text{р}(\mathrm{HCl})=12\;\text{г}.
\]
При этом
\[
\nu(\mathrm{AgNO_3})=\frac{m(\mathrm{AgNO_3})}{M(\mathrm{AgNO_3})}=\frac{30}{170}\approx{0{,}18}\;\text{моль}
\]
и
\[
\nu(\mathrm{HCl})=\frac{m(\mathrm{HCl})}{M(\mathrm{HCl})}=\frac{12}{36{,}5}\approx{0{,}33}\;\text{моль}.
\]
Как видно из уравнения реакции и расчётов, хлороводород находится в избытке, а значит
\[
\nu(\mathrm{AgCl})=\nu(\mathrm{AgNO_3})\approx{0{,}18}\;\text{моль}
\]
и
\[
m(\mathrm{AgCl})=\nu(\mathrm{AgCl})\cdot M(\mathrm{AgCl})=0{,}18\cdot143{,}5\approx25{,}83\;\text{г}.
\]