\subsection{8}

Реакция сульфида железа (II) с соляной кислотой протекает следующими образом:
\[
\rm FeS+2HCl\longrightarrow FeCl_2+H_2S,
\]
а сероводорода с нитратом свинца ---
\[
\rm H_2S+Pb(NO_3)_2\longrightarrow PbS+2HNO_3.
\]
Известно, что
\[
m_\text{р}\bigl(\mathrm{Pb(NO_3)_2}\bigr)=22{,}7\;\text{г},\quad\omega\bigl(\mathrm{Pb(NO_3)_2}\bigr)=10\%.
\]
Тогда
\[
m\bigl(\mathrm{Pb(NO_3)_2}\bigr)=\frac{\omega\bigl(\mathrm{Pb(NO_3)_2}\bigr)}{100}\cdot m_\text{р}\bigl(\mathrm{Pb(NO_3)_2}\bigr)=2{,}27\;\text{г}
\]
и
\[
\nu\bigl(\mathrm{Pb(NO_3)_2}\bigr)=\frac{m\bigl(\mathrm{Pb(NO_3)_2}\bigr)}{M\bigl(\mathrm{Pb(NO_3)_2}\bigr)}
=\frac{2{,}27}{331}\approx0{,}007\;\text{моль}.
\]
Как видно из уравнения реакции,
\[
\nu(\mathrm{H_2S})=\nu\bigl(\mathrm{Pb(NO_3)_2}\bigr)=0{,}007\;\text{моль}.
\]
При этом,
\[
\nu(\mathrm{S})=\nu(\mathrm{H_2S})=0{,}007\;\text{моль}
\]
и
\[
m(\mathrm{S})=\nu(\mathrm{S})\cdot M(\mathrm{S})=0{,}007\cdot32=0{,}224\;\text{г}.
\]
Тогда сталь, содержащая сульфид железа (II), массой $m=100\;\text{г}$ содержит
\[
\omega(\mathrm{S})=\frac{m(\mathrm{S})}{m}\cdot100\%=0{,}224\%
\]
серы.