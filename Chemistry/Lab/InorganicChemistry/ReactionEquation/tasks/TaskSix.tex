\subsection{6}

Реакция разложения перманганата калия протекает следующими образом:
\[
\rm 2KMnO_4\longrightarrow K_2MnO_4+MnO_2+O_2.
\]
Известно, что масса перманганата равна $m(\mathrm{KMnO_4})=20\;\text{г}$. Тогда
\[
\nu(\mathrm{KMnO_4})=\frac{m(\mathrm{KMnO_4})}{M(\mathrm{KMnO_4})}=\frac{20}{158}\approx0{,}127\;\text{моль}.
\]
Как видно из уравнения реакции,
\[
\nu(\mathrm{O_2})=\frac{\nu(\mathrm{KMnO_4})}{2}\approx0{,}064\;\text{моль}.
\]
Поскольку выход реакции равен $86\%$, то
\[
\nu_\text{н}(\mathrm{O_2})=0{,}86\cdot\nu(\mathrm{O_2})\approx0{,}055\;\text{моль}.
\]
Тогда
\[
m(\mathrm{O_2})=\nu_\text{н}(\mathrm{O_2})\cdot M(\mathrm{O_2})=0{,}055\cdot32=1{,}76\;\text{г}.
\]