\subsection{6}

В результате сгорания органического вещества массой $m=3{,}1\;\text{г}$ образовались: углекислый газ массой $m(\mathrm{CO_2})=8{,}8\;\text{г}$, вода массой $m(\mathrm{H_2O})=2{,}1\;\text{г}$ и азот массой $m(\mathrm{N_2})=0{,}47\;\text{г}$. Значит,
\begin{gather*}
\nu(\mathrm{C})=\nu(\mathrm{CO_2})=\frac{m(\mathrm{CO_2})}{M(\mathrm{CO_2})}=\frac{8{,}8}{44}=0{,}2\;\text{моль}, \\
\nu(\mathrm{H})=2\nu(\mathrm{H_2O})=\frac{2m(\mathrm{H_2O})}{M(\mathrm{H_2O})}=\frac{2\cdot2{,}1}{18}\approx0{,}233\;\text{моль}, \\
\nu(\mathrm{N})=2\nu(\mathrm{N_2})=\frac{2m(\mathrm{N_2})}{M(\mathrm{N_2})}=\frac{2\cdot0{,}47}{28}\approx0{,}034\;\text{моль}.
\end{gather*}
Тогда
\begin{multline*}
m(\mathrm{O})=m-m(\mathrm{C})-m(\mathrm{H})-m(\mathrm{N})= \\
=m-\nu(\mathrm{C})\cdot M(\mathrm{C})-\nu(\mathrm{H})\cdot M(\mathrm{H})-\nu(\mathrm{N})\cdot M(\mathrm{N})= \\
=3{,}1-0{,}2\cdot12-0{,}233\cdot1-0{,}034\cdot14\approx0\;\text{г}
\end{multline*}
и
\[
\nu(\mathrm{O})=\frac{m(\mathrm{O})}{M(\mathrm{O})}=0\;\text{моль},
\]
то есть вещество не содержит кислород. Очевидно, что
\[
N(\mathrm{C}):N(\mathrm{H}):N(\mathrm{N})=\nu(\mathrm{C}):\nu(\mathrm{H}):\nu(\mathrm{N})\approx6:7:1.
\]
Тогда формула вещества имеет вид $\mathrm{C_{6\alpha}H_{7\alpha}N_{\alpha}}$. 

При нормальных условиях пары искомого вещества объёмом $V_\text{п}=1\;\text{л}$ имеют массу $m_\text{п}=4{,}15\;\text{г}$. Значит, его молярная масса равна
\[
M=\frac{m_\text{п}}{\nu_\text{п}}=\cfrac{m_\text{п}}{\cfrac{V_\text{п}}{V_m}}=\frac{m_\text{п}V_m}{V_\text{п}}=\frac{4{,}15\cdot22{,}4}{1}\approx93\;\text{г/моль}.
\]
Значит,
\[
6\alpha\cdot M(\mathrm{C})+7\alpha\cdot M(\mathrm{H})+\alpha\cdot M(\mathrm{N})=M\LR93\alpha=93\LR\alpha=1.
\]
В таком случае, формула вещества имеет вид $\mathrm{C_6H_7N}$. 