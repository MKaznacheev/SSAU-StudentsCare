\subsection{13}

Массовая доля элемента в его сульфиде составляет $\omega(\text{Э})=27{,}93\%$. Очевидно, что
\[
\omega(\mathrm{S})=100\%-\omega(\text{Э})=100-27{,}93=72{,}07\%.
\]
Очевидно, искомое вещество имеет формулу $\mathrm{\text{Э}_x^{z}S_y^{-2}}$. Аналогично задаче 11,
\[
M(\text{Э})=\frac{\omega(\text{Э})\cdot M(\mathrm{S})}{\omega(\mathrm{S})}\cdot\frac{z}{2},
\]
или, в числовом приближении,
\[
M(\text{Э})=\frac{27{,}93\cdot32}{72{,}07}\cdot\frac{z}{2}\approx6{,}201z.
\]

Так же, как и в задаче 11 придётся действовать перебором. Допустимыми являются значения $z=2$ и $z=5$, которым соответствуют элементы $\mathrm{C}$ и $\mathrm{P}$ (сульфиды $\mathrm{CS}$ и $\mathrm{P_2O_5}$).