\subsection{1}

Известно, что масляная кислота содержит углерод, кислород и водород, при том их массовое содержание выражается следующими значениями:
\[
\omega(\mathrm{C})=54{,}5\%,\quad\omega(\mathrm{O})=36{,}4\%,\quad\omega(\mathrm{H})=9{,}1\%.
\]
Пусть $m$ --- масса кислоты. Тогда
\begin{multline*}
\nu(\mathrm{C}):\nu(\mathrm{O}):\nu(\mathrm{H})=\frac{m(\mathrm{C})}{M(\mathrm{C})}:\frac{m(\mathrm{O})}{M(\mathrm{O})}:\frac{m(\mathrm{H})}{M(\mathrm{H})}= \\
=\frac{\omega(\mathrm{C})\cdot m}{M(\mathrm{C})}:\frac{\omega(\mathrm{O})\cdot m}{M(\mathrm{O})}:\frac{\omega(\mathrm{H})\cdot m}{M(\mathrm{H})}=\frac{\omega(\mathrm{C})}{M(\mathrm{C})}:\frac{\omega(\mathrm{O})}{M(\mathrm{O})}:\frac{\omega(\mathrm{H})}{M(\mathrm{H})}.
\end{multline*}
Очевидно, с другой стороны, что
\[
\nu(\mathrm{C}):\nu(\mathrm{O}):\nu(\mathrm{H})=\frac{N(\mathrm{C})}{N_A}:\frac{N(\mathrm{O})}{N_A}:\frac{N(\mathrm{H})}{N_A}=N(\mathrm{C}):N(\mathrm{O}):N(\mathrm{H}).
\]
То есть
\[
N(\mathrm{C}):N(\mathrm{O}):N(\mathrm{H})=\frac{\omega(\mathrm{C})}{M(\mathrm{C})}:\frac{\omega(\mathrm{O})}{M(\mathrm{O})}:\frac{\omega(\mathrm{H})}{M(\mathrm{H})}=\frac{54{,}5}{12}:\frac{36{,}4}{16}:\frac{9{,}1}{1}.
\]
Удобно действовать следующим образом: умножить каждую дробь на наименьшее общее кратное их знаменателей, а затем разделить каждое полученное число на наименьшее из них. В нашем случае, а именно при числах
\[
1,\quad12=2^2\cdot3,\quad16=2^4,
\]
наименьшее общее кратное вычисляется, как
\[
\Nok(1,12,16)=2^4\cdot3=48.
\]
Тогда~отношение количеств атомов в молекуле масляной кислоты $N(\mathrm{C}):N(\mathrm{O}):N(\mathrm{H})$ равно
\[
48\cdot\frac{54{,}5}{12}:48\cdot\frac{36{,}4}{16}:48\cdot9{,}1=218:109{,}2:436{,}8=\frac{218}{109{,}2}:\frac{109{,}2}{109{,}2}:\frac{436{,}8}{109{,}2}\approx2:1:4.
\]
Значит, молекулярная формула кислоты имеет вид
\[
\mathrm{C_{2\alpha}H_{4\alpha}O_{\alpha}},
\]
где $\alpha$ --- некоторый натуральный коэффициент.

Плотность паров кислоты по водороду составляет $D_{\mathrm{H_2}}=44$. В таком случае, её молярная масса равна
\[
M=M(\mathrm{H_2})\cdot D_{\mathrm{H_2}}=2\cdot44=88\;\text{г/моль}.
\]
Исходя из формулы вещества,
\[
2\alpha\cdot M(\mathrm{C})+4\alpha\cdot M(\mathrm{H})+\alpha\cdot M(\mathrm{O})=M\LR 44\alpha=88\LR\alpha=2.
\]
Таким образом, молекулярная формула масляной кислоты принимает вид $\mathrm{C_4H_8O_2}$.