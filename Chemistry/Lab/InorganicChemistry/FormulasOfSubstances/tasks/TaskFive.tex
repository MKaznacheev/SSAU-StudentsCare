\subsection{5}

Некоторое вещество состоит только из атомов азота и кислорода. При этом в его навеске на $m(\mathrm{N})=7\;\text{г}$ приходится $m(\mathrm{O})=4\;\text{г}$. Значит,
\[
\nu(\mathrm{N})=\frac{m(\mathrm{N})}{M(\mathrm{N})}=\frac{7}{14}=0{,}5\;\text{моль}
\]
и
\[
\nu(\mathrm{O})=\frac{m(\mathrm{O})}{M(\mathrm{O})}=\frac{4}{16}=0{,}25\;\text{моль}.
\]

Очевидно, что
\[
N(\mathrm{N}):N(\mathrm{O})=\nu(\mathrm{N}):\nu(\mathrm{O})=2:1.
\]
Значит, простейшая формула вещества выглядит, как $\mathrm{N_2O}$.