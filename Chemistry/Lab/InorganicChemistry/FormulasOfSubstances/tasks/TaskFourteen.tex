\subsection{14}

Массовая доля остатка серной кислоты в сульфате составляет $\omega\bigl(\mathrm{(SO_4)^{-2}}\bigr)=61{,}8\%$. Очевидно, что
\[
\omega(\text{Э})=100\%-\omega\left(\mathrm{(SO_4)^{-2}}\right)=100-61{,}8=38{,}2\%.
\]
Очевидно, искомое вещество имеет формулу $\mathrm{\text{Э}_x^{z}(SO_4)_y^{-2}}$. Аналогично задаче 11,
\[
M(\text{Э})=\frac{\omega(\text{Э})\cdot M\bigl(\mathrm{(SO_4)^{-2}}\bigr)}{\omega\bigl(\mathrm{(SO_4)^{-2}}\bigr)}\cdot\frac{z}{2},
\]
или, в числовом приближении,
\[
M(\text{Э})=\frac{38{,}2\cdot96}{61{,}8}\cdot\frac{z}{2}\approx 29{,}670z.
\]

Так же, как и в задаче 11 придётся действовать перебором. Допустимыми являются значения $z=2$, $z=3$, $z=4$, которым соответствуют элементы $\mathrm{Co}$, $\mathrm{Y}$, $\mathrm{Sn}$ \big(сульфаты $\mathrm{CoSO_4}$, $\mathrm{Y_2(SO_4)_3}$, $\mathrm{Sn(SO_4)_2}$\big).