\subsection{10}

В результате сгорания навески неизвестного вещества массой $m=0{,}7\;\text{г}$ образовался углекислый газ и вода в количествах
\[
\nu(\mathrm{CO_2})=\nu(\mathrm{H_2O})=0{,}05\;\text{моль}.
\]
Очевидно, что исходное вещество содержало углерод и водород в следующих количествах:
\[
\nu(\mathrm{C})=\nu(\mathrm{CO_2})=0{,}05\;\text{моль}
\]
и
\[
\nu(\mathrm{H})=2\nu(\mathrm{H_2O})=0{,}1\;\text{моль}.
\]
Проверим, содержит ли оно кислород:
\[
m(\mathrm{O})=m-m(\mathrm{C})-m(\mathrm{H})=m-\nu(\mathrm{C})\cdot M(\mathrm{C})-\nu(\mathrm{H})\cdot M(\mathrm{H})=0{,}7-0{,}05\cdot12-0{,}1\cdot1=0,
\]
а равно кислород оно не содержит. То количественное отношение между атомами в веществе выглядит следующим образом:
\[
N(\mathrm{C}):N(\mathrm{H})=\nu(\mathrm{C}):\nu(\mathrm{H})=1:2.
\]
То есть формула вещества принимает вид $\mathrm{C_\alpha H_{2\alpha}}$.

Известно, что при нормальных условиях пары искомого вещества объёмом $V_\text{п}=32\;\text{мл}=0{,}032\;\text{л}$ имеют массу $m_\text{п}=0{,}1\;\text{г}$. В таком случае,
\[
\nu_\text{п}=\frac{V_\text{п}}{V_m},
\]
а значит его молярная масса равна
\[
M=\frac{m_\text{п}}{\nu_\text{п}}=\cfrac{m_\text{п}}{\cfrac{V_\text{п}}{V_m}}=\frac{m_\text{п}V_m}{V_\text{п}}=\frac{0{,}1\cdot22{,}4}{0{,}032}=7\;\text{г/моль}.
\]
Тогда
\[
\alpha\cdot M(\mathrm{C})+2\alpha\cdot M(\mathrm{H})=70\LR14\alpha=70\LR\alpha=5,
\]
а значит истинная формула вещества представлена в виде $\mathrm{C_5H_{10}}$.