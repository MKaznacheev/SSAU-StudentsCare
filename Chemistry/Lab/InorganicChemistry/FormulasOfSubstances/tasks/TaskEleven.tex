\subsection{11}

Массовая доля элемента в его оксиде составляет $\omega(\text{Э})=74{,}82\%$. Очевидно, что
\[
\omega(\mathrm{O})=100\%-\omega(\text{Э})=100-74{,}82=25{,}18\%.
\]
Очевидно, искомое вещество имеет формулу $\mathrm{\text{Э}_x^{z}O_y^{-2}}$. Ввиду электронейтральности частицы истинно соотношение
\[
xz+(-2y)=0\LR \frac{y}{x}=\frac{z}{2}. 
\]

Как было показано в первой задаче,
\[
\frac{x}{y}=\cfrac{\cfrac{\omega(\text{Э})}{M(\text{Э})}}{\cfrac{\omega(\mathrm{O})}{M(\mathrm{O})}}=\frac{\omega(\text{Э})\cdot M(\mathrm{O})}{\omega(\mathrm{O})\cdot M(\text{Э})}\LR M(\text{Э})=\frac{\omega(\text{Э})\cdot M(\mathrm{O})}{\omega(\mathrm{O})}\cdot\frac{y}{x}=\frac{\omega(\text{Э})\cdot M(\mathrm{O})}{\omega(\mathrm{O})}\cdot\frac{z}{2},
\]
или, в числовом приближении,
\[
M(\text{Э})=\frac{74{,}82\cdot16}{25{,}18}\cdot\frac{z}{2}\approx23{,}771z.
\]

Как вы можете видеть, математически задача не имеет однозначного решения, поэтому действовать придётся перебором. Придавая неизвестной $z$ различные натуральные значения (в разумных пределах), обнаружим, что значению $z=2$ соответствует элемент $\mathrm{Ti}$ (оксидом будет являться $\mathrm{TiO}$), а значению $z=8$ --- $\mathrm{Os}$ (оксид $\mathrm{OsO_4}$). Если при прочих значениях $z$ имеет смысл говорить об элементе, с молярной массой $23{,}771z$, то оксид в данной степени окисления для него не сущетсвует (как при $z=1$ не существует $\mathrm{Mg_2O}$).