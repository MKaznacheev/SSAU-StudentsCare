\subsection{4}

В результате сгорания органического вещества массой $m=1{,}88\;\text{г}$ образуется углекислый газ массой $m(\mathrm{CO_2})=0{,}88\;\text{г}$. При этом оно содержит бром, который может быть целиком переведён в бромид серебра массой $m(\mathrm{AgBr})=3{,}76\;\text{г}$. Значит,
\[
\nu(\mathrm{C})=\nu(\mathrm{CO_2})=\frac{m(\mathrm{CO_2})}{M(\mathrm{CO_2})}=\frac{0{,}88}{44}=0{,}02\;\text{моль}
\]
и
\[
\nu(\mathrm{Br})=\nu(\mathrm{AgBr})=\frac{m(\mathrm{AgBr})}{M(\mathrm{AgBr})}=\frac{3{,}76}{188}=0{,}02\;\text{моль}.
\]
Значит,
\begin{multline*}
m(\mathrm{H})=m-m(\mathrm{C})-m(\mathrm{Br})=m-\nu(\mathrm{C})\cdot M(\mathrm{C})-\nu(\mathrm{Br})\cdot M(\mathrm{Br})= \\
=1{,}88-0{,}02\cdot12-0{,}02\cdot80=0{,}04\;\text{г}
\end{multline*}
и
\[
\nu(\mathrm{H})=\frac{m(\mathrm{H})}{M(\mathrm{H})}=\frac{0{,}04}{1}=0{,}04\;\text{моль}.
\]
Как видно из задачи 1,
\[
N(\mathrm{C}):N(\mathrm{H}):N(\mathrm{Br})=\nu(\mathrm{C}):\nu(\mathrm{H}):\nu(\mathrm{Br})=1:2:1
\]
и формула вещества выглядит следующим образом: $\mathrm{C_\alpha H_{2\alpha}Br_\alpha}$.

Плотность паров этого вещества по водороду составляет $D_\mathrm{H_2}=94$, а его молярная масса ---
\[
M=M(\mathrm{H_2})\cdot D_\mathrm{H_2}=2\cdot94=188\;\text{г/моль}.
\]
Значит,
\[
\alpha\cdot M(\mathrm{C})+2\alpha\cdot M(\mathrm{H})+\alpha\cdot M(\mathrm{Br})=M\LR94\alpha=188\LR\alpha=2
\]
и формула принимает вид $\mathrm{C_2H_4Br_2}$.