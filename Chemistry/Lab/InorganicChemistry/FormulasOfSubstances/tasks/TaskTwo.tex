\subsection{2}

Известно, что вещество содержит углерод и водород, при том их массовое содержание выражается следующими значениями:
\[
\omega(\mathrm{C})=93{,}75\%,\quad\omega(\mathrm{H})=6{,}25\%.
\]
Аналогично предыдущей задаче, заключим, что
\[
N(\mathrm{C}):N(\mathrm{H})=\frac{\omega(\mathrm{C})}{M(\mathrm{C})}:\frac{\omega(\mathrm{H})}{M(\mathrm{H})}=\frac{93{,}75}{12}:\frac{6{,}25}{1}=93{,}75:75=1{,}25:1.
\]
Поскольку число $1{,}25$ значительно отличается от ближайшего целого, округлять его до единицы будет неправильно. Вместо этого умножим результат на $4$:
\[
N(\mathrm{C}):N(\mathrm{H})=5:4.
\]
Тогда формула примет вид $\mathrm{C_{5\alpha}H_{4\alpha}}$.

Плотность вещества по воздуху равна $D_\text{возд}=4{,}4$. Значит, его молярная масса равна
\[
M=M(\text{возд})\cdot D_\text{возд}=29\cdot4{,}4\approx128\;\text{г/моль}.
\]
Из формулы вещества видно, что
\[
5\alpha\cdot M(\mathrm{C})+4\alpha\cdot M(\mathrm{H})=M\LR64\alpha=128\LR\alpha=2.
\]
Так, формула вещества имеет вид $\mathrm{C_{10}H_8}$.