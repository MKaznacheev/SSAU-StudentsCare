\subsection{8}

При сгорании некоторого газа в атмосфере хлора образуется азот и хлороводород. При этом
\[
V(\mathrm{Cl_2}):V(\mathrm{N_2})=3:1.
\]
Заметим, что
\[
\nu(\mathrm{Cl_2}):\nu(\mathrm{N_2})=\frac{V(\mathrm{Cl_2})}{V_m}:\frac{V(\mathrm{N_2})}{V_m}=V(\mathrm{Cl_2}):V(\mathrm{N_2})=3:1.
\]
Тогда
\[
\nu(\mathrm{Cl}):\nu(\mathrm{N})=2\nu(\mathrm{Cl_2}):2\nu(\mathrm{N_2})=\nu(\mathrm{Cl_2}):\nu(\mathrm{N_2})=3:1.
\]
При этом, очевидно,
\[
\nu(\mathrm{H})=\nu(\mathrm{HCl})=\nu(\mathrm{Cl}).
\]
Значит,
\[
\nu(\mathrm{N}):\nu(\mathrm{H})=\nu(\mathrm{N}):\nu(\mathrm{Cl})=1:3.
\]
То есть искомый газ --- $\mathrm{NH_3}$.