\subsection{7}

Для полного сгорания газа объёмом $V=30\;\text{л}$ требуется кислород в объёме $V(\mathrm{O_2})=45\;\text{л}$. При этом в качестве продуктов образуется сернистый газ объёмом $V(\mathrm{SO_2})=30\;\text{л}$ и вода в том же объёме: $V(\mathrm{H_2O})=30\;\text{л}$. Значит, исходный газ содержал серу, водород и кислород в следующих количествах:
\begin{gather*}
\nu(\mathrm{S})=\nu(\mathrm{SO_2})=\frac{V(\mathrm{SO_2})}{V_m}=\frac{30}{22{,}4}\approx1{,}339\;\text{моль}, \\
\nu(\mathrm{H})=2\nu(\mathrm{H_2O})=\frac{2V(\mathrm{H_2O})}{V_m}\approx2{,}679\;\text{моль}, \\
\nu(\mathrm{O})=2\nu(\mathrm{SO_2})+\nu(\mathrm{H_2O})-2\nu(\mathrm{O_2})=2\nu(\mathrm{S})+\frac{\nu(\mathrm{H})}{2}-\frac{2V(\mathrm{O_2})}{V_m}\approx0\;\text{моль}.
\end{gather*}
Значит,
\[
N(\mathrm{H}):N(\mathrm{S})=\nu(\mathrm{H}):\nu(\mathrm{S})=\frac{2V(\mathrm{H_2O})}{V_m}:\frac{V(\mathrm{SO_2})}{V_m}=2:1.
\]
В таком случае искомым газом является $\mathrm{H_2S}$.