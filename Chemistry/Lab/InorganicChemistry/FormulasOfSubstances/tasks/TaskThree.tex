\subsection{3}

В результате сгорания навески вещества, состоящего только из атомов углерода и водорода, массой $m=4{,}3\;\text{г}$ выделился углекислый газ массой $m(\mathrm{CO_2})=13{,}3\;\text{г}$. Очевидно, что количество атомов углерода в реактанте совпадает с количеством получившегося газа:
\[
\nu(\mathrm{C})=\nu(\mathrm{CO_2})=\frac{m(\mathrm{CO_2})}{M(\mathrm{CO_2})}=\frac{13{,}3}{44}\approx0{,}3\;\text{моль}.
\]
При этом
\[
m(\mathrm{H})=m-m(\mathrm{C})=m-\nu(\mathrm{C})\cdot M(\mathrm{C})=4{,}3-0{,}3\cdot12=0{,}7\;\text{г}.
\]
Тогда
\[
\nu(\mathrm{H})=\frac{m(\mathrm{H})}{M(\mathrm{H})}=\frac{0{,}7}{1}=0{,}7\;\text{моль}.
\]
Как видно из задачи 1,
\[
N(\mathrm{C}):N(\mathrm{H})=\nu(\mathrm{C}):\nu(\mathrm{H})=3:7.
\]
Тогда формула примет вид $\mathrm{C_{3\alpha}H_{7\alpha}}$.

Плотность углеводорода по водороду составляет $D_\mathrm{H_2}=43$. Тогда его молярная масса равна
\[
M=M(\mathrm{H_2})\cdot D_\mathrm{H_2}=2\cdot43=86\;\text{г/моль}.
\]
При этом
\[
3\alpha\cdot M(\mathrm{C})+7\alpha\cdot M(\mathrm{H})=M\LR43\alpha=86\LR\alpha=2.
\]
Тогда молекулярная формула вещества примет вид $\mathrm{C_6H_{14}}$.