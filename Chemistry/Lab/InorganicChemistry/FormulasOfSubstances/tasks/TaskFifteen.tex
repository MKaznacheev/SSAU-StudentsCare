\subsection{15}

Известно, что в некотором веществе (примем его количество за $\nu$) содержатся элементы в следующих мольных долях:
\[
\chi(\mathrm{Ag})=7{,}69\%,\quad\chi(\mathrm{N})=23{,}08\%,\quad\chi(\mathrm{H})=46{,}15\%,\quad\chi(\mathrm{O})=23{,}08\%.
\]
Тогда (см. задачу 1)
\begin{multline*}
N(\mathrm{Ag}):N(\mathrm{N}):N(\mathrm{H}):N(\mathrm{O})=\nu(\mathrm{Ag}):\nu(\mathrm{N}):\nu(\mathrm{H}):\nu(\mathrm{O})= \\
=\chi(\mathrm{Ag})\cdot\nu:\chi(\mathrm{N})\cdot\nu:\chi(\mathrm{H})\cdot\nu:\chi(\mathrm{O})\cdot\nu= \\
=\chi(\mathrm{Ag}):\chi(\mathrm{N}):\chi(\mathrm{H}):\chi(\mathrm{O})\approx1:3:6:3.
\end{multline*}
Тогда простейшая формула вещества имеет вид $\mathrm{AgN_3H_6O_3}$.