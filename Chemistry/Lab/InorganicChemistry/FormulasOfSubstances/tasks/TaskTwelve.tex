\subsection{12}

Массовая доля элемента в его хлориде составляет $\omega(\text{Э})=35{,}09\%$. Очевидно, что
\[
\omega(\mathrm{Cl})=100\%-\omega(\text{Э})=100-35{,}09=64{,}91\%.
\]
Очевидно, искомое вещество имеет формулу $\mathrm{\text{Э}^zCl_z^{-}}$. Аналогично задаче 11,
\[
\frac{1}{z}=\frac{\omega(\text{Э})\cdot M(\mathrm{Cl})}{\omega(\mathrm{Cl})\cdot M(\text{Э})}\LR M(\text{Э})=\frac{\omega(\text{Э})\cdot M(\mathrm{Cl})}{\omega(\mathrm{Cl})}\cdot z,
\]
или, в числовом приближении,
\[
M(\text{Э})=\frac{35{,}09\cdot35{,}5}{64{,}91}\cdot z\approx19{,}191z.
\]

Так же, как и в задаче 11 придётся действовать перебором. Допустимым является значение $z=5$, которому соответствует элемент $\mathrm{Mo}$ (образует хлорид $\mathrm{MoCl_5}$).