\subsection{9}

В результате сгорания некоторого вещества массой $m=34\;\text{г}$ образуется азот и вода объёмами $V(\mathrm{N_2})=22{,}4\;\text{л}$ и $V(\mathrm{H_2O})=54\;\text{мл}$ соответсвенно. Учитывая, что плотность воды равна $\rho(\mathrm{H_2O})=1\;\text{г/мл}$, вычислим её массу:
\[
m(\mathrm{H_2O})=\rho(\mathrm{H_2O})\cdot V(\mathrm{H_2O})=54\;\text{г}.
\]
Значит,
\[
\nu(\mathrm{H_2O})=\frac{m(\mathrm{H_2O})}{M(\mathrm{H_2O})}=\frac{54}{18}=3\;\text{моль}.
\]
Поскольку объём азота измерен при нормальных условиях, то
\[
\nu(\mathrm{N_2})=\frac{V(\mathrm{N_2})}{V_m}=\frac{22{,}4}{22{,}4}=1\;\text{моль}.
\]
Проверим, содержит ли исходное вещество кислород:
\begin{multline*}
m(\mathrm{O})=m-m(\mathrm{H})-m(\mathrm{N})=m-\nu(\mathrm{H})\cdot M(\mathrm{H})-\nu(\mathrm{N})\cdot M(\mathrm{N})= \\
=m-2\nu(\mathrm{H_2O})\cdot M(\mathrm{H})-2\nu(\mathrm{N_2})\cdot M(\mathrm{N})=34-2\cdot3\cdot1-2\cdot1\cdot14=0\;\text{г},
\end{multline*}
то есть кислорода нет в составе искомого реактанта. Тогда отношение числа атомов в его составе представлено следующим образом:
\[
N(\mathrm{N}):N(\mathrm{H})=\nu(\mathrm{N}):\nu(\mathrm{H})=2\nu(\mathrm{N_2}):2\nu(\mathrm{H_2O})=\nu(\mathrm{N_2}):\nu(\mathrm{H_2O})=1:3,
\]
а его формула имеет вид $\mathrm{N_\alpha H_{3\alpha}}$.

Плотность этого газа по воздуху составляет $D_\text{возд}=0{,}586$. Тогда его молярная масса равна
\[
M=D_\text{возд}\cdot M(\text{возд})=0{,}586\cdot29\approx17\;\text{г/моль}.
\]
Значит,
\[
\alpha\cdot M(\mathrm{N})+3\alpha\cdot M(\mathrm{H})=M\LR17\alpha=17\LR\alpha=1.
\]
То есть истинная формула вещества имеет вид $\mathrm{NH_3}$.