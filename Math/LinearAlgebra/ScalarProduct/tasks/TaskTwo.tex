\subsection{3.2.18}

Пусть дан вектор
\[
\vec a=\vec i+2\vec j-3\vec k,
\]
то есть
\[
\vec a=(1;2;-3).
\]
Известно, что вектор $\vec b$ коллинеарен вектору $\vec a$, а значит
\[
\vec b=\alpha\cdot\vec a.
\]
Тогда
\[
\vec b\cdot\vec a=(\alpha\cdot\vec a)\cdot\vec a=\alpha\cdot(\vec a\cdot\vec a)=\alpha\cdot a^2=\alpha\cdot\left(1^2+2^2+(-3)^2\right)=14\alpha.
\]
Поскольку
\[
\vec b\cdot\vec a=28,
\]
то
\[
14\alpha=28\LR\alpha=2.
\]
В таком случае,
\[
\vec b=\bigl(2\cdot1;2\cdot2;2\cdot(-3)\bigr)=(2;4;-6).
\]