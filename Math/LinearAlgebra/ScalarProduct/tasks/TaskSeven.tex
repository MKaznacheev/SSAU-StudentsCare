\subsection{3.2.26}

Известно, что $\vec a=(\sqrt{2};-3;-5)$, а так же, что некоторая ось составляет углы
\[
\alpha=45^\circ,\quad0<\beta<\frac{\pi}{2},\quad\gamma=60^\circ
\]
с осями абсцисс, ординат и аппликат соотвественно. Обозначим последнюю через~$L$, а её направляющий единичный вектор --- через  $\vec l$. Тогда
\[
\vec l=\cos\alpha\cdot\vec i+\cos\beta\cdot\vec j+\cos\gamma\cdot\vec k
\]
и
\[
l^2=\cos^2\alpha+\cos^2\beta+\cos^2\gamma\LR\beta=\arccos\pm\sqrt{l^2-\cos^2\alpha-\cos^2\gamma}\Rightarrow\beta=60^\circ.
\]
Значит,
\[
\text{ПР}_L\vec a=\frac{\left(\vec a,\vec l\right)}{l}=\frac{\sqrt{2}\cdot\cos45^\circ+(-3)\cdot\cos60^\circ+(-5)\cdot\cos60^\circ}{1}=-3.
\]