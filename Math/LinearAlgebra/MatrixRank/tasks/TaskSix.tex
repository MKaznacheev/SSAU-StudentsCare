\subsection{1.4.57}

Решим матричное уравнение:
\[
AX=B,
\]
где
\[
A=
\begin{pmatrix}
	1 & -2 & 3 \\
	2 & 3 & -1 \\
	0 & -2 & 1
\end{pmatrix}
,\quad B=
\begin{pmatrix}
	2 \\
	-1 \\
	3
\end{pmatrix}.
\]
Проверим, существует ли матрица, обратная к матрице $A$:
\[
\det(A)=
\begin{vmatrix}
	1 & -2 & 3 \\
	2 & 3 & -1 \\
	0 & -2 & 1
\end{vmatrix}
=-7.
\]
Поскольку $\det(A)\neq0$, то $A^{-1}$ существует. Найдём её:
\begin{multline*}
(A|E)=
\begin{pNiceArray}{ccc|ccc}
	1 & -2 & 3 & 1 & 0 & 0 \\
	2 & 3 & -1 & 0 & 1 & 0 \\
	0 & -2 & 1 & 0 & 0 & 1
\end{pNiceArray}
\sim
\begin{pNiceArray}{ccc|ccc}
	1 & -2 & 3 & 1 & 0 & 0 \\
	0 & 7 & -7 & -2 & 1 & 0 \\
	0 & -2 & 1 & 0 & 0 & 1
\end{pNiceArray}
\sim \\
\sim
\begin{pNiceArray}{ccc|ccc}
	1 & -2 & 3 & 1 & 0 & 0 \\
	0 & 14 & -14 & -4 & 2 & 0 \\
	0 & -14 & 7 & 0 & 0 & 7
\end{pNiceArray}
\sim
\begin{pNiceArray}{ccc|ccc}
	1 & -2 & 3 & 1 & 0 & 0 \\
	0 & 0 & -7 & -4 & 2 & 7 \\
	0 & -14 & 7 & 0 & 0 & 7
\end{pNiceArray}
\sim \\
\sim
\begin{pNiceArray}{ccc|ccc}
	1 & -2 & 3 & 1 & 0 & 0 \\
	0 & 0 & -7 & -4 & 2 & 7 \\
	0 & -14 & 0 & -4 & 2 & 14
\end{pNiceArray}
\sim
\begin{pNiceArray}{ccc|ccc}
	1 & -2 & 3 & 1 & 0 & 0 \\
	0 & 0 & 1 & \dfrac{4}{7} & -\dfrac{2}{7} & -1 \\
	0 & 1 & 0 & \dfrac{2}{7} & -\dfrac{1}{7} & -1
\end{pNiceArray}
\sim \\
\sim
\begin{pNiceArray}{ccc|ccc}
	1 & -2 & 3 & 1 & 0 & 0 \\
	0 & 1 & 0 & \dfrac{2}{7} & -\dfrac{1}{7} & -1 \\
	0 & 0 & 1 & \dfrac{4}{7} & -\dfrac{2}{7} & -1
\end{pNiceArray}
\sim
\begin{pNiceArray}{ccc|ccc}
	1 & 0 & 0 & -\dfrac{1}{7} & \dfrac{4}{7} & 1 \\
	0 & 1 & 0 & \dfrac{2}{7} & -\dfrac{1}{7} & -1 \\
	0 & 0 & 1 & \dfrac{4}{7} & -\dfrac{2}{7} & -1
\end{pNiceArray}
=\left(E|A^{-1}\right).
\end{multline*}
При этом последовательно были выполнены следующие преобразования\footnotemark:
\begin{gather*}
(2)\to(2)-2\cdot(1), \\
\left\{
\begin{NiceArray}{l}
	(2)\to2\cdot(2) \\
	(3)\to7\cdot(3)
\end{NiceArray}
\right.\;, \\
(2)\to(2)+(3), \\
(3)\to(3)+(2), \\
\left\{
\begin{NiceArray}{l}
	(2)\to-\dfrac{1}{7}\cdot(2) \\
	(3)\to-\dfrac{1}{14}\cdot(3)
\end{NiceArray}
\right.\;,
\end{gather*}

\footnotetext{Запись "$(n)\to(n)+a\cdot(m)$"{} следует понимать, как прибавление к строке под номером $n$ строки под номером $m$, умноженной на число $a$. Запись "$(m)\leftrightarrow(n)$"\ означает перемену мест строками $m$ и $n$ между собой.}

\begin{gather*}
(2)\leftrightarrow(3), \\
(1)\to(1)+2\cdot(2)-3\cdot(3).
\end{gather*}
Как мы видим,
\[
A^{-1}=
\begin{pNiceMatrix}
	-\dfrac{1}{7} & \dfrac{4}{7} & 1 \\
	\dfrac{2}{7} & -\dfrac{1}{7} & -1 \\
	\dfrac{4}{7} & -\dfrac{2}{7} & -1
\end{pNiceMatrix}.
\]
Тогда, аналогично предыдущей задаче,
\[
AX=B\LR A^{-1}AX=A^{-1}B\LR EX=X=A^{-1}B.
\]
То есть
\[
X=
\begin{pNiceMatrix}
	-\dfrac{1}{7} & \dfrac{4}{7} & 1 \\
	\dfrac{2}{7} & -\dfrac{1}{7} & -1 \\
	\dfrac{4}{7} & -\dfrac{2}{7} & -1
\end{pNiceMatrix}
\cdot
\begin{pmatrix}
	2 \\
	-1 \\
	3
\end{pmatrix}
=
\begin{pNiceMatrix}
	\dfrac{15}{7} \\
	-\dfrac{16}{7} \\
	-\dfrac{11}{7}
\end{pNiceMatrix}.
\]