\subsection{2.1.33}

Решим систему
\[
\left\{
\begin{NiceArray}{l}
	3x+2y=5 \\
	6x+4y=10
\end{NiceArray}
\right.\;,
\]
если это возможно. Расширенная матрица системы имеет вид
\[
\overline{A}=
\begin{pNiceArray}{cc|c}
	3 & 2 & 5 \\
	6 & 4 & 10
\end{pNiceArray}.
\]
Приведём её к ступенчатому виду. Для этого из второй строки вычтем удвоенную первую:
\[
\overline{A}\sim
\begin{pNiceArray}{cc|c}
	3 & 2 & 5 \\
	0 & 0 & 0
\end{pNiceArray}.
\]
Как мы видим, $\Rk(A)=\Rk\left(\overline{A}\right)=1$, где $A$ --- основная матрица системы. Тогда по теореме Кронекера-Капелли, исходная система совместна. Поскольку ранг меньше числа неизвестных ($1<2$), то система неопределённая. При этом $x$ будет считаться главной неизвестной, а $y$ --- свободной. Возвращаясь к преобразованной ранее матрице, получим уравнение, определяющее решение системы:
\[
3x+2y=5\LR x=\frac{5-2y}{3}.
\]
Общее решение будет выглядеть следующим образом:
\[
X=
\begin{pNiceMatrix}
	x=\dfrac{5-2y}{3} \\
	y
\end{pNiceMatrix}.
\]
В качестве примера частного решения, можно взять следующее (при $y=1$):
\[
X=
\begin{pNiceMatrix}
	1 \\
	1
\end{pNiceMatrix}.
\]