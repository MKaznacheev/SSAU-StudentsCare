\subsection{2.1.41}

Решим систему
\[
\left\{
\begin{NiceArray}{l}
	3x_1+2x_2+x_3=5 \\
	2x_1+3x_2+x_3=1 \\
	2x_1+x_2+3x_3=11 \\
	3x_1+4x_2-x_3=-5
\end{NiceArray}
\right.\;,
\]
если это возможно. Её расширенная матрица выглядит следующим образом:
\[
\overline{A}=
\begin{pNiceArray}{ccc|c}
	3 & 2 & 1 & 5 \\
	2 & 3 & 1 & 1 \\
	2 & 1 & 3 & 11 \\
	3 & 4 & -1 & -5
\end{pNiceArray}.
\]
Последовательно приведём её к ступенчатому виду следующими преобразованиями:
\begin{gather*}
\left\{
\begin{NiceArray}{l}
	(2)\to(2)-(3) \\
	(4)\to(4)-(1)
\end{NiceArray}
\right.\;, \\
\left\{
\begin{NiceArray}{l}
	(4)\to(4)-(2) \\
	(3)\to3\cdot(3)
\end{NiceArray}
\right.\;, \\
(3)\to(3)-2\cdot(1), \\
(2)\to(2)+2\cdot(3), \\
(2)\leftrightarrow(3).
\end{gather*}
Так,
\begin{multline*}
\overline{A}\sim
\begin{pNiceArray}{ccc|c}
	3 & 2 & 1 & 5 \\
	0 & 2 & -2 & -10 \\
	2 & 1 & 3 & 11 \\
	0 & 2 & -2 & -10
\end{pNiceArray}
\sim
\begin{pNiceArray}{ccc|c}
	3 & 2 & 1 & 5 \\
	0 & 2 & -2 & -10 \\
	6 & 3 & 9 & 33 \\
	0 & 0 & 0 & 0
\end{pNiceArray}
\sim
\begin{pNiceArray}{ccc|c}
	3 & 2 & 1 & 5 \\
	0 & 2 & -2 & -10 \\
	0 & -1 & 7 & 23 \\
	0 & 0 & 0 & 0
\end{pNiceArray}
\sim \\
\sim
\begin{pNiceArray}{ccc|c}
	3 & 2 & 1 & 5 \\
	0 & 0 & 12 & 36 \\
	0 & -1 & 7 & 23 \\
	0 & 0 & 0 & 0
\end{pNiceArray}
\sim
\begin{pNiceArray}{ccc|c}
	3 & 2 & 1 & 5 \\
	0 & -1 & 7 & 23 \\
	0 & 0 & 12 & 36 \\
	0 & 0 & 0 & 0
\end{pNiceArray}.
\end{multline*}
Как мы видим, $\Rk(A)=\Rk\left(\overline{A}\right)=3$, где $A$ --- основная матрица системы. Тогда по теореме Кронекера-Капелли, исходная система совместна. Поскольку ранг совпадает с числом неизвестных, то система определённая. Возвращаясь к преобразованной ранее матрице, получим систему уравнений, эквивалентную исходной:
\[
\left\{
\begin{NiceArray}{l}
	3x_1+2x_2+x_3=5 \\
	-x_2+7x_3=23 \\
	12x_3=36
\end{NiceArray}
\right.
\;\LR
\left\{
\begin{NiceArray}{l}
	x_1=\dfrac{5-2x_2-x_3}{3} \\
	x_2=7x_3-23 \\
	x_3=3
\end{NiceArray}
\right.
\;\LR
\left\{
\begin{NiceArray}{l}
	x_1=2 \\
	x_2=-2 \\
	x_3=3
\end{NiceArray}
\right.\;.
\]
Таким образом, система имеет однозначное решение:
\[
X=
\begin{pmatrix}
	2 \\
	-2 \\
	3
\end{pmatrix}.
\]