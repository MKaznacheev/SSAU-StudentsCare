\subsection{2.1.37}

Решим систему
\[
\left\{
\begin{NiceArray}{l}
	3x+4y+2z=8 \\
	2x-4y-3z=-1 \\
	x+5y+z=0
\end{NiceArray}
\right.\;,
\]
если это возможно. Её расширенная матрица выглядит следующим образом:
\[
\overline{A}=
\begin{pNiceArray}{ccc|c}
	3 & 4 & 2 & 8 \\
	2 & -4 & -3 & -1 \\
	1 & 5 & 1 & 0
\end{pNiceArray}.
\]
Последовательно приведём её к ступенчатому виду следующими преобразованиями:
\begin{gather*}
\left\{
\begin{NiceArray}{l}
	(1)\to(1)-3\cdot(3) \\
	(2)\to(2)-2\cdot(3)
\end{NiceArray}
\right.\;, \\
\left\{
\begin{NiceArray}{l}
	(1)\to14\cdot(1) \\
	(2)\to11\cdot(2)
\end{NiceArray}
\right.\;, \\
(1)\to(1)-(2), \\
(1)\leftrightarrow(3).
\end{gather*}
Так,
\begin{multline*}
\overline{A}\sim
\begin{pNiceArray}{ccc|c}
	0 & -11 & -1 & 8 \\
	0 & -14 & -5 & -1 \\
	1 & 5 & 1 & 0
\end{pNiceArray}
\sim
\begin{pNiceArray}{ccc|c}
	0 & -154 & -14 & 112 \\
	0 & -154 & -55 & -11 \\
	1 & 5 & 1 & 0
\end{pNiceArray}
\sim \\
\sim
\begin{pNiceArray}{ccc|c}
	0 & 0 & 41 & 123 \\
	0 & -154 & -55 & -11 \\
	1 & 5 & 1 & 0
\end{pNiceArray}
\sim
\begin{pNiceArray}{ccc|c}
	1 & 5 & 1 & 0 \\
	0 & -154 & -55 & -11 \\
	0 & 0 & 41 & 123
\end{pNiceArray}.
\end{multline*}
Как мы видим, $\Rk(A)=\Rk\left(\overline{A}\right)=3$, где $A$ --- основная матрица системы. Тогда по теореме Кронекера-Капелли, исходная система совместна. Поскольку ранг совпадает с числом неизвестных, то система определённая. Возвращаясь к преобразованной ранее матрице, получим систему уравнений, эквивалентную исходной:
\[
\left\{
\begin{NiceArray}{l}
	x+5y+z=0 \\
	-154y-55z=-11 \\
	41z=123
\end{NiceArray}
\right.
\;\LR
\left\{
\begin{NiceArray}{l}
	x=-5y-z \\
	y=\dfrac{1-5z}{14} \\
	z=3
\end{NiceArray}
\right.
\;\LR
\left\{
\begin{NiceArray}{l}
	x=2 \\
	y=-1 \\
	z=3
\end{NiceArray}
\right.\;.
\]
Таким образом, система имеет однозначное решение:
\[
X=
\begin{pmatrix}
	2 \\
	-1 \\
	3
\end{pmatrix}.
\]