\subsection{1.3.26}

Найдём ранг матрицы
\[
A=
\begin{pmatrix}
	1 & -2 & 3 & -4 & 4 \\
	0 & 1 & -1 & 1 & -3 \\
	1 & 3 & 0 & -3 & 1 \\
	0 & -7 & 3 & 1 & -3
\end{pmatrix}
\]
методом окаймляющих миноров. Начнём с элемента, стоящего на пересечении первой строки и первого столбца:
\[
M_1^1=
\begin{vmatrix}
	1
\end{vmatrix}
=1.
\]
Добавим вторую строку и второй столбец:
\[
M_{1,2}^{1,2}=
\begin{vmatrix}
	1 & -2 \\
	0 & 1
\end{vmatrix}
=1.
\]
Добавим третью строку и третий столбец:
\[
M_{1,2,3}^{1,2,3}=
\begin{vmatrix}
	1 & -2 & 3 \\
	0 & 1 & -1 \\
	1 & 3 & 0
\end{vmatrix}
=2.
\]
Добавим четвёртую строку и четвёртый столбец и найдём минор разложением по второй строке:
\[
\begin{split}
M_{1,2,3,4}^{1,2,3,4}=
\begin{vmatrix}
	1 & -2 & 3 & -4 \\
	0 & 1 & -1 & 1 \\
	1 & 3 & 0 & -3 \\
	0 & -7 & 3 & 1
\end{vmatrix}
= & \sum_{k=1}^4(-1)^{2+k}a_{2k}\overline{M}_k^2=(-1)^{2+1}\cdot0\cdot\overline{M}_1^2+ \\
& +(-1)^{2+2}\cdot1\cdot\overline{M}_2^2+(-1)^{2+3}\cdot(-1)\cdot\overline{M}_3^2+ \\
& +(-1)^{2+4}\cdot1\cdot\overline{M}_4^2=\overline{M}_2^2+\overline{M}_3^2+\overline{M}_4^2= \\
= &
\begin{vmatrix}
	1 & 3 & -4 \\
	1 & 0 & -3 \\
	0 & 3 & 1
\end{vmatrix}
+
\begin{vmatrix}
	1 & -2 & -4 \\
	1 & 3 & -3 \\
	0 & -7 & 1
\end{vmatrix}
+
\begin{vmatrix}
	1 & -2 & 3 \\
	1 & 3 & 0 \\
	0 & -7 & 3
\end{vmatrix}
= \\
= & -6+12-6=0.
\end{split}
\]
Заменим четвёртый столбец пятым и найдём минор разложением по третьей строке:
\[
\begin{split}
M_{1,2,3,5}^{1,2,3,4}=
\begin{vmatrix}
	1 & -2 & 3 & 4 \\
	0 & 1 & -1 & -3 \\
	1 & 3 & 0 & 1 \\
	0 & -7 & 3 & -3
\end{vmatrix}
= & \sum_{k=1}^4(-1)^{3+k}a_{3k}\overline{M}_k^3=(-1)^{3+1}\cdot1\cdot\overline{M}_1^3+ \\
& +(-1)^{3+2}\cdot3\cdot\overline{M}_2^3+(-1)^{3+3}\cdot0\cdot\overline{M}_3^3+ \\
& +(-1)^{3+4}\cdot1\cdot\overline{M}_4^3=\overline{M}_1^3-3\overline{M}_2^3-\overline{M}_4^3= \\
= &
\begin{vmatrix}
	-2 & 3 & 4 \\
	1 & -1 & -3 \\
	-7 & 3 & -3
\end{vmatrix}
-3
\begin{vmatrix}
	1 & 3 & 4 \\
	0 & -1 & -3 \\
	0 & 3 & -3
\end{vmatrix}
-
\begin{vmatrix}
	1 & -2 & 3 \\
	0 & 1 & -1 \\
	0 & -7 & 3
\end{vmatrix}
= \\
= & 32-3\cdot12+4=0.
\end{split}
\]
Таким образом, $\Rk(A)=3$, а минор $M_{1,2,3}^{1,2,3}$ является базисным.