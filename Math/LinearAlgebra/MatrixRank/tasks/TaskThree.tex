\subsection{1.3.25}

Найдём ранг матрицы
\[
A=
\begin{pmatrix}
	2 & -1 & 5 & 6 \\
	1 & 1 & 3 & 5 \\
	1& -5 & 1 & -3
\end{pmatrix}
\]
методом окаймляющих миноров. Начнём с элемента, стоящего на пересечении третьей строки и первого столбца:
\[
M_1^3=
\begin{vmatrix}
	1
\end{vmatrix}
=1.
\]
Добавим вторую строку и второй столбец:
\[
M_{1, 2}^{2,3}=
\begin{vmatrix}
	1 & 1 \\
	1 & -5
\end{vmatrix}
=-6.
\]
Добавим первую строку и третий столбец:
\[
M_{1, 2, 3}^{1, 2, 3}=
\begin{vmatrix}
	2 & -1 & 5 \\
	1 & 1 & 3 \\
	1& -5 & 1
\end{vmatrix}
=0.
\]
Третий столбец заменим четвёртым:
\[
M_{1, 2, 4}^{1, 2, 3}=
\begin{vmatrix}
	2 & -1 & 6 \\
	1 & 1 & 5 \\
	1& -5 & -3
\end{vmatrix}
=0.
\]
Таким образом, $\Rk(A)=2$, а минор $M_{1, 2}^{2,3}$ является базисным.