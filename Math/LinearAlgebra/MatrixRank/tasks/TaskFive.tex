\subsection{1.4.50}

Решим уравнение
\[
XA=O,
\]
где
\[
A=
\begin{pmatrix}
	1 & 2 \\
	3 & 4
\end{pmatrix}
,\quad O=
\begin{pmatrix}
	0 & 0 \\
	0 & 0
\end{pmatrix}.
\]
Проверим, существует ли матрица $A^{-1}$:
\[
\det(A)=
\begin{vmatrix}
	1 & 2 \\
	3 & 4
\end{vmatrix}
=-2.
\]
Поскольку $\det(A)\neq0$, то $A^{-1}$ существует. Тогда
\[
XA=O\LR XAA^{-1}=OA^{-1}\LR XE=O\LR X=O,
\]
поскольку для каждой матрицы $A$ порядка $n$ выполнены условия $AA^{-1}=E$, $AE=A$, $OA=O$. Таким образом,
\[
X=
\begin{pmatrix}
	0 & 0 \\
	0 & 0
\end{pmatrix}.
\]