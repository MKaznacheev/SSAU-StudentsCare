\subsection{2.1.32}

Решим систему
\[
\left\{
\begin{NiceArray}{l}
	x_1-x_2=1 \\
	2x_1-2x_2=5
\end{NiceArray}
\right.\;,
\]
если это возможно. Запишем основную и расширенную матрицы системы:
\[
A=
\begin{pmatrix}
	1 & -1 \\
	2 & -2
\end{pmatrix}
,\quad \overline{A}=
\begin{pNiceArray}{cc|c}
	1 & -1 & 1\\
	2 & -2 & 5
\end{pNiceArray}.
\]
Приведём последнюю к ступенчатому виду. Для этого вычтем из второй строки первую, умноженную на $2$:
\[
\overline{A}\sim
\begin{pNiceArray}{cc|c}
	1 & -1 & 1\\
	0 & 0 & 3
\end{pNiceArray}.
\]
Тем же преобразованием, но применительно к основной матрице, мы получили бы
\[
A\sim
\begin{pmatrix}
	1 & -1 \\
	0 & 0
\end{pmatrix}
\]
(это видно сразу, исходя из того, что основная матрица является подматрицей расширенной матрицы, и в дальнейшем поясняться не будет). Так, $\Rk(A)=1$ и $\Rk\left(\overline{A}\right)=2$, то есть $\Rk(A)\neq\Rk\left(\overline{A}\right)$. Тогда по теореме Кронекера-Капелли (а точнее по её отрицанию), исходная система несовместна, а равно не имеет решений.