\subsection{1.4.38}

Найдем матрицу, обратную к
\[
A=
\begin{pmatrix}
	1 & 1 & -1 \\
	8 & 3 & -6 \\
	-4 & -1 & 3
\end{pmatrix}
.
\]
Присоединим единичную матрицу:
\[
(A|E)=
\begin{pNiceArray}{ccc|ccc}
	1 & 1 & -1 & 1 & 0 & 0\\
	8 & 3 & -6 & 0 & 1 & 0 \\
	-4 & -1 & 3 & 0 & 0 & 1
\end{pNiceArray}.
\]
Ко второй строке прибавим удвоенную третью:
\[
(A|E)\sim
\begin{pNiceArray}{ccc|ccc}
	1 & 1 & -1 & 1 & 0 & 0\\
	0 & 1 & 0 & 0 & 1 & 2 \\
	-4 & -1 & 3 & 0 & 0 & 1
\end{pNiceArray}.
\]
К третьей строке прибавим первую, умноженную на $4$, и вычтем вторую, умноженную на $3$:
\[
(A|E)\sim
\begin{pNiceArray}{ccc|ccc}
	1 & 1 & -1 & 1 & 0 & 0\\
	0 & 1 & 0 & 0 & 1 & 2 \\
	0 & 0 & -1 & 4 & -3 & -5
\end{pNiceArray}.
\]
Третью строку умножим на $-1$ и прибавим к первой:
\[
(A|E)\sim
\begin{pNiceArray}{ccc|ccc}
	1 & 1 & 0 & -3 & 3 & 5\\
	0 & 1 & 0 & 0 & 1 & 2 \\
	0 & 0 & 1 & -4 & 3 & 5
\end{pNiceArray}.
\]
Наконец, из первой строки вычтем вторую:
\[
(A|E)\sim
\begin{pNiceArray}{ccc|ccc}
	1 & 0 & 0 & -3 & 2 & 3\\
	0 & 1 & 0 & 0 & 1 & 2 \\
	0 & 0 & 1 & -4 & 3 & 5
\end{pNiceArray}.
\]
Так, искомая матрица имеет вид
\[
A^{-1}=
\begin{pmatrix}
	-3 & 2 & 3\\
	0 & 1 & 2 \\
	-4 & 3 & 5
\end{pmatrix}.
\]