\subsection{1.1.53}

Найдём значение матричного многочлена $f(A)$, где
\[
f(X)=2X^2-3X+E,\quad A=
\begin{pmatrix}
	1 & 0 \\
	0 & -1
\end{pmatrix}
.
\]
Рассмотрим слагаемые выражения
\[
f(A)=2A^2-3A+E
\]
по отдельности. Начнём с первого:
\[
2A^2=2(A\cdot A)=2\cdot
\begin{pmatrix}
	1 & 0 \\
	0 & -1
\end{pmatrix}
\cdot
\begin{pmatrix}
	1 & 0 \\
	0 & -1
\end{pmatrix}
=2\cdot
\begin{pmatrix}
	1 & 0 \\
	0 & 1
\end{pmatrix}
=2E.
\]
Заметим, что
\[
f(A)=2E-3A+E=3E-3A=3(E-A).
\]
При этом
\[
E-A=
\begin{pmatrix}
	1 & 0 \\
	0 & 1
\end{pmatrix}
-
\begin{pmatrix}
	1 & 0 \\
	0 & -1
\end{pmatrix}
=
\begin{pmatrix}
	0 & 0 \\
	0 & 2
\end{pmatrix},
\]
а значит,
\[
f(A)=3\cdot
\begin{pmatrix}
	0 & 0 \\
	0 & 2
\end{pmatrix}
=
\begin{pmatrix}
	0 & 0 \\
	0 & 6
\end{pmatrix}
.
\]