\subsection{3.3.17}

Пусть $\vec a=(3;-1;2)$ и $\vec b=(-1;3;-1)$. Найдём вектор $\vec c$, такой, что $\vec c\perp\vec a$ и $\vec c\perp\vec b$. Для этого найдём векторное произведение векторов $\vec a$ и $\vec b$:
\[
\left[\vec a,\vec b\right]=
\begin{vNiceMatrix}
	\vec i & \vec j & \vec k \\
	3 & -1 & 2 \\
	-1 & 3 & -1
\end{vNiceMatrix}
=-5\vec i+\vec j+8\vec k.
\]
Длина найденного вектора будет равна
\[
l=\biggl|\left[\vec a,\vec b\right]\biggr|=\sqrt{(-5)^2+1^2+8^2}=3\sqrt{10}.
\]
Тогда
\[
\vec c=\pm\dfrac{\left[\vec a,\vec b\right]}{l}=
\left[
\begin{NiceArray}{l}
	-\dfrac{\sqrt{10}}{6}\vec i+\dfrac{\sqrt{10}}{30}\vec j+\dfrac{4\sqrt{10}}{15}\vec k \\
	\dfrac{\sqrt{10}}{6}\vec i-\dfrac{\sqrt{10}}{30}\vec j-\dfrac{4\sqrt{10}}{15}\vec k
\end{NiceArray}
\right.\;,
\]
то есть
\[
\vec c=
\left[
\begin{NiceArray}{l}
	\left(-\dfrac{\sqrt{10}}{6};\dfrac{\sqrt{10}}{30};\dfrac{4\sqrt{10}}{15}\right) \\
	\left(\dfrac{\sqrt{10}}{6};-\dfrac{\sqrt{10}}{30};-\dfrac{4\sqrt{10}}{15}\right)
\end{NiceArray}
\right.\;.
\]
При этом, только в первом случае векторы образуют правую тройку
\[
\left\{\vec a,\vec b,\vec c\right\}.
\]