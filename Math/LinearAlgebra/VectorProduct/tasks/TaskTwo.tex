\subsection{3.3.19}

Пусть
\[
\vec a=3\vec p+2\vec q,\quad\vec b=2\vec p-\vec q,
\]
а так же
\[
p=4,\quad q=3,\quad\left(\widehat{\vec p,\vec q}\right)=\frac{3\pi}{4}.
\]
Преобразуем следующее:
\begin{multline*}
\left[\vec a,\vec b\right]=\left[3\vec p+2\vec q,2\vec p-\vec q\,\right]=3\left[\vec p,2\vec p-\vec q\,\right]+2\left[\vec q,2\vec p-\vec q\,\right]= \\
=3\bigl(2\left[\vec p,\vec p\,\right]-\left[\vec p,\vec q\,\right]\bigr)+2\bigl(2\left[\vec q,\vec p\,\right]-\left[\vec q,\vec q\,\right]\bigr)= \\
=6\left[\vec p,\vec p\,\right]-3\left[\vec p,\vec q\,\right]+4\left[\vec q,\vec p\,\right]-2\left[\vec q,\vec q\,\right]=7\left[\vec q,\vec p\,\right].
\end{multline*}
Найдём площадь параллелограмма, построенного на векторах $\vec a$ и $\vec b$:
\[
S=\biggl|\left[\vec a,\vec b\right]\biggr|=7\bigl|\left[\vec q,\vec p\,\right]\bigr|=7qp\sin\left(\widehat{\vec p,\vec q}\right)=42\sqrt{2}.
\]