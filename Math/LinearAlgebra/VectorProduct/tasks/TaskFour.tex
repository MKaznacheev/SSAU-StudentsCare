\subsection{3.3.21}

Пусть $\vec a=(-4;-8;8)$ и $\vec b=(4;3;2)$. Тогда,
\begin{gather*}
a=\sqrt{(-4)^2+(-8)^2+8^2}=12, \\
b=\sqrt{4^2+3^2+2^2}=\sqrt{29}, \\
\left[\vec a,\vec b\right]=
\begin{vNiceMatrix}
	\vec i & \vec j & \vec k \\
	-4 & -8 & 8 \\
	4 & 3 & 2
\end{vNiceMatrix}
=-40\vec i+40\vec j+20\vec k, \\
S=\biggl|\left[\vec a,\vec b\right]\biggr|=\sqrt{(-40)^2+40^2+20^2}=60, \\
\sin\left(\widehat{\vec a,\vec b}\right)=\frac{\biggl|\left[\vec a,\vec b\right]\biggr|}{ab}=\frac{5\sqrt{29}}{29},
\end{gather*}
где $S$ --- площадь параллелограмма, построенного на векторах $\vec a$ и $\vec b$.